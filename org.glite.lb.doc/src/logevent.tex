\subsection{glite-lb-logevent}
\label{glite-lb-logevent}

Besides the API's \LB\ offers its users a simple command-line interface for
logging events. The command \verb'glite-lb-logevent' is used for this purpose.
However, it is intended for internal WMS debugging tests in the first place and
should not be used for common event logging because of possibility of confusing
\LB\ server job state automaton.

The command \verb'glite-lb-logevent' is a complex logging tool and the complete
list of parameters can be obtained using the \verb'-h' option. However,
the only legal user usage is for logging \verb'UserTag' and \verb'ChangeACL'
events. The following description is therefore concentrating only on options
dealing with these two events.

Command usage is:

\begin{verbatim}
    glite-lb-logevent [-h] [-p] [-c seq_code] \
        -j <dg_jobid> -s Application -e <event_name> [key=value ...]
\end{verbatim}

where

\begin{tabularx}{\textwidth}{lX}
\texttt{  -h  -{}-help} &           this help message\\
\texttt{  -p  -{}-priority} &       send a priority event\\
\texttt{  -c  -{}-sequence} &       event sequence code\\
\texttt{  -j  -{}-jobid} &          JobId\\
\texttt{  -e  -{}-event} &           select event type (see -e help)\\
\end{tabularx}

\medskip

Each event specified after \verb'-e' option has different sub-options enabling
to set event specific values.

Sub-options usable with \verb'UserTag' event are:


\begin{tabularx}{\textwidth}{lX}
\texttt{      -{}-name}  &          tag name\\
\texttt{      -{}-value} &          tag value\\
\end{tabularx}

\medskip

Sub-options usable with \verb'ChangeACL' event are:

\begin{tabularx}{\textwidth}{lX}
\texttt{      -{}-operation} &       operation requested to perform with ACL (add, remove)\\
\texttt{      -{}-permission} &      ACL permission to change (currently only READ)\\
\texttt{      -{}-permission\_type} & type of permission requested (0 = 'allow', 1 = 'deny')\\
\texttt{      -{}-user\_id} &         DN or VOMS parameter (in format VO:group)\\
\texttt{      -{}-user\_id\_type} &    type of information given in \verb'user_id' (DN or VOMS)\\
\end{tabularx}

\bigskip

To be able to use this command several environmental variables must be set
properly. User must specify where the event should be sent. This is address and
port of glite-lb-logd daemon. It is done using environmental variable
\verb'EDG_WL_LOG_DESTINATION' in a form \verb'address:port'.

Because user is allowed to change ACL or add user tags only for her jobs, paths
to valid X509 user credentials has to be set to authorise her. This is done
using two environmental variables \verb'EDG_WL_X509_KEY' and
\verb'EDG_WL_X509_CERT' in a form \verb'path_to_cred'.

