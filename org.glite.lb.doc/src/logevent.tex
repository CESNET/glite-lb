\subsection{glite-lb-logevent}
\label{glite-lb-logevent}

Besides the API's \LB\ offers its users a simple command-line interface for
logging events. The command \verb'glite-lb-logevent' is used for this purpose.
However, it is intended for internal WMS debugging tests in the first place and
should not be used for common event logging because of possibility of confusing
\LB\ server job state automaton.

The command \verb'glite-lb-logevent' is a complex logging tool and the complete
list of parameters can be obtained using the \verb'-h' option.  However,
the only legal user usage is for logging \verb'UserTag' and \verb'ChangeACL'
events. The following description is therefore concentrating only on options
dealing with these two events.

Command usage is:

\begin{verbatim}
    glite-lb-logevent [-h] [-p] [-c seq_code] \
        -j <dg_jobid> -s Application -e <event_name> [key=value ...]
\end{verbatim}

where

\begin{tabularx}{\textwidth}{lX}
\texttt{  -h  -{}-help} &           this help message\\
\texttt{  -p  -{}-priority} &       send a priority event\\
\texttt{  -c  -{}-sequence} &       event sequence code\\
\texttt{  -j  -{}-jobid} &          JobId\\
\texttt{  -e  -{}-event} &           select event type (see -e help)\\
\end{tabularx}

%\medskip

Each event specified after \verb'-e' option has different sub-options enabling
to set event specific values.

Address of local logger, where event is sent, must be specified by environment
variable \verb'GLITE_WMS_LOG_DESTINATION' in a form \verb'address:port'.

Because user is allowed to change ACL or add user tags only for her jobs, paths
to valid X509 user credentials has to be set to authorise her. This is done
using standard X509 environment variables \verb'X509_USER_KEY' and
\verb'X509_USER_CERT'.

For additional information see also manual page GLITE-LB-LOGEVENT(1).

\input log_usertag.tex

\input change_acl.tex
