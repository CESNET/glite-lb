%
%% Copyright (c) Members of the EGEE Collaboration. 2004-2010.
%% See http://www.eu-egee.org/partners for details on the copyright holders.
%% 
%% Licensed under the Apache License, Version 2.0 (the "License");
%% you may not use this file except in compliance with the License.
%% You may obtain a copy of the License at
%% 
%%     http://www.apache.org/licenses/LICENSE-2.0
%% 
%% Unless required by applicable law or agreed to in writing, software
%% distributed under the License is distributed on an "AS IS" BASIS,
%% WITHOUT WARRANTIES OR CONDITIONS OF ANY KIND, either express or implied.
%% See the License for the specific language governing permissions and
%% limitations under the License.
%
\subsection{glite-lb-logevent}
\label{glite-lb-logevent}

Besides the API's \LB\ offers its users a simple command-line interface for
logging events. The command \verb'glite-lb-logevent' is used for this purpose.
However, it is intended for internal WMS debugging tests in the first place and
should not be used for common event logging because of possibility of confusing
\LB\ server job state automaton.

The command \verb'glite-lb-logevent' is a complex logging tool and the complete
list of parameters can be obtained using the \verb'-h' option.  However,
the only legal user usage is for logging \verb'UserTag' and \verb'ChangeACL'
events. The following description is therefore concentrating only on options
dealing with these two events.

Command usage is:

\begin{verbatim}
    glite-lb-logevent [-h] [-p] [-c seq_code] 
        -j <dg_jobid> -s Application -e <event_name> [key=value ...]
\end{verbatim}

where

\begin{tabularx}{\textwidth}{lX}
\texttt{  -p  -{}-priority} &       send a priority event\\
\texttt{  -c  -{}-sequence} &       event sequence code\\
\texttt{  -j  -{}-jobid} &          JobId\\
\texttt{  -e  -{}-event} &           select event type (see -e help)\\
\end{tabularx}

%\medskip

Each event specified after \verb'-e' option has different sub-options enabling
to set event specific values.

Address of local-logger, daemon responsible for further message delivery, must 
be specified by environment
variable \verb'GLITE_WMS_LOG_DESTINATION' in a form \verb'address:port'.

Because user is allowed to change ACL or add user tags only for her jobs, paths
to valid X509 user credentials has to be set to authorise her. This is done
using standard X509 environment variables \verb'X509_USER_KEY' and
\verb'X509_USER_CERT'.

For additional information see also manual page glite-lb-logevent(1).

\input log_usertag.tex

\input change_acl.tex

\subsubsection{Example: Setting payload owner}
\label{e:payload_owner}

In order to change the owner of the payload (see also \ref{sec:job-authz}),
a pair of \LB events is used. In order for a job owner to specify a new
owner of the payload, the \verb'GrantPayloadOwnership' event is used, \eg:

\begin{verbatim}
glite-lb-logevent -e GrantPayloadOwnership -s UserInterface -j <job_id> \
        --payload_owner <subject_name>
\end{verbatim}

where

\begin{tabularx}{\textwidth}{>{\texttt}lX}
\verb'<job_id>'    & specifies the job to change access to\\
\verb'<subject_name_id>'   & specifies the X.509 subject name of the new payload user. \\
\end{tabularx}

The new payload owner confirm they accept the ownership using the
\verb'TakePayloadOwnership' event. Note that this event event must be
logged using the credentials of the new user:

\begin{verbatim}
glite-lb-logevent -e TakePayloadOwnership -s UserInterface -j <job_id>
\end{verbatim}
