%
%% Copyright (c) Members of the EGEE Collaboration. 2004-2010.
%% See http://www.eu-egee.org/partners for details on the copyright holders.
%% 
%% Licensed under the Apache License, Version 2.0 (the "License");
%% you may not use this file except in compliance with the License.
%% You may obtain a copy of the License at
%% 
%%     http://www.apache.org/licenses/LICENSE-2.0
%% 
%% Unless required by applicable law or agreed to in writing, software
%% distributed under the License is distributed on an "AS IS" BASIS,
%% WITHOUT WARRANTIES OR CONDITIONS OF ANY KIND, either express or implied.
%% See the License for the specific language governing permissions and
%% limitations under the License.
%
\subsubsection{\LB API and library}
Both logging events and querying the service are implemented via
calls to a~public \LB API.
The complete API (both logging and queries)
is available in ANSI~C binding, most of the querying capabilities also in C++.
These APIs are provided as sets of C/C++ header files and shared libraries.
The library implements communication protocol with other \LB components
(logger and server), including encryption, authentication etc.
Since \LBver{2.0} an experimental Java binding of the logging API is available.

We do not describe the API here in detail; it is documented in~\LB Developer's
Guide\cite{lbdg},
including complete reference and both simple and complex usage examples.

Events can be also logged with a~standalone program (using the C~API in turn),
intended for usage in scripts.

The query interface is also available as a~web-service provided by the
\LB server (Sect.~\ref{server}).

Finally, certain frequent queries (all user's jobs, single job status, \dots)
are available as HTML pages (by pointing ordinary web browser to the \LB server
endpoint), or as simple text queries (since \LBver{2.0}) intended for scripts.
See~%
\ifx\insideUG\undefined
\cite{lbug}
\else
\ref{simple}
\fi
for details.

\subsubsection{Logger}
\label{comp:logger}
The task of the \emph{logger} component is taking over the events from
the logging library, storing them reliably, and forwarding to the destination
server.
The component should be deployed very close to each source of events---on the
same machine ideally, or, in the case of computing elements with many
worker nodes, on the head node of the cluster%
\footnote{In this setup logger also serves as an application proxy,
overcoming networking issues like private address space of the worker nodes,
blocked outbound connectivity etc.}.

Technically the functionality is realized with two daemons:
\begin{itemize}
\item \emph{Local-logger} accepts incoming events,
appends them in a~plain disk file (one file per Grid job),
and forwards to inter-logger.
It is kept as simple as possible in order to achieve
maximal reliability. 
\item \emph{Inter-logger} accepts the events from the local-logger,
implements the event routing (currently trivial as the destination
address is a~part of the jobid), and manages
delivery queues (one per destination server).
It is also responsible for crash recovery---on startup, the queues are
populated with undelivered events read from the local-logger files.
Finally, the inter-logger purges the files when the events are delivered to
their final destination.
\end{itemize}

\subsubsection{Server}
\label{server}
\emph{\LB server} is the destination component where the events are delivered,
stored and processed to be made available for user queries.
The server storage backend is implemented using MySQL database.

Incoming events are parsed, checked for correctness, authorized (only the job
owner can store events belonging to a~particular job), and stored into the
database.
In addition, the current state of the job is retrieved from the database,
the event is fed
into the state machine 
\ifx\insideUG\undefined\relax\else
(Sect.~\ref{evprocess}),
\fi
and the job state updated
accordingly.

On the other hand, the server exposes querying interface
(Fig.~\ref{f:comp-query}\ifx\insideUG\undefined\relax\else, Sect.~\ref{retrieve}\fi).
The incoming user queries are transformed into SQL queries on the underlying
database engine.
The query result is filtered, authorization rules applied, and the result
sent back to the user.

While using the SQL database, its full query power is not made available
to end users. 
In order to avoid either intentional or unintentional denial-of-service
attacks, the queries are restricted in such a~way that the transformed SQL
query must hit a~highly selective index on the database.
Otherwise the query is refused, as full database scan would yield unacceptable
load.
The set of indices is configurable, and it may involve both \LB system
attributes (\eg job owner, computing element,
timestamps of entering particular state,~\dots) and user defined ones.

The server also maintains the active notification handles%
\ifx\insideUG\undefined\relax\else
(Sect.~\ref{retrieve})
\fi
, providing the subscription interface to the user.
Whenever an event arrives and the updated job state is computed,
it is matched against the active handles%
\footnote{The current implementation enforces specifying an~actual jobid
in the subscription hence the matching has minimal performance impact.}.
Each match generates a~notification message, an extended \LB event
containing the job state data, notification handle,
and the current user's listener location.
The event is passed to the \emph{notification inter-logger} 
via persistent disk file and directly (see Fig.~\ref{f:comp-query}).
The daemon delivers events either in a standard way, using the specified
listener as destination, or forwards them to a messaging broker for delivery through the messaging infrastructure.
When using the standard delivery mechanism,
the server generates control messages when the user re-subscribes,
changing the listener location.
Inter-logger recognizes these messages, and changes the routing of all
pending events belonging to this handle accordingly.


% asi nepotrebujeme \subsubsection{Clients}

\subsubsection{Proxy}
\label{s:proxy}

\emph{\LB proxy} is the implementation of the concept of local view on job state (see
\ifx\insideUG\undefined{\cite{lbug}}\else{Sect.~\ref{local}}\fi). Since
\LBver{2.0}, \LB proxy is intergrated into \LB server executable.  When deployed (on
the WMS node in the current gLite middleware) it takes over the role of the
local-logger daemon---it accepts the incoming events, stores them in files, and
forwards them to the inter-logger.

In addition, the proxy provides the basic principal functionality of \LB server,
\ie processing events into job state and providing a~query interface,
with the following differences:
\begin{itemize}
\item only events coming from sources on this node are considered; hence
the job state may be incomplete,
\item proxy is accessed through local UNIX-domain socket instead of network
interface,
\item no authorization checks are performed---proxy is intended for
privileged access only (enforced by the file permissions on the socket),
\item aggressive purge strategy is applied---whenever a~job reaches
a~known terminal state (which means that no further events are expected), it is purged
from the local database immediately,
\item no index checks are applied---we both trust the privileged parties
and do not expect the database to grow due to the purge strategy.
\end{itemize}

