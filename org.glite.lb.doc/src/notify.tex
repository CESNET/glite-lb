\subsection{glite-lb-notify}
\label{s:lb-notify}

\verb'glite-lb-notify' is a fairly simple wrapper on the \LB notification API
(see also \cite{lbdg}).
It allows to create a notification (with a restricted richness of specifying 
conditions), bind to it for receiving notifications, and drop it finally.

\LB notification is a user-initiated trigger at the server.
Whenever a job enters a state matching conditions specified with the notification,
the current state of the job is sent to the notification client.
On the other hand, the notification client is a network listener
which receives server-initiated connections to get the notifications.
Unless \verb'-s' is specified, the notification library creates the listener
socket.

Within the notification validity, clients can disappear and even migrate.
However, only a single active client for a notification is allowed.

\LB server and port to contact is specified with GLITE\_WMS\_NOTIF\_SERVER 
environment variable.

\verb'glite-lb-notify' support these actions:

\begin{tabularx}{\textwidth}{lX}
\texttt{new} & Create new notification registration.\\
\texttt{bind} &  Binds an notification registration to a client.\\
\texttt{refresh} &  Enlarge notification registration validity.\\
\texttt{receive}  & Binds to an existing notification registration and listen to
server.\\
\texttt{drop}     & Drop the notification registration.\\
\end{tabularx}

For action \verb'new', command usage is:

\begin{verbatim}
  glite-lb-notify new [ { -s socket_fd | -a fake_addr } -t requested_validity ] 
           {-j jobid | -o owner | -n network_server | -v virtual_organization }
\end{verbatim}

For action \verb'bind', command usage is:
\begin{verbatim}
  glite-lb-notify bind [ { -s socket_fd | -a fake_addr } -t requested_validity ] 
           notifid
\end{verbatim}

For action \verb'refresh', command usage is:
\begin{verbatim}
  glite-lb-notify refresh [-t requested_validity ] notifid
\end{verbatim}

For action \verb'receive', command usage is:
\begin{verbatim}
  glite-lb-notify receive [ { -s socket_fd | -a fake_addr } ] [-t requested_validity ] 
            [-i timeout] [-f field1,field2,...] [notifid]
\end{verbatim}

For action \verb'drop', command usage is:
\begin{verbatim}
   glite-lb-notify drop notifid
\end{verbatim}

where

\begin{tabularx}{\textwidth}{lX}
\texttt{  notifid} & Notification ID \\
\texttt{  -s socket\_fd} &  allows  to  pass  a opened listening socket  \\
\texttt{  -a fake\_addr} &  fake the client address \\
\texttt{  -t requested\_validity} & requested validity of the notification (in seconds)   \\
\texttt{  -j jobid} & job ID to connect notification registration with   \\
\texttt{  -o owner} & match this owner DN   \\
\texttt{  -n network\_server} &  match only this network server (WMS entry point)  \\
\texttt{  -v virtual\_organization} & match only jobs of this Virtual Organization  \\
\texttt{  -i timeout} & timeout to receive operation in seconds   \\
\texttt{  -f field1,field2,...} & list of status fields to print (only owner by default)   \\
\end{tabularx}

For additional information see also manual page glite-lb-notify(1).

\subsubsection{Example: Registration and waiting for simple notification}
\label{e:notify}

The two steps bellow describe basic testing procedure of the notification
system by registering a notification on any state change of this job
and waiting for notification.

Register notification:
\begin{verbatim}
  export EDG_WL_NOTIF_SERVER=HOSTNAME:9000
  ./glite-lb-notify new JOBID
\end{verbatim}

Wait for notification:
\begin{verbatim}
 TODO
\end{verbatim}

When you let notification client running several minutes without any incomming notification, it will finish and remove your registration from the server automatically.

\subsubsection{Example: Waiting for notifications on all user's jobs}

\TODO{}

\subsubsection{Example: Waiting for more notifications on one socket}

\TODO{ruda}
