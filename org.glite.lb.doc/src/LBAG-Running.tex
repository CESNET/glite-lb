\section{Maintenance}

\subsection{\LB server}

This section deals with several typical but more peculiar tasks
that need more verbose description.
It is complemented with the full commands reference that is provided
as standard manual pages installed with the \LB packages.

\subsubsection{Changing index configuration}

% full-scan skodi, LB se tomu brani
Inefficient queries, yielding full scan of \LB database tables (up to millions of tuples) would degrade server performance.
Therefore \LB does not allow arbitrary queries in general
(server option \verb'--no-index' can change this behaviour).
On the contrary, a~query has to hit a~\emph{job index}, build on one or
more job attributes.
It is left up to the specific \LB server administrator to decide
which job attributes are selective enough to be indexed and allow queries
(\eg for many-users communities job owner can be a~sufficient criterion;
for others, where only a~few users submit thousands of jobs, it is not).

% indexy  -- implementace jako extra sloupce => zmeny offline, konfigurace
% olizovana z DB
Technically, job indices are implemented via dedicated columns
in a~database table.
These columns and their indices are scanned by the \LB server on startup,
therefore there is no specific configuration file.
Changing the index configuration is rather heavyweight operation
(depending on the number of jobs in the database), it performs
updates of all tuples in general, and it should be done when the server is not
running.

% utilitka bkindex -- pouziti
Indices are manipulated with a~standalone utility \verb'glite-lb-bkindex'
(see its man page for complete usage reference).
A~general sequence of changing the indices is:
\begin{enumerate}
\item stop the running server
\item retrieve current index configuration
\begin{quote}
\verb'glite-lb-bkindex -d >index_file'
\end{quote}
\item edit \verb'index_file' appropriately
\item re-index the database (it may take long time)
\begin{quote}
\verb'glite-lb-bkindex -r -v index_file'
\end{quote}
\verb'-r' stands for ``really do it'', \verb'-v' is ``be verbose''
\item start the server again
\end{enumerate}

% format vstupniho souboru
The index description file follows the classad format, having the following grammar:
\begin{quote}
\emph{IndexFile} ::= [ JobIndices = \{ \emph{IndexList} \} ] \\
\emph{IndexList} ::= \emph{IndexDef} $|$ \emph{IndexDef}, \emph{IndexList} \\
\emph{IndexDef} ::= \emph{IndexColumn} $|$ \emph{ComplexIndex} \\
\emph{IndexColumn} ::= [ type = "\emph{IndexType}"; name = "\emph{IndexName}" ]\\
\emph{ComplexIndex} ::= \{ \emph{ColumnList} \} \\
\emph{ColumnList} ::= \emph{IndexColumn} $|$ \emph{IndexColumn}, \emph{ColumnList}
\end{quote}

where eligible \emph{IndexType}, \emph{IndexName} combinations are given
in Tab.~\ref{t:indexcols}.
A~template index configuration, containing indices on the most frequently
used attributes, can be found in /opt/glite/etc/glite-lb-index.conf.template.


\begin{table}
\begin{center}
\begin{tabularx}{.9\hsize}{|l|l|X|}
\hline
\emph{IndexType} & \emph{IndexName} & description \\
\hline
system & owner & job owner \\
 & destination & where the job is heading to (computing element name) \\
 & location & where is the job being processed \\
 & network\_server & endpoint of WMS \\
 & stateEnterTime & time when current status was entered \\
 & lastUpdateTime & last time when the job status was updated \\
\hline
time & \emph{state name} & when the job entered given state (Waiting, Ready, \dots) \\
\hline
user & \emph{arbitrary} & arbitrary user tag \\
\hline
\end{tabularx}
\end{center}
\caption{Available index column types and names}
\label{t:indexcols}
\end{table}

% super user muze vsechno

\subsubsection{Multiple server instances}

% lze to, i nad jednou databazi, zadna automaticka podpora

\subsubsection{Backup dumps}

Denni dumpy (ne purge)

\subsubsection{Purging and processing old data}

\TODO{salvet}

\paragraph{Post-mortem statistics}

\paragraph{Export to Job Provenance}

\subsubsection{On-line monitoring and statistics}

CE Rank

DB mon (a mon)

histogramy :-)

\subsection{\LB proxy}

Purge zamrzlych jobu (overit v kodu, na ktere verzi to mame)

\subsection{\LB logger}

Karantena (od ktere verze to mame?)

Cistky pri zaseknuti, nesmyslna jobid apod.

\subsection{Used resources}

Adresare, porty, ...
