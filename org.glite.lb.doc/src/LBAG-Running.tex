%
%% Copyright (c) Members of the EGEE Collaboration. 2004-2010.
%% See http://www.eu-egee.org/partners for details on the copyright holders.
%% 
%% Licensed under the Apache License, Version 2.0 (the "License");
%% you may not use this file except in compliance with the License.
%% You may obtain a copy of the License at
%% 
%%     http://www.apache.org/licenses/LICENSE-2.0
%% 
%% Unless required by applicable law or agreed to in writing, software
%% distributed under the License is distributed on an "AS IS" BASIS,
%% WITHOUT WARRANTIES OR CONDITIONS OF ANY KIND, either express or implied.
%% See the License for the specific language governing permissions and
%% limitations under the License.
%
\section{Maintenance}

\subsection{Changing default settings}
All configurable settings of the \LB daemons (network and local sockets, 
file paths, modified behaviour etc.) can be set with specific command line
options.
In the following only relevant options are discussed whenever appropriate.
See specific manual pages for complete reference.

\subsection{\LB server and proxy}

This section deals with several typical but more peculiar tasks
that need more verbose description.
It is complemented with the full commands reference that is provided
as standard manual pages installed with the \LB packages.

\subsubsection{Standard and debug logs}

In normal operation \LB server sends error messages to syslog.
Informational messages are generally avoided in order to prevent syslog congestion.

Since \LBver{2.1}, the service implements a common logging format\footnote{\url{http://en.wikipedia.org/wiki/Common_Log_Format}} (see section\,\ref{inst:comlog}, page\,\pageref{inst:comlog}).

For \LBver{2.0 and lower}, the following instructions apply:

\begin{sloppypar}
When tracing problems, GLITE\_LB\_SERVER\_DEBUG environment variable can be set to
non-empty value when starting the service.
Then verbose log \$GLITE\_LOCATION\_VAR/lb.log 
(as well as \$GLITE\_LOCATION\_VAR/notif-il.log eventually when notifications are enabled).
Beware that these can grow huge easily.
\end{sloppypar}

\textbf{\LBver{1.x} only:} not available for \LB proxy, \verb'-d' and output redirection
have to be added manually if necessary.

\subsubsection{Changing index configuration}
\label{maintain:index}

\LB server only (\LB proxy database is neither so huge nor accessed directly by users).

% full-scan skodi, LB se tomu brani
Inefficient queries, yielding full scan of \LB database tables (up to millions of tuples) would degrade server performance.
Therefore \LB does not allow arbitrary queries in general
(server option \verb'--no-index' can change this behaviour).
On the contrary, a~query has to hit a~\emph{job index}, build on one or
more job attributes.
It is left up to the specific \LB server administrator to decide
which job attributes are selective enough to be indexed and allow queries
(\eg for many-users communities job owner can be a~sufficient criterion;
for others, where only a~few users submit thousands of jobs, it is not).

% indexy  -- implementace jako extra sloupce => zmeny offline, konfigurace
% olizovana z DB
Technically, job indices are implemented via dedicated columns
in a~database table.
These columns and their indices are scanned by the \LB server on startup,
therefore there is no specific configuration file.
Changing the index configuration is rather heavyweight operation
(depending on the number of jobs in the database), it performs
updates of all tuples in general, and it should be done when the server is not
running.

% utilitka bkindex -- pouziti
Indices are manipulated with a~standalone utility \verb'glite-lb-bkindex'
(see its man page for complete usage reference).
A~general sequence of changing the indices is:
\begin{enumerate}
\item stop the running server
\item retrieve current index configuration
\begin{quote}
\verb'glite-lb-bkindex -d >index_file'
\end{quote}
\item edit \verb'index_file' appropriately
\item re-index the database (it may take long time)
\begin{quote}
\verb'glite-lb-bkindex -r -v index_file'
\end{quote}
\verb'-r' stands for ``really do it'', \verb'-v' is ``be verbose''
\item start the server again
\end{enumerate}

% format vstupniho souboru
The index description file follows the classad format, having the following grammar:
\begin{quote}
\emph{IndexFile} ::= [ JobIndices = \{ \emph{IndexList} \} ] \\
\emph{IndexList} ::= \emph{IndexDef} $|$ \emph{IndexDef}, \emph{IndexList} \\
\emph{IndexDef} ::= \emph{IndexColumn} $|$ \emph{ComplexIndex} \\
\emph{IndexColumn} ::= [ type = "\emph{IndexType}"; name = "\emph{IndexName}" ]\\
\emph{ComplexIndex} ::= \{ \emph{ColumnList} \} \\
\emph{ColumnList} ::= \emph{IndexColumn} $|$ \emph{IndexColumn}, \emph{ColumnList}
\end{quote}

where eligible \emph{IndexType}, \emph{IndexName} combinations are given
in Tab.~\ref{t:indexcols}.
A~template index configuration, containing indices on the most frequently
used attributes, can be found in\\ \texttt{[/opt/glite]/etc/glite-lb-index.conf.template}.\footnote{The \texttt{/opt/glite} prefix applies for \LB installed from gLite's \LB node repository.} 


\begin{table}
\begin{center}
\begin{tabularx}{.9\hsize}{|l|l|X|}
\hline
\emph{IndexType} & \emph{IndexName} & description \\
\hline
system & owner & job owner \\
 & destination & where the job is heading to (computing element name) \\
 & location & where is the job being processed \\
 & network\_server & endpoint of WMS \\
 & stateEnterTime & time when current status was entered \\
 & lastUpdateTime & last time when the job status was updated \\
\hline
time & \emph{state name} & when the job entered given state (Waiting, Ready, \dots) \\
\hline
user & \emph{arbitrary} & arbitrary user tag \\
\hline
\end{tabularx}
\end{center}
\caption{Available index column types and names}
\label{t:indexcols}
\end{table}

% super user muze vsechno

\subsubsection{Multiple instances}

% lze to, i nad jednou databazi, zadna automaticka podpora

Specific conditions (\eg debugging, different authorization setup, \dots)
may require running multiple \LB server instances
on the same machine.
Such setup is available, however, there is no specific support in automated
configuration, the additional non-default server instances must be run manually.

The other server instance must use different ports (changed with \verb'-p'
and \verb'-w' options), as well as use different pid file (\verb'-i' option).

The servers may or may not share the database (non-default is specified
with  \verb'-m')%
\footnote{Even when sharing the database, the servers are still 
partially isolated from
one another, \eg a~job \url{https://my.machine:9000/xyz} cannot be queried 
as \url{https://my.machine:8000/xyz}.
However, due to implementation internals, the second job cannot be registered.}.

Though it may have little sense to run multiple \LB proxy instances, it is possible too.
Non-default listening socket have to be specified via \verb'-p' option then.

Note that in more recent \LB versions, clients observe the \texttt{GLITE\_WMS\_LBPROXY\_SERVERNAME} environmental variable, which holds the \texttt{hostname:port} of the \LB server local to their \LB proxy. This allows them to avoid redundant job registrations through the proxy as well as directly to the \LB server. If multiple instances are used, it is necessary to make sure that this variable is either unset, or points to the appropriate \LB server instance.


\subsubsection{Backup dumps}
\label{run:dump}

\LB server only, not supported by proxy.

(This functionality should not be confused with per-job dumps, Sect.~\ref{inst:purge} and \ref{run:purge})

Besides setting up \LB server database on a~reliable storage or
backing it up directly (Sect.~\ref{inst:backup})
\LB server supports backing up only incremental changes in the data.
Advantages of this approach are lower volume of data to be backed up,
and possibility to load them to another instance (\eg for heavyweight
queries which should not disturb normal operation), disadvantage is
a~more complex and more fragile setup. 

Using an external utility \verb'glite-lb-dump' (typical invocation is with
a~single option \verb'-m' \emph{server\_name:port}, see man page for
details) the server is triggered to dump events, which arrived in
a~specified time interval, into a~text file. (Default interval is from last
dump till the current time.)

\verb'glite-lb-dump' is a~standalone client program, however, 
the events are stored at server side (\ie not transferred to the client,
due to performance reasons),
in a~uniquely named text
file prefixed with the value of \verb'-D' server option. This kind of dump
contains events according to their arrival time, regardless of jobs they belong
to.

It is sufficient to run the dump regularly (from a~cron job), with a~frequency
matching an acceptable risk of loosing data (several hours typically), and back
up the resulting dump files. 

In the event of server crash, its database should be recreated empty,
and the server started up.
Then the dump files can be loaded back with complementary
\verb'glite-lb-load' utility.

Server privileges granting \verb'ADMIN_ACCESS' (see section~\ref{inst:authz}) are required to run \verb'glite-lb-dump' and \verb'glite-lb-load'.
Dumping the events does not interfere with normal server operation.

This backup strategy can interfere with too aggressive setting of old
data purging (Sect.~\ref{run:purge}), 
If the purging grace period is shorter than the dump interval,
events may get purged before they are captured by the backup dump.
However, this interference is unlikely (reasonable purge grace period
is several times longer than dump period),
and it is not fatal in general (data were purged on purpose either).

\subsubsection{Purging and processing old data}
\label{run:purge}

Primary purpose of the LB purge operation  is removal of aged data from LB database. This is necessary in
production in order to prevent ever-increasing database and sustain reasonable
performance of the server. Therefore the purge should be invoked periodically.

The purge operation has additional important ``side effect'' -- dumping the
purged data into a plain text file. These dumps can be archived ``as is'' or
uploaded to Job Provenance. 

\paragraph{Purge setup}

The purge operation itself is performed by a~running \LB server
(there is no need to shut it down, then).
However, it is triggered with \verb'glite-lb-purge' client command
(complete usage reference is given in its man page).
A~typical invocation specifies \LB server to purge (\verb'-m' option),
and purge timeouts (grace periods) for several job states -- options
\verb'-a' (aborted), \verb'-n' (canceled), \verb'-c' (cleared), and
\verb'-o' (other). \LBver{2.0} and newer support also \verb'-e' (done) option.
It is important to note that the \emph{other} option refers to a fixed set of states that do not have their own controlling option. Its meaning does not stretch dynamically to cover all unlisted states. In other words, if, e.g., argument \texttt{-a} is not passed to the purge command, it is \textbf{not} automatically covered by \texttt{-o}. Instead, the internal default defined in the \LB server/proxy is used.

A~job falling in one of the four categories is purged when it has not been
touched (\ie an event arrived) for time longer than the specified category
timeout.
Suggested values are several days for aborted and canceled jobs,
and one day for cleared jobs, however, the values may strongly vary
with \LB server policy.

Optionally, \verb'-s' purge command option instructs the server to
dump the purged data into a~file at the server side.
It's location (prefix) is given by \verb'-S' server option,
the purge command reports a~specific file name on its output.

It is recommended (and the default YAIM setup does so, via
the \verb'glite-lb-export.sh' wrapper) to run the purge
command periodically from cron.

Server privileges granting \verb'ADMIN_ACCESS' (see section~\ref{inst:authz}) are required to run \verb'glite-lb-purge'.

If the server database has already grown huge, the purge operation can take
rather long and hit the \LB server operation timeout. At client side, \ie the
glite-lb-purge command, it can be increased by setting GLITE\_WMS\_QUERY\_TIMEOUT
environment variable.
Sometimes hardcoded server-side timeout can be still reached. In either case the
server fails to return a correct response to the client but the purge is done anyway. 

\LB proxy purges jobs automatically when they reach a~state ensuring that WMS will
neither query nor log events to them anymore.
Therefore routine purging is not required theoretically.
However, frozen jobs which never reach such a~state may occur in an unstable environment,
and they may cumulate in \LB proxy database for ever.
Therefore occasional purging is recommended too.
\LBver{2.x} supports \verb'-x' option of \verb'glite-lb-purge', allowing
to purge \LB proxy database too.
With \LBver{1.x} the emergency purge procedure described bellow is the only option.

\paragraph{Emergency purge}

When regular purge was not invoked for some time, it may happen that 
the database grows huge and the regular (on-line) purge fails.
In order to work around such situation we provide an off-line emergency
purge script \verb'glite-lb-bkpurge-offline.sh'

The script accepts the same \verb'-acno' options, and adds \verb'-d' for ``done'' jobs. 
Via \verb'-p' also \LB proxy database can be purged (all \LB versions).

On startup, a~warning message is printed and interactive confirmation
requested.
Re-check that \LB server (proxy) is not running, and carry on only when you
know what you are doing.

\paragraph{Post-mortem statistics}

Once a job is purged from the database, all important data about the job can be
processed offline from the corresponding dump file. The idea of post-mortem
statistics is the following:

\begin{itemize}
\item LB server produces dump files (during each purge on regular basis),
see LB server startup script; option \verb'-D / --dump-prefix' of \verb'glite-lb-bkserverd',
\item these dumps are exported for the purposes of JP also on regular basis,
see LB/JP deplyment module; option \verb'-s/ --store' of \verb'glite-lb-lb_dump_exporter',
\item it depends on the LB server policy if dumps in this directory are used for
the statistics purposes or all files are hardlinked for example to a different
directory
\item general idea is such that data are available for statistics server that downloads
and removes dumps after download! Dump files are then processed on the statistics
server.
\end{itemize}

What needs to be done on the LB server:
\begin{itemize}
\item \verb'glite-lb-bkserverd' and \verb'glite-lb-lb_dump_exporter' running
\item \verb'gridftp' running (allowing statistics server to download and remove files from 
a given directory
\end{itemize}


What needs to be done on the statistics server:
\begin{itemize}
\item \verb'glite-lb-utils' package installed
\item download and remove files from the LB server
see \verb'glite-lb-statistics-gsi.sh' (shell script in the examples directory)
\item process dump files using the \verb'glite-lb-statistics' tool
see \verb'glite-lb-statistics.sh' (shell script in the examples directory)
\end{itemize}
all scripts are supposed to be run from a crontab.


\paragraph{Export to Job Provenance}

An important, though currently optional, processing of \LB dumps
is their upload to the Job Provenance service for permanent preservation
of the data.


When enabled (via configuration environment variables, see bellow), 
the export is done in two steps:
\begin{itemize}
\item \verb'glite-lb-export.sh' wrapper script, after calling \verb'glite-lb-purge', breaks up the resulting dump file on a~per-job basis.
The individual job dump files are stored in a~dedicated spool directory.
\item \verb'glite-jp-importer' daemon (installed optionally in glite-jp-client.rpm) checks the spool directory periodically,
and tries to upload the files into JP.
\end{itemize}

Details, including the configuration variables, are covered at the following
wiki page:

\url{http://egee.cesnet.cz/mediawiki/index.php/LB_purge_and_export_to_JP}.

\paragraph{Persistent Information on Purged Jobs (``Zombies'')}

Since \LB version 2.0, the JobID of a purged job is not fully discarded but stored in a separate table. A query for the status of such job returns \emph{Identifier removed}.\footnote{As of \LB\,3.0. In 2.0 and 2.1 releases an empty job status structure was returned with job state set to \emph{purged}.} There are two purposes to this arrangement:

\begin{enumerate}
\item Confirming that the job existed and that its details have been exported to Job Provenance (if deployed and configured),
\item Preventing reuse of the JobID in case of flag \code{EDG\_WLL\_LOGLFLAG\_EXCL} set on registration.
\end{enumerate}

\subsubsection{On-line monitoring and statistics}
\label{maintain:statistics}

\paragraph{CE reputability rank}

Rather frequent problem in the grid production are ``black hole'' sites (Computing Elements).
Such a~site declares itself to have an empty queue, therefore schedulers usually prefer sending
jobs there. The site accepts the job but it fails there immediately.
In this way large number of jobs can be swallowed, affecting the overall success rate
(namely for non-resubmittable jobs).

\LB data as a~whole contain enough information to detect such sites.
However, due to the primary per-job structure certain reorganization is required.

A~job is always assigned to a~\emph{group} according to
the CE where it is executed (cf.\ ``destination'' job state attribute).
Similarly to RRDtool\footnote{\url{http://oss.oetiker.ch/rrdtool/}}
for each recently active group (CE),
and for each job state (Ready, Scheduled, Running, Done/OK, Done/Failed),
a~fixed sized series of counters is maintained.
At time $t$, the counters cover intervals $[t-T,t]$, $[t-2T,t-T]$, \dots
where $T$ a~fixed interval size.
Whenever a~job state changes, the series matching the group and new state
is shifted eventually (dropping its expired tail), and the current counter
is incremented.
In addition, multiple series for different $T$ values (\ie covering different
total times) are available. 

% API
The data are available via statistics calls of the client API,
see \verb'statistics.h' for details (coming with glite-lb-client in \LBver{2.x},
glite-lb-client-interface in \LBver{1.x}).
The call specifies the group and job state of interest, as well as queried
time interval.
The interval is fitted to the running counter series as accurately as possible,
and the average number of jobs per second which entered the specific state for
the given group is computed.  The resolution ($T$) of the used counters is also
returned.

\begin{sloppypar}
% successFraction(CEId) classad gLite 3.1 WMS, nedokumentovana, netestovana
In gLite 3.1 WMS the calls can be accessed from inside of the matchmaking process
via \verb'successFraction(CEId)'
JDL function.
The function computes the ratio of successful vs.\ all jobs for a~given CE,
and it can be directly used to penalize detected black hole CEs in the ranking
JDL expression.
\end{sloppypar}


% zapnuti na serveru, volatilita, privilegia
The functionality is enabled with \verb'--count-statistics' \LB server option
(disabled by default).

The gathered information is currently not persistent, it is lost when the server is stopped.
Despite the statistics call API is defined in a~general way, the implementation is
restricted to a~hardcoded configuration of a~single grouping criterion (the destination),
and a~fixed set of counter series (60 counters of $T=10s$, 30 of 1 minute, and 12 of 15 minutes).
The functionality has not been very thoroughly tested yet.

% omezeni implementace: hardcoded konfigurace, jen Rate, neprilis dukladne testovane



\paragraph{glite-lb-mon} is a program for monitoring the number of jobs on the
LB server and their several statistics. It is part of the
\verb'glite-lb-utils' package, so the monitoring can be done from remote machine
where this package is installed and the enironment variable
\verb'GLITE_WMS_QUERY_SERVER' properly set. Values like minimum, average and
maximum time spent in the system are calulated for jobs that entered  the
final  state (Aborted,  Cleared,  Cancelled) in specific time (default last
hour). Also number of jobs that entered the system during this time is
calculated.

A special bkindex configuration is needed. 
The following time indices must be defined:
\begin{verbatim}
            [ type = "time"; name = "submitted" ],
            [ type = "time"; name = "cleared" ],
            [ type = "time"; name = "aborted" ],
            [ type = "time"; name = "cancelled" ],
\end{verbatim}
For more details se man page glite-lb-mon(1).


\paragraph{glite-lb-mon-db} is a low-level program for monitoring the the
number of jobs in the LB system. Using the LB internals, it connects directly
to the underlying MySQL database and reads the number of jobs in each state.
The tool is distributed itogether with the server in the \verb'glite-lb-server' package.
It can be used to read data also from the database of LB Proxy.
For more details se man page glite-lb-mon-db(1).


\paragraph{Subjob states in a collection} can be calculated on demand on the server and
returned as a histogram using standard job status query. There are two ways how to obtain the 
histogram:
\begin{itemize}
\item fast histograms, the last known states are returned, see e.g.
\begin{verbatim}
   glite-lb-job_status -fasthist <collectionJobId>
\end{verbatim}
\item full histograms. the states of all collection subjobs are recalculated, see o.g.
\begin{verbatim}
   glite-lb-job_status -fullhist <collectionJobId>
\end{verbatim}
\end{itemize}
The command \verb'glite-lb-job_status' is a low level query program that can be
found in the package \verb'glite-lb-client' among examples.


%\subsection{\LB proxy}

%\TODO{ljocha}
%Purge zamrzlych jobu (overit v kodu, na ktere verzi to mame )


\subsection{\LB logger}

\iffalse
\TODO{ljocha}

Karantena (od ktere verze to mame?)
- kdyz se nepodari rozparsovat soubor
- client/examples -- parse-logevent-file??, lze pouzit

Cistky pri zaseknuti, nesmyslna jobid apod.

Debugovaci rezim

Notifikacni IL
\fi

The logger component (implemented by \verb'glite-lb-interlogd' daemon fed by
either \verb'glite-lb-logd' or \LB proxy)
is responsible for the store-and-forward event delivery in \LB
(Sect~\ref{comp:logger}).
Therefore eventual operational problems are related mostly to 
cumulating undelivered events.

\subsubsection{Event files}

\LB logger stores events in one file per job, named
\verb'$GLITE_LOCATION_VAR/log/dglogd.log.JOBID' by default
(JOBID is only the part after the \LB server address prefix).
The format is text (ULM~\cite{ulm}), one event per line.
In addition, control information on delivery status is stored in additional
file with \verb'.ctl' suffix.

\begin{sloppypar}
In case of emergency (\eg corrupted file) the files can be examined
with \verb'glite-lb-parse_eventsfile'.\footnote{Not fully supported tool, installed
by \texttt{glite-lb-client} package among examples.}
It is possible to hand-edit the event files in emergency (remove corrupted lines).
However, glite-lb-interlogd must not be running, and the corresponding .ctl file
must be removed.
\end{sloppypar}

\subsubsection{Backlog reasons}

\paragraph{Undeliverable jobid.}
In normal gLite job processing, jobids are verified on job submission
(via synchronous job registration, see~\cite{lbug}), hence occurrence of
undeliverable jobid (\ie its prefix does not point to
a~working \LB server) is unlikely.
On the other hand, if it happens,
and an event with such a~jobid is logged,
\eg due to a~third-party job processing software bug,
glite-lb-interlogd keeps trying to deliver it indefinitely.\footnote{Unless
event expiration is set, though it is not done for normal events.}
The unsuccessful attempts are reported via syslog.
The only solution is manual
removal of the corresponding files
and restart of the service.

\paragraph{Corrupted event file.} 
For various reasons the files may get corrupted.
In general, corrupted file is detected by glite-lb-interlogd, and it is moved
to \emph{quarantine} (by renaming the file to contain ``quarantine'' in its name).
The action is reported in syslog.
The renamed files can be removed or repaired by hand and renamed back
for glite-lb-interlogd to pick them up again
(in this case, the service needn't be stopped).

\paragraph{Slow delivery.}
Either glite-lb-interlogd or the target \LB server(s) may not keep pace 
with the incoming stream of events. 
Amount of such backlog can be quickly assessed by looking at timestamps of the oldest
event files (with \verb'ls -lt $GLITE_LOCATION_VAR/log/', for example). 
Fully processed event files are deleted in approx. one minute intervals, 
so files last modified one minute ago or newer do not constitute backlog at all.
Unless the backlog situation is permanent, no specific action is required---the 
event backlog decreases once the source is drained.
Otherwise hardware bottlenecks (CPU, disk, network) have to be identified
(with standard OS monitoring) and removed, see sections~\ref{inst:hw_req} and
\ref{inst:db_tuning} for \LB server tuning tips. In order to maximize 
performance of \verb'glite-lb-interlogd', it is recommended to distribute
jobs to multiple \LB servers that can be shared by multiple WMS setups in turn.

\subsubsection{Notification delivery}
\begin{sloppypar}
When \verb'glite-lb-logger' package is optionally installed with \LB server, 
a~modified \verb'glite-lb-notif-interlogd' is run by the server startup
script.
This version of the daemon is specialized for \LB notifications delivery;
it uses the same mechanism, however, the events (notifications) are routed
by \emph{notification id} rather than jobid, and targeted to user's listeners,
not \LB servers.
See~\cite{lbug, lbdg} for details.
\end{sloppypar}

On the contrary to normal events, it is more likely that the event destination
disappears permanently.
Therefore the notification events have their expiration time set,
and glite-lb-interlogd purges expired undelivered notifications by default.
Therefore the need for manual purge is even less likely.

The event files have different prefix (\verb'/var/spool/lb-notif/dglogd.log' by
default, \verb'/var/tmp/glite-lb-notif' in older versions of \LB).

\subsubsection{Debug mode}

All the logger daemons, \ie glite-lb-logd, glite-lb-interlogd, and
glite-lb-notif-interlogd, can be started with \verb'-d' 
to avoid detaching control terminal, and \verb'-v' to increase 
debug message verbosity.
See manual pages for details.

\subsection{Used resources}
\label{run:resources}

\subsubsection{Server and proxy}

\begin{description}
\item[Processes and threads.]
By default \LB server runs as one master process,
and 10 slave processes.
Threads are not used.

Master pid is stored in the file \verb'$HOME/edg-bkserverd.pid' (\LBver{1.x})
or in the file \verb'$HOME/glite-lb-bkserverd.pid' (\LBver{2.x}) respectively.
Number of slaves can be set with \verb'-s, --slaves' option,
pid file location with~\verb'-i, --pidfile'.

Slave server processes are restarted regularly in order to prevent
memory leakage.

\item[Network and UNIX ports.]
\LB server listens on port 9000 for incoming queries,
9001 for incoming events, and 9003 for WS interface queries.
The former two can be changed with \verb'-p, --port' option
(incoming events are always one port higher than queries), 
the latter with~\verb'-w, --wsport'.

When \LB notifications are enabled (automatically in server startup script when 
glite-lb-logger is also installed with the server), 
\verb'/var/run/glite/glite-lb-notif-interlogd.sock' is used for communication
with notification interlogger. It can be changed with \verb'--notif-il-sock'.

\LB proxy communicates on two UNIX sockets:
\verb'/tmp/lb_proxy_server.sock' (queries) and 
\verb'/tmp/lb_proxy_store.sock' (incoming events).

\LBver{2.x}: as proxy and server are merged, the mode of operation determines
the actual server ports (network or UNIX).
Notifications are not delivered from proxy-only mode.

\item[IPC.] \LBver{1.x} only: mutual exclusion of server slaves is done
via SYSV semaphores. The semset key is obtained by calling ftok(3) on 
the pid file.

\LBver{2.x} does not use semaphores anymore.

\item[Database.]
Server data are stored in MySQL database. Normal setup assumes ``server scenario'', 
\ie the database IS NOT protected with password and it is not accessible over network.
Option \verb'-m user/password@machine:database' can be used to change
the default connect string \verb'lbserver/@localhost:lbserver20'.

\item[Dump files.]
Backup dump (Sect.~\ref{run:dump}) files are stored in
\verb'/tmp/dump' by default, however, gLite startup script uses
\verb'--dump-prefix' option to relocate them into \verb'$GLITE_LOCATION_VAR/dump'.
Similarly dumps resulting from purge operation (Sect.~\ref{run:purge})
go to \verb'/tmp/purge' by default, and startup script uses \verb'--purge-prefix'
to set \verb'$GLITE_LOCATION_VAR/purge'.

When export to Job Provenance is enabled (Sect.~\ref{run:purge})
job registrations are exported to directory \verb'/tmp/jpreg'
overridden in startup script with \verb'--jpreg-dir' to \verb'$GLITE_LOCATION_VAR/jpreg'.
In addition, dump files are further processed in \verb'GLITE_LOCATION_VAR/jpdump'.

Normal operation of JP export should not leave any files behind,
however, abnormal situations (JP unavailable for longer time etc.)
may start filling the disk space. Therefore periodic checks are
recommended.

Also when purge and dump operations are invoked resulting files are left in their
directories and it is the responsibility of the operation caller to clean them up.

\item[Notifications]
(see above with Network ports) are stored in files with \verb'/var/spool/glite/lb-notif/dglogd.log'
prefix. It can be changed with \verb'--notif-il-fprefix'.

Normal operation of notification interlogger purges these files when
they are either delivered or expire.

\end{description}

\subsubsection{Logger}

\begin{description}
\item[Processes and threads.]
glite-lb-logd uses one permanent process and forks a~child to serve
each incoming connection.

glite-lb-interlogd (and derived glite-lb-notif-interlogd) is multithreaded:
one thread for handling input, another for recovery, and a~dynamic pool of others
to deliver events to their destinations.

\item[Network and UNIX ports.]
glite-lb-logd listens on 9002 (can be changed with \verb'-p' option).
The daemons communicate over UNIX socket \verb'/var/run/glite/glite-lb-interlogd.sock',
can be changed with \verb'-s' (for both daemons simultaneously).

\item[Event files] are stored
with prefix \verb'/var/spool/glite/lb-locallogger/dglogd.log' (changed with \verb'-f').


\end{description}

