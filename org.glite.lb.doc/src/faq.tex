%
%% Copyright (c) Members of the EGEE Collaboration. 2004-2010.
%% See http://www.eu-egee.org/partners for details on the copyright holders.
%% 
%% Licensed under the Apache License, Version 2.0 (the "License");
%% you may not use this file except in compliance with the License.
%% You may obtain a copy of the License at
%% 
%%     http://www.apache.org/licenses/LICENSE-2.0
%% 
%% Unless required by applicable law or agreed to in writing, software
%% distributed under the License is distributed on an "AS IS" BASIS,
%% WITHOUT WARRANTIES OR CONDITIONS OF ANY KIND, either express or implied.
%% See the License for the specific language governing permissions and
%% limitations under the License.
%

\section{FAQ---Frequently Asked Questions}
\label{s:faq}

\subsection{Job in State `Running' Despite Having Received the `Done' Event from LRMS}

Jobs stay in state \emph{Running} until a \emph{Done} event is received from the workload management system. \emph{Done} events from local resource managers are not enough since the job in question may have been resubmitted in the meantime.

\subsection{WMS Cannot Purge Jobs or Perform Other Privileged Tasks}
\label{FAQ:WMS_superusers}

In short, WMS has not been given adequate permissions when configuring the \LB server. You need to modify your configuration and restart the server:

\subsubsection{For \LBver {3.0.11 or higher}, using YAIM}
\label{FAQ:WMS_superusers_3_0_11}
Modify your \texttt{siteinfo.def}, specifying the DN of your WMS server in YAIM parameter \texttt{GLITE\_LB\_WMS\_DN}; for instance:

\begin{center}
\texttt{GLITE\_LB\_WMS\_DN=/DC=cz/DC=cesnet-ca/O=CESNET/CN=wms01.cesnet.cz}
\end{center}

Then rerun YAIM: 
\texttt{/opt/glite/yaim/bin/yaim -c -s site-info.def -n glite-LB}

 This will give your WMS exactly the right permissions to carry out all required operations.

\subsubsection{For all versions of \LB, using YAIM}

Modify your \texttt{siteinfo.def}, specifying the DN of your WMS server in YAIM parameter \texttt{GLITE\_LB\_SUPER\_USERS}; for instance:

\begin{center}
\texttt{GLITE\_LB\_SUPER\_USERS=/DC=cz/DC=cesnet-ca/O=CESNET/CN=wms01.cesnet.cz}
\end{center}

Then rerun YAIM: 
\texttt{/opt/glite/yaim/bin/yaim -c -s site-info.def -n glite-LB}

This will give your WMS adequate rights to perform its operations and requests (running purge, querying for statistics, etc.) but it will also grant it additional administrator rights (such as granting job ownership). On newer installations, the method explained in section \ref{FAQ:WMS_superusers_3_0_11} is preferrable.

\subsubsection{For \LBver {2.1 or higher}, without YAIM}

\LB{}'s authorization settings can be found in file \texttt{[/opt/glite]/etc/glite-lb/glite-lb-authz.conf}

Permit actions \texttt{PURGE}, \texttt{READ\_ALL} and \texttt{GET\_STATISTICS} for your WMS and restart the \LB server.
This will lead to results equivalent to \ref{FAQ:WMS_superusers_3_0_11}.
For instance, change the adequate sections in \texttt{glite-lb-authz.conf} to:

\begin{verbatim}
action "READ_ALL" {
        rule permit {
                subject = "/DC=cz/DC=cesnet-ca/O=CESNET/CN=wms01.cesnet.cz"
        }
}

action "PURGE" {
        rule permit {
                subject = "/DC=cz/DC=cesnet-ca/O=CESNET/CN=wms01.cesnet.cz"
        }
}

action "GET_STATISTICS" {
        rule permit {
                subject = "/DC=cz/DC=cesnet-ca/O=CESNET/CN=wms01.cesnet.cz"
        }
}
\end{verbatim}

