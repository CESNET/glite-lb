\subsection{Indexed attributes}\label{ConsIndx}
\LB\ queries can be fairly complicated and they can potentially return large 
amounts of data. In such cases, there is a~risk of server overload.
In order to prevent the overload, every \LB\ server configuration
includes a~set of configurable \emph{indices}, empty by default
(job ID is always indexed).
In general, at least one indexed attribute must be
present in each query\,---\,queries over only attributes without indices
are refused.

Unfortunately, in some cases queries that may be considered very important from
the user's point of view (like all user's jobs) fall into this category. 
Indexing specific attributes makes these queries tractable
(while increasing the risk of server overload).
It's left up to the server administrator to decide which query types are
supported by configuring indices over attribute sets of desired queries using
the \texttt{glite-lb-bkindex} utility.
I.\,e.\ if the attribute \texttt{EDG\_WLL\_QUERY\_ATTR\_OWNER} is not indexed,
some of 
queries in the presented examples 
% odkazuje na totez :-( (\eg\ Sect.'s~\ref{JQ-auj},\ref{JQ-rj})
(Sect.'s~\ref{JQ-auj})
are forbidden.

\subsection{Timeouts}

All blocking \LB\ API calls are subject to timeouts. Timeout values can be changed 
from their respective defaults with \texttt{edg\_wll\_SetParam} 
call or using variables of the process environment (floating point number of seconds).

\begin{tabular}{lcll}
affected calls&default(max)&parameter&env. variable\\
async. logging&120 (1800)&\texttt{EDG\_WLL\_PARAM\_LOG\_TIMEOUT}&\texttt{EDG\_WL\_LOG\_TIMEOUT}\\
sync. logging&120 (1800)&\texttt{EDG\_WLL\_PARAM\_LOG\_SYNC\_TIMEOUT}&\texttt{EDG\_WL\_LOG\_SYNC\_TIMEOUT}\\
queries&120 (1800)&\texttt{EDG\_WLL\_PARAM\_QUERY\_TIMEOUT}&\texttt{EDG\_WL\_QUERY\_TIMEOUT}\\
notifications&120 (1800)&\texttt{EDG\_WLL\_PARAM\_NOTIF\_TIMEOUT}&\texttt{EDG\_WL\_NOTIF\_TIMEOUT}\\
\end{tabular}

\subsection{Timestamps}
Timestamps of \LB\ event is recorded by the process calling a~logging API call. It is strongly recommended
to keep clocks of all systems that produce \LB\ events (UI, WM, WN) in reasonable synchronization so that
timestamp attributes of events and job status can be interpreted easily.

On the other hand, timestamps are not authoritative when \LB\ events
are sorted in order to compute job state\Dash a~more robust mechanism
of the hierarchical \LB\ sequence code is used instead. 
Consequently, strict timestamp sorting of events coming from desynchronised
sources may give different (incorrect) results from what \LB\ reports in job
state.

\subsection{Size limitations}

Current implementation of \LB\ has a~few built-in size limits, mostly related
to schema of underlaying MySQL database. By default, the limits are:

\begin{tabular}{lc}
item & maximum size\\
user certificate subject&255 bytes\\
event attribute (JDL etc.) &16 megabytes\\
tag name&200 bytes\\
tag value&255 bytes\\
job ACL&16 megabytes\\
notification destination&200 bytes\\
\end{tabular}

The restriction on tag value is twofold.
Values up to 16\,MB may be logged and retrieved as raw events.
However, when reported in job state longer values are truncated to 255 bytes.
The restriction is inherited from MySQL limit on index size.

\subsection{Running startup scripts}

Startup scripts of the \LB\ service daemons should not be called without preconditions 
being satisfied, \eg\ one should not try to start a~service when it is already running
or try to restart it from a~cron job without checking whether the machine is shutting down
at the time.

\subsection{Dependencies}

It is strongly discouraged to use \LB\ with different revisions
of external dependencies (MySQL, Globus Toolkit)
than those described in the release documentation. While the \LB\ has been designed to achieve
reasonable backward and forward compatibility, some disruptive changes in minor revisions
of external software have been observed before.
