%
%% Copyright (c) Members of the EGEE Collaboration. 2004-2010.
%% See http://www.eu-egee.org/partners for details on the copyright holders.
%% 
%% Licensed under the Apache License, Version 2.0 (the "License");
%% you may not use this file except in compliance with the License.
%% You may obtain a copy of the License at
%% 
%%     http://www.apache.org/licenses/LICENSE-2.0
%% 
%% Unless required by applicable law or agreed to in writing, software
%% distributed under the License is distributed on an "AS IS" BASIS,
%% WITHOUT WARRANTIES OR CONDITIONS OF ANY KIND, either express or implied.
%% See the License for the specific language governing permissions and
%% limitations under the License.
%
% -*- mode: latex -*-

\section{\LB\ Querying (Consumer) API}
\label{s:Consumer-API}
The \LB Consumer API is used to obtain information from \LB server
or Proxy using simple query language (see
Sect.~\ref{s:querylang}). There are two types of queries based
on the results returned:
\begin{itemize}
\item query for events -- the result contains events satisfying given
criteria,
\item query for jobs -- the result contains JobId's and job states of jobs
satisfying given criteria.
\end{itemize}
The potential result sets can be very large; the \LB server imposes
limits on the result set size, which can be further restricted by the
client.


\subsection{Query Language}
\label{s:querylang}
The \LB query language is based on simple value assertions on job and
event attributes. There are two types of queries based on the
complexity of selection criteria, \textit{simple} and
\textit{complex}.
Simple queries are can be described by the following formula:
\begin{displaymath}
\textit{attr}_1 \mathop{\textrm{ OP }} \textit{value}_1 \wedge \dots \wedge
\textit{attr}_n \mathop{\textrm{ OP }} \textit{value}_n
\end{displaymath}
where $\textit{attr}_i$ is attribute name, $\mathop{\textrm{ OP }}$ is
one of the $=$, $<$, $>$, $\neq$ and $\in$ relational operators and
$\textit{value}$ is single value (or, in the case of $\in$ operator,
interval) from attribute type.

Complex queries can be described using the following formula:
\begin{multline*}
(\textit{attr}_1 \mathop{\textrm{ OP }} \textit{value}_{1,1} \vee \dots \vee
\textit{attr}_1 \mathop{\textrm{ OP }} \textit{value}_{1,i_1}) \wedge \\
(\textit{attr}_2 \mathop{\textrm{ OP }} \textit{value}_{2,1} \vee \dots \vee
\textit{attr}_2 \mathop{\textrm{ OP }} \textit{value}_{2,i_2}) \wedge \\
\vdots \\
\wedge (\textit{attr}_n \mathop{\textrm{ OP }} \textit{value}_{n,1} \vee \dots \vee
\textit{attr}_n \mathop{\textrm{ OP }} \textit{value}_{n,i_n})
\end{multline*}
The complex query can, in contrast to simple query, contain more
assertions on value of single attribute, which are ORed together.

\marginpar{Indexed attributes}%
The query must always contain at least one attribute indexed
on the \LB server; this restriction is necessary to avoid matching the
selection criteria against all jobs in the \LB database. The list of
indexed attributes for given \LB server can be obtained by \LB API
call.

\subsection{C Language Binding}

\subsubsection{Call Semantics}
The \LB server queries are, in contrast to logging event calls,
synchronous (for asynchronous variant see Sect.~\ref{s:Notification-API},
notifications). The server response contains \jobid's, job states
and/or events known to the server at the moment of processing the
query. Due to the asynchronous nature of event delivery it may not
contain all data that was actually sent; the job state computation is
designed to be resilient to event loss to some extent.

\marginpar{Result size limits}%
When the item count returned by \LB\ server exceeds the defined
limits, the \verb'E2BIG' error occur. There are two limits\,---\,the server
and the user limit. The user defined limit may be set in the context
at the client side, while the server imposed limit is configured at
the server and can be only queried by the client. The way the \LB
library and server handles the over--limit result size can be
specified by setting context parameter
\verb'EDG_WLL_PARAM_QUERY_RESULTS' to one of the following values:
\begin{itemize}
\item \verb'EDG_WLL_QUERYRES_NONE'\,---\,In case the limit is reached,
no results are returned at all.
\item \verb'EDG_WLL_QUERYRES_LIMITED'\,---\,A result contains at most
``limit'' item count.
\item \verb'EDG_WLL_QUERYRES_ALL'\,---\,All results are returned and
limits have no effect. This option is available only in special cases
such as ``user jobs query'' and  the ``job status query''. Otherwise
the EINVAL error is returned.
\end{itemize}
Default value is \verb'EDG_WLL_QUERYRES_NONE'.


\subsubsection{Header Files}
\begin{table}[h!]
\begin{tabularx}{\textwidth}{>{\tt}lX}
glite/lb/consumer.h & Prototypes for all query functions. \\
\end{tabularx}
\end{table}

\subsubsection{Context Parameters}
The table~\ref{t:ccontext} shows parameters relevant to the query API.

\begin{table}[h!]
\begin{tabularx}{\textwidth}{lX}
{\bf Name} & {\bf Description} \\
\hline
\lstinline'EDG_WLL_PARAM_QUERY_SERVER' &
Default server name to query.
\\
\lstinline'EDG_WLL_PARAM_QUERY_SERVER_PORT' &
Default server port to query.
\\
\lstinline'EDG_WLL_PARAM_QUERY_SERVER_OVERRIDE' &
host:port parameter setting override even values in \jobid (useful for
debugging \& hacking only)
\\
\lstinline'EDG_WLL_PARAM_QUERY_TIMEOUT' &
Query timeout.
\\
\lstinline'EDG_WLL_PARAM_QUERY_JOBS_LIMIT' &
Maximal query jobs result size.
\\
\lstinline'EDG_WLL_PARAM_QUERY_EVENTS_LIMIT' &
Maximal query events result size.
\\
\lstinline'EDG_WLL_PARAM_QUERY_RESULTS' &
Flag to indicate handling of too large results.
\\
\end{tabularx}
\caption{Consumer specific context parameters}
\label{t:ccontext}
\end{table}


\subsubsection{Return Values}
\LB\ server returns errors which are classified as hard and soft errors.
The main difference between these categories is that in the case of soft
errors results may still be returned. The authorization errors belong to
``soft error'' sort. Hard errors like \verb'ENOMEM' are typically all
unrecoverable, to obtain results the query must be repeated, possibly
after correcting the failure condition the error indicated.

Depending on the setting of context parameter
\verb'EDG_WLL_PARAM_QUERY_RESULTS', the \verb'E2BIG' error may fall into both
categories. 


\subsubsection{Query Condition Encoding}
\label{s:queryrec}
The \LB query language is mapped into (one- or two-dimensional) array
of attribute value assertions represented by
\verb'edg_wll_QueryRec' structure:
\begin{lstlisting}
typedef struct _edg_wll_QueryRec {
        edg_wll_QueryAttr       attr;   //* \textit{attribute to query}
        edg_wll_QueryOp         op;     //* \textit{query operation}
        union {
                char *                  tag;    //* \textit{user tag name / JDL attribute "path"}
                edg_wll_JobStatCode     state;  //* \textit{job status code}
        } attr_id;
        union edg_wll_QueryVal {
                int     i;      	//* \textit{integer query attribute value}
                char    *c;     	//* \textit{character query attribute value}
                struct timeval  t;      //* \textit{time query attribute value}
                glite_jobid_t   j;      //* \textit{JobId query attribute value}
        } value, value2;
} edg_wll_QueryRec;
\end{lstlisting}

% \TODO{pro prehlednost bych mozna pridal seznam vsech atributu na ktere se lze ptat}

The table~\ref{t:cqueryattr} shows the most common query attributes.
For a complete list see \texttt{query\_rec.h}.

\begin{table}[ht]
\begin{tabularx}{\textwidth}{lX}
{\bf Name} & {\bf Description} \\
\hline
\lstinline'EDG_WLL_QUERY_ATTR_JOBID' & Job ID to query. \\
\lstinline'EDG_WLL_QUERY_ATTR_OWNER' & Job owner. \\
\lstinline'EDG_WLL_QUERY_ATTR_STATUS' & Current job status. \\
\lstinline'EDG_WLL_QUERY_ATTR_LOCATION' & Where is the job processed. \\
\lstinline'EDG_WLL_QUERY_ATTR_DESTINATION' & Destination CE. \\
\lstinline'EDG_WLL_QUERY_ATTR_DONECODE' & Minor done status (OK,failed,cancelled). \\
\lstinline'EDG_WLL_QUERY_ATTR_USERTAG' & User tag. \\
\lstinline'EDG_WLL_QUERY_ATTR_JDL_ATTR' & Arbitrary JDL attribute. \\
\lstinline'EDG_WLL_QUERY_ATTR_STATEENTERTIME' & When entered current status. \\
\lstinline'EDG_WLL_QUERY_ATTR_LASTUPDATETIME' & Time of the last known event of the job. \\
\lstinline'EDG_WLL_QUERY_ATTR_JOB_TYPE' & Job type. \\
\end{tabularx}
\caption{Query record specific attributes.}
\label{t:cqueryattr}
\end{table}

The table~\ref{t:cqueryop} shows all supported query operations. 

\begin{table}[ht]
\begin{tabularx}{\textwidth}{lX}
{\bf Name} & {\bf Description} \\
\hline
\lstinline'EDG_WLL_QUERY_OP_EQUAL' & Attribute is equal to the operand value. \\
\lstinline'EDG_WLL_QUERY_OP_LESS' & Attribute is grater than the operand value. \\
\lstinline'EDG_WLL_QUERY_OP_GREATER' & Attribute is less than the operand value. \\
\lstinline'EDG_WLL_QUERY_OP_WITHIN' & Attribute is in given interval. \\
\lstinline'EDG_WLL_QUERY_OP_UNEQUAL' & Attribute is not equal to the operand value. \\
\lstinline'EDG_WLL_QUERY_OP_CHANGED' & Attribute has changed from last check (supported since \LBver{2.0} in notification matching). \\
\end{tabularx}
\caption{Query record specific operations.}
\label{t:cqueryop}
\end{table}



\subsubsection{Query Jobs Examples}
\label{s:qjobs}

The simplest use case corresponds to the situation when an exact job ID
is known and the only information requested is the job status. The job ID
format is described in~\cite{djra1.4}. Since \LBver{2.0}, it is also possible to
query all jobs belonging to a specified user, VO or RB.

The following example shows how to retrieve the status information
about all user's jobs running at a specified CE.

First we have to include neccessary headers:
\lstinputlisting[title={\bf File: }\lstname,numbers=left,linerange=headers-end\ headers]{cons_example1.c}

Define and initialize variables:
\lstinputlisting[title={\bf File: }\lstname,numbers=left,linerange=variables-end\ variables]{cons_example1.c}

Initialize context and set parameters:
\lstinputlisting[title={\bf File: }\lstname,numbers=left,linerange=context-end\ context]{cons_example1.c}

Set the query record to \emph{all (user's) jobs running at CE 'XYZ'}:
\lstinputlisting[title={\bf File: }\lstname,numbers=left,linerange=queryrec-end\ queryrec]{cons_example1.c}

Query jobs:
\lstinputlisting[title={\bf File: }\lstname,numbers=left,linerange=query-end\ query]{cons_example1.c}

Now we can for example print the job states:
\lstinputlisting[title={\bf File: }\lstname,numbers=left,linerange=printstates-end\ printstates]{cons_example1.c}


In many cases the basic logic using only conjunctions is not sufficient.
For example, if you need all your jobs running at the destination XXX or at
the destination YYY, the only way to do this with the \texttt{edg\_wll\_QueryJobs()}
call is to call it twice. The \texttt{edg\_wll\_QueryJobsExt()} call allows to make
such a~query in a single step.
The function accepts an array of condition lists. Conditions within a~single list are
OR-ed and the lists themselves are AND-ed.

The next query example describes how to get all user's jobs running at
CE 'XXX' or 'YYY'. 

We will need an array of three conditions (plus one last empty):

\lstinputlisting[title={\bf File: }\lstname,numbers=left,linerange=variables-end\ variables]{cons_example2.c}

The query condition is the following:

\lstinputlisting[title={\bf File: }\lstname,numbers=left,linerange=queryrec-end\ queryrec]{cons_example2.c}

As can be clearly seen, there are three lists supplied to
\texttt{edg\_wll\_QueryJobsExt()}. The first list specifies the owner of the
job, the second list provides the required status (\texttt{Running}) and
the last list specifies the two destinations.
The list of lists is terminated with \texttt{NULL}.
This query equals to the formula
\begin{quote}
\texttt{(user=NULL) and (state=Running) and (dest='XXX' or dest='YYY')}.
\end{quote}

To query the jobs, we simply call
\lstinputlisting[title={\bf File: }\lstname,numbers=left,linerange=query-end\ query]{cons_example2.c}



\subsubsection{Query Events Examples}

Event queries and job queries are similar. Obviously, the return type is
different \Dash the \LB\ raw events. There is one more input parameter
representing specific conditions on events (possibly empty) in addition to
conditions on jobs.

The following example shows how to select all events (and therefore jobs)
marking red jobs (jobs that were marked red at some time in the past) as green.

\lstinputlisting[title={\bf File: }\lstname,numbers=left,linerange=variables-end\ variables]{cons_example3.c}

\lstinputlisting[title={\bf File: }\lstname,numbers=left,linerange=queryrec-end\ queryrec]{cons_example3.c}

This example uses \texttt{edg\_wll\_QueryEvents()} call. Two condition lists are
given to \texttt{edg\_wll\_QueryEvents()} call. One represents job conditions and
the second represents event conditions. These two lists are joined together with
logical and (both condition lists have to be satisfied). This is necessary as
events represent a state of a job in a particular moment and this changes in time.

\lstinputlisting[title={\bf File: }\lstname,numbers=left,linerange=query-end\ query]{cons_example3.c}

The \texttt{edg\_wll\_QueryEvents()} returns matched events and save them in the
\texttt{eventsOut} variable. Required job IDs are stored in the edg\_wll\_Event
structure.

\lstinputlisting[title={\bf File: }\lstname,numbers=left,linerange=printevents-end\ printevents]{cons_example3.c}

In a similar manor to \texttt{edg\_wll\_QueryJobsExt()}, there exists also \texttt{edg\_wll\_QueryEventsExt()} 
that can be used to more complex queries related to events. See also \texttt{README.queries} for more examples.


Last \LB Querying API call is \texttt{edg\_wll\_JobLog()} that returns all events related to a single job.
In fact, it is a convenience wrapper around \texttt{edg\_wll\_QueryEvents()} and its usage is clearly
demonstrated in the client example \texttt{job\_log.c} (in the client module).



\subsection{C++ Language Binding}
The querying C++ \LB API is modelled after the C \LB API using these basic principles:
\begin{itemize}
\item queries are expressed as vectors of
\verb'glite::lb::QueryRecord' instances,
\item \LB context and general query methods are represented by class
\verb'glite::lb::ServerConnection',
\item \LB job specific queries are encapsulated within class
\verb'glite::lb::Job',
\item query results are returned as (vector or list of)
\verb'glite::lb::Event' or \verb'glite::lb::JobStatus' read-only instances.
\end{itemize}


\subsubsection{Header Files}
Header files for the \LB consumer API are summarized in table~\ref{t:ccppheaders}.
\begin{table}[h]
\begin{tabularx}{\textwidth}{>{\tt}lX}
glite/lb/Event.h & Event class for event query results. \\
glite/lb/JobStatus.h & JobStatus class for job query results. \\
glite/lb/ServerConnection.h & Core of the C++ \LB API, defines
\verb'QueryRecord' class for specifying queries and
\verb'ServerConnection' class for performing the queries. \\
glite/lb/Job.h & Defines \verb'Job' class with methods for job
specific queries. \\
\end{tabularx}
\caption{Consumer C++ API header files}
\label{t:ccppheaders}
\end{table}

\subsubsection{QueryRecord}
The \verb'glite::lb::QueryRecord' class serves as the base for mapping
the \LB query language into C++, similarly to the C counterpart
\verb'edg_wll_QueryRecord'. The \verb'QueryRecord' object represents
condition on value of single attribute:
\begin{lstlisting}
using namespace glite::lb;

QueryRecord a(QueryRecord::OWNER, QueryRecord::EQUAL, "me");
\end{lstlisting}
The \verb'QueryRecord' class defines symbolic names for attributes (in
fact just aliases to \verb'EDG_WLL_QUERY_ATTR_' symbols described in table\
\ref{t:cqueryattr}) and for logical operations (aliases to
\verb'EDG_WLL_QUERY_OP_' symbols, table\ \ref{t:cqueryop}). The last
parameter to the \verb'QueryRecord' constructor is the attribute
value.

There are constructors with additional arguments for specific
attribute conditions or logical operators that require it, \ie\
the \verb'QueryRecord::WITHIN' operator and queries about state enter
times. The query condition ``job that started running between \verb'start'
and \verb'end' times' can be represented in the following way:
\begin{lstlisting}
struct timeval start, end;

QueryRecord a(QueryRecord::TIME, QueryRecord::WITHIN, JobStatus::RUNNING, 
              start, end);
\end{lstlisting}


\subsubsection{Event}
The objects of class \verb'glite::lb::Event' are returned by the \LB event
queries. The \verb'Event' class intgstr	roduces symbolic names for event
type (enum \verb'Event::Type'), event attributes (enum
\verb'Event::Attr') and their types (enum
\verb'Event::AttrType'), feature not available through the C API, as
well as (read only) access to the attribute values.  Using 
these methods you can:
\begin{itemize}
\item get the event type (both symbolic and string):
\begin{lstlisting}
      Event event;

      // we suppose event gets somehow filled in
      cout << "Event type: " << event.type << endl;
   
      cout << "Event name:" << endl;
      // these two lines should print the same string
      cout << Event::getEventName(event.type) << endl;
      cout << event.name() << endl;
\end{lstlisting}
\item get the list of attribute types and values (see
line~\ref{l:getattrs} of the example),
\item get string representation of attribute names,
\item get value of given attribute.
\end{itemize}
The following example demonstrates this by printing event name and attributes:
\lstinputlisting[title={\bf File:}\lstname,numbers=left,linerange=event-end\ event]{util.C}

\subsubsection{JobStatus}
The \verb'glite::lb::JobStatus' is a result type of job status
queries in the same way the \verb'glite::lb::Event' is used in event
queries. The \verb'JobStatus' class provides symbolic names for job
states (enum \verb'JobStatus::Code'), state attributes
(enum \verb'JobStatus::Attr') and their types (enum
\verb'JobStatus::AttrType'), and read only access to the
attribute values. Using the \verb'JobStatus' interface you can:
\begin{itemize}
\item get the string name for the symbolic job state:
\begin{lstlisting}
	JobStatus status;

	// we suppose status gets somehow filled in
        cout << "Job state: " << status.type << endl;

        cout << "State name: " << endl;
        // these two lines should print the same string
        cout << JobStatus::getStateName(status.type) << endl;
        cout << status.name() << endl;
\end{lstlisting}
\item get the job state name (both symbolic and string),
\item get the list of job state attributes and types,	
\item convert the attribute names from symbolic to string form and
vice versa,
\item get value of given attribute.
\end{itemize}
The following example demostrates this by printing job status (name
and attributes):
\lstinputlisting[title={\bf File:}\lstname,numbers=left,linerange=status-end\ status]{util.C}

\subsubsection{ServerConnection}\label{s:ServerConnection}
The \verb'glite::lb::ServerConnection' class represents particular \LB
server and allows for queries not specific to particular job (these
are separated into \verb'glite::lb:Job' class). The
\verb'ServerConnection' instance thus encapsulates client part of
\verb'edg_wll_Context' and general query methods.

There are accessor methods for every consumer context parameter listed
in table \ref{t:ccontext}, \eg for \verb'EDG_WLL_PARAM_QUERY_SERVER'
we have the following methods:
\begin{lstlisting}
void setQueryServer(const std::string& host, int port);
std::pair<std::string, int> getQueryServer() const;
\end{lstlisting}
We can also use the generic accessors defined for the parameter types
\verb'Int', \verb'String' and \verb'Time', \eg:
\begin{lstlisting}
void setParam(edg_wll_ContextParam name, int value);
int getParamInt(edg_wll_ContextParam name) const;
\end{lstlisting}

The \verb'ServerConnection' class provides methods for both event and job queries:
\begin{lstlisting}
void queryJobs(const std::vector<QueryRecord>& query,
	       std::vector<glite::jobid::JobId>& jobList) const;

void queryJobs(const std::vector<std::vector<QueryRecord> >& query,
	       std::vector<glite::jobid::JobId>& jobList) const;

void queryJobStates(const std::vector<QueryRecord>& query, 
		    int flags,
		    std::vector<JobStatus> & states) const;

void queryJobStates(const std::vector<std::vector<QueryRecord> >& query, 
		    int flags,
		    std::vector<JobStatus> & states) const;

void queryEvents(const std::vector<QueryRecord>& job_cond,
		 const std::vector<QueryRecord>& event_cond,
		 std::vector<Event>& events) const;

void queryEvents(const std::vector<std::vector<QueryRecord> >& job_cond,
		 const std::vector<std::vector<QueryRecord> >& event_cond,
		 std::vector<Event>& eventList) const;
\end{lstlisting}
You can see that we use \verb'std::vector' instead of \verb'NULL' terminated
arrays for both query condition lists and results. The API does
not differentiate simple and extended queries by method name
(\verb'queryJobs' and \verb'queryJobsExt' in C), but by parameter
type (\verb'vector<QueryRecord>'
vs. \verb'vector<vector<QueryRecord>>'). On the other hand there are
different methods for obtaining \jobid's and full job states as well 
as convenience methods for getting user jobs. 

Now we can show the first example of job query from section
\ref{s:qjobs} rewritten in C++. First we have to include the headers:
\lstinputlisting[title={\bf File: }\lstname,numbers=left,linerange=headers-end\ headers]{cons_example1.cpp}

Define variables:
\lstinputlisting[title={\bf File: }\lstname,numbers=left,linerange=variables-end\ variables]{cons_example1.cpp}

Initialize server object:
\lstinputlisting[title={\bf File: }\lstname,numbers=left,linerange=queryserver-end\ queryserver]{cons_example1.cpp}

Create the query condition vector:
\lstinputlisting[title={\bf File: }\lstname,numbers=left,linerange=querycond-end\ querycond]{cons_example1.cpp}

Perform the query:
\lstinputlisting[title={\bf File: }\lstname,numbers=left,linerange=query-end\ query]{cons_example1.cpp}

Print the results:
\lstinputlisting[title={\bf File: }\lstname,numbers=left,linerange=printstates-end\ printstates]{cons_example1.cpp}

The operations can throw an exception, so the code should be enclosed
within try--catch clause.

The second example rewritten to C++ is shown here; first the query
condition vector:
\lstinputlisting[title={\bf File: }\lstname,numbers=left,linerange=queryrec-end\ queryrec]{cons_example2.cpp}

The query itself:
\lstinputlisting[title={\bf File: }\lstname,numbers=left,linerange=query-end\ query]{cons_example2.cpp}

The third example shows event query (as opposed to job state query in
the first two examples). We are looking for events of jobs, that were
in past painted (tagged by user) green, but now they are red. The
necessary query condition vectors are here:
\lstinputlisting[title={\bf File: }\lstname,numbers=left,linerange=queryrec-end\ queryrec]{cons_example3.cpp}

The query itself:
\lstinputlisting[title={\bf File: }\lstname,numbers=left,linerange=query-end\ query]{cons_example3.cpp}

The resulting event vector is dumped using the utility function
\verb'dumpEvent()' listed above:
\lstinputlisting[title={\bf File: }\lstname,numbers=left,linerange=printevents-end\ printevents]{cons_example3.cpp}


\subsubsection{Job}
The \verb'glite::lb::Job' class encapsulates \LB server queries
specific for particular job as well as client part of context. The
\verb'Job' object provides method for getting the job status and the
event log (\ie all events belonging to the job):
\begin{lstlisting}
JobStatus status(int flags) const;

void log(std::vector<Event> &events) const;
\end{lstlisting}

\marginpar{\bf Important!}%
It is important to notice that \verb'Job' contain
\verb'ServerConnection' as private member and thus encapsulate client
part of context. That makes them relatively heavy--weight objects and
therefore it is not recommended to create too many instances, but
reuse one instance by assigning different \jobid's to it.


\subsection{Web-Services Binding}\label{s:Consumer-API-WS}

\TODO{ljocha: Complete review, list of all relevant (WSDL) files, their location, etc.}

In this section we describe the operations defined in the \LB\ WSDL
file (\texttt{LB.wsdl}) as well as its custom types (\texttt{LBTypes.wsdl}).

For the sake of readability this documentation does not follow the structure
of WSDL strictly, avoiding to duplicate information which is already present
here. Consequently, the SOAP messages are not documented, for example, as they
are derived from operation inputs and outputs mechanically.
The same holds for types: \eg\ we do not document defined elements
which correspond 1:1 to types but are required due to the literal SOAP
encoding.

For exact definition of the operations and types see the WSDL file.

\TODO{ljocha: Add fully functional WS examples - in Java, Python, C?}


Aby se na to neapomnelo:

perl-SOAP-Lite-0.69 funguje
perl-SOAP-Lite-0.65 ne 	(stejne rve document/literal support is EXPERIMENTAL in SOAP::Lite ), tak ma asi pravdu


musi mit metodu ns()

