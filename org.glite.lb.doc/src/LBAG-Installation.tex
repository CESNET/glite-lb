\section{Installation and Configuration}

\subsection{Complete list of packages}
%\subsection{Daemons description}
%\subsection{CLI tools description: purge/dump/load}

\LB is currently distributed mainly in RPMs packages. It is available also in
binary form packed as .tar.gz. Recent attempts to multiplatform porting and
recent ETICS building system development promise a future possibility to
distribute the software in other distribution formats, e.g. DEB packages. 

In \LBold, the list of all LB packages was the following:

\TODO{ljocha: vylepsit popisky, taky do zdrojaku}

\begin{tabularx}{\textwidth}{>{\tt}lX}
glite-lb-common & LB common files \\ 
glite-lb-client & LB client library \\ 
glite-lb-client-interface & LB client library interface (header files) \\ 
glite-lb-logger & LB local-logger end inter-logger \\ 
glite-lb-proxy & LB proxy \\ 
glite-lb-server & LB server \\ 
glite-lb-server-bones & LB server bones (common to proxy and server) \\ 
glite-lb-utils & LB utilities \\ 
glite-lb-ws-interface & LB Web Services Interface 
\end{tabularx}

In \LBnew, the code has been restructured quite a lot, especially the dependencies were lightened,
and the new list of packages is now the following:

\begin{tabularx}{\textwidth}{>{\tt}lX}
glite-lb-doc & LB documentation \\ 
glite-lb-common & LB common files \\ 
glite-lb-client & LB client library \\ 
glite-lb-logger & LB local-logger end inter-logger \\
glite-lb-server & LB server \\
glite-lb-state-machine & LB state machine (LB plugin) \\ 
glite-lb-utils & LB utilities \\
glite-lb-ws-interface & LB Web Services Interface      
\end{tabularx}

More detailed description together with the dependencies can be read directly from each package,
for example by issuing the command 
\begin{verbatim}
   rpm -qiR <package_name>
\end{verbatim}

Some of the LB packages may depend among others also on the following packages
that are worth to mention in this documentation:

\begin{tabularx}{\textwidth}{>{\tt}lX}
glite-jobid-api-c & gLite jobId C API library \\ 
glite-jp-common & JP common files \\ 
glite-lbjp-common-db & database (MySQL) backend \\
glite-lbjp-common-maildir & maildir backend common files \\
glite-lbjp-common-server-bones & server bones (common files) \\
glite-lbjp-common-trio & trio printf modification \\
glite-security-gsoap-plugin & gSOAP plugin \\ 
glite-security-gss & GSS client/server C API library \\
\end{tabularx}

where all \verb'glite-lbjp-common-*' packages are common both to \LB and 
Job Provenance (\JP).


\subsection{\LB server}

\subsubsection{Hardware requirements}
\label{inst:hw_req}

Hardware requirements depend on performance and storage time requirements.
Disk space used by LB server consists of database space and working space 
for backup dumps and temporary files used by exports to Job Provenance and
R-GMA tables. Necessary database space can be calculated by multiplying 
job retention length (job purge timeout), job submission rate, and  
per-job space requirements (120\,KB per job is recommended for current common 
EGEE usage pattern; jobs can consume more than that with use of very long
JDL descriptions, user tags, or very high number of resubmissions).
For temporary files, approximately 10\,GB is sufficient for LB server setups
working normally, more can be needed when backlog forms in data export
to any external service. For example, typical setup processing 40\,000 jobs per 
day where all jobs are purged after 10 days needs about 58
($10 \cdot 40000 \cdot 0.000000120 + 10$) gigabytes (not accounting for operating 
system and system logs).

\TODO{salvet: vylepsit citelnost formulky}

For smooth handling of 40\,000 jobs/day, this or better machine configuration 
is necessary:
\begin{itemize}
\item 1GB RAM
\item CPU equivalent of single Xeon/Opteron 1.5GHz
\item single 7200 rpm SATA disk.
\end{itemize}
In order to achieve higher performace, following changes are recommended:
\begin{itemize}
\item Faster disks. Disk access speed is crucial for LB server, couple of 15k rpm
SCSI or SAS disks (one for MySQL database data file, the second for DB logs, LB server's
working directories, and operating system files) or RAID with battery backed 
write-back cache is preferable.
\item More memory. Large RAM improves performance through memory caching,
relative speed gain is likely to be rougly proportional to memory/database size ratio.
To use 3\,GB or more efficiently, 64bit OS and MySQL server versions are recommended.
\item Faster or more processors. CPU requirements scale approximately linearly with
offered load.
\end{itemize}

\subsubsection{Standard installation}

\TODO{valtri: odkazy na metapackage a repository, zkusit najit oficialni gLite dokumentaci}

After installation and configuration of OS and basic services
(certificates, CAs, time synchronization), glite-LB metapackage 
from appropriate gLite sotware repository should be installed.
YAIM configuration for \emph{glite-LB} node type 
(\texttt{/opt/glite/yaim/bin/yaim -c -s site-info.def -n glite-LB}) 
can done then. Available parameters specific to LB server are:

%variable&meaning&default value &further details\\
\begin{itemize}
\item \texttt{MYSQL\_PASSWORD} -- root password of MySQL server (mandatory)
\item \texttt{GLITE\_WMS\_LCGMON\_FILE} -- pathname of file where job state
export data are written for use by lgcmon/R-GMA 
(default: \texttt{/var/glite/logging/status.log}
\end{itemize}
\TODO{add more parameters to YAIM module}

In addition to those, YAIM LB module uses following parameters:
\texttt{INSTALL\_ROOT}, \texttt{GLITE\_LOCATION\_VAR}, \texttt{GLITE\_USER}, \texttt{GLITE\_USER\_HOME},\texttt{SITE\_EMAIL}.

\subsubsection{Migration from previous versions}

\TODO{salvet}

updaty databaze, kdy jak, migrace na 3.1 i na HEAD

transakce


\subsubsection{Index configuration}

\TODO{Initial YAIM way only, rest in Sect.~\ref{maintain:index}}

\subsubsection{Server superusers}
\label{inst:superusers}

Certain administrative operations (identified below when appropriate) on \LB
server are privileged and special authorization is required to invoke them.  By
default, the \LB server identity (X509 certificate subject name) is considered
privileged.  Additional administrator identitites can be specified in a
\emph{superusers file}, specified by the \verb'--super-users-file' server
option.  A client is granted superuser privileges if they present credentials
matching the superusers specifications in the file.  The file consists of one
or more lines, each one containing either a subject name or VOMS attribute(s)
in the FQAN format (in the latter case the line must start with \verb'FQAN:').
After changing the file, the server has to be restarted. 

The default startup script checks for existence of 
/opt/glite/etc/LB-super-users and uses it eventually.


\subsubsection{Notification delivery}

\subsubsection{Export to R-GMA}

\subsubsection{Data backup}
\label{inst:backup}

\subsubsection{Purging old data}
\label{inst:purge}

\TODO{Setup cron job, refer to Sect.~\ref{maintain:purge}}

\subsubsection{Export to Job Provenance}


\subsubsection{Exploiting parallelism}

\subsubsection{Tuning database engine}
\label{inst:db_tuning}

In order to achieve high performance with LB server underlaying MySQL 
database server has to be configured reasonably well too. 
Default values of some MySQL settings are likely to be suboptimal
and need tuning, especially for larger machines.
These are MySQL configuration variables (to be configured in \texttt{[mysqld]} 
section of \texttt{/etc/my.cnf}) that need tuning most often:
\begin{itemize}
\item \texttt{innodb\_buffer\_pool\_size} -- size of database memory pool/cache. 
It is generally recommended to set it to aroung 75\% of RAM size
(32bit OS/MySQL versions limit this to approx. 2GB due to address space 
constraints).

\item \texttt{innodb\_flush\_logs\_at\_trx\_commit} -- frequency of flushing to disk.
Recommended values include:
\begin{itemize}
\item 1 (default) -- flush at each write transaction commit; relatively
slow without battery-backed disk cache but offers highest level of data safety
\item 0 -- flush once per second; fast, use if loss of latest updates on MySQL
or OS crash (e.g. unhandled power outage) is acceptable (database remains consistent)
\end{itemize}

\item \texttt{innodb\_log\_file\_size} -- size of database log file. Larger values
save some I/O activity, but also make database shutdown and crash recovery slower.
Recommended value: 50MB. Clean mysqld shutdown and deletion of log files 
(\texttt{/var/lib/mysql/ib\_logfile*} by default) is necessary before change.

\item \texttt{innodb\_data\_file\_path} -- path to main database file. File on
disk separate from OS and MySQL log files (\texttt{innodb\_log\_group\_home\_dir} variable,
\texttt{/var/lib/mysql/} by default) is recommended.

\end{itemize}

\subsection{\LB proxy}

Namergovat do serveru

\subsection{\LB logger}


\subsection{Smoke tests}

\TODO{ljocha: get something from the old testing docs}

Thorough tests of \LB, including performance measurement, are
covered in the \LB Test Plan document \cite{lbtp}.
This section describes only elementary tests that verify basic
functionality of the services.

The following test description assumes the \LB services installed
and started as described above.

\def\req{\noindent\textbf{Prerequisities:}\xspace}
\def\how{\noindent\textbf{How to run:}\xspace}
\def\result{\noindent\textbf{Expected result:}\xspace}

\subsubsection{Job registration}

Register a~new job with the \LB server and check that its status is
reported correctly.

\req Installed glite-lb-client package, valid user's X509 credentials,
known destination (address:port) of running \LB server.
Can be invoked from any machine.

\how 
\begin{quote}
\verb'/opt/glite/examples/glite-lb-job_reg -m' \emph{my.server.name:port}
\end{quote}
A~new jobid is generated and printed.  Run 
\begin{quote}
\verb'/opt/glite/examples/glite-lb-job_status' \emph{the\_\/new\_\/jobid}
\end{quote}

\result
The command should report ``Submitted'' job status.

\subsubsection{Logging events via lb-logger}

Send several \LB events, simulating normal job life cycle.
Checks delivery of the events via \LB logger.

\req Installed glite-lb-client package, valid user's X509 credentials,
known destination (address:port) of running \LB server.
Must be run on a~machine where glite-lb-logger package is set up and running.

\how
\begin{quote}
\verb'/opt/glite/examples/glite-lb-running.sh -m ' \emph{my.server.name:port}
\end{quote}

The command prints a~new jobid, followed by diagnostic messages as the events are logged. 
Check the status of the new job with
\begin{quote}
\verb'/opt/glite/examples/glite-lb-job_status' \emph{the\_\/new\_\/jobid}
\end{quote}

\result
Due to asynchronous event
delivery various job states can be reported for limited time (several seconds).
Finally the
``Running'' status should be reached.

\subsubsection{Logging events via lb-proxy}

Send events via \LB proxy. Checks the proxy functionality.

\TODO{}

\subsubsection{Notification delivery}

Register for receiving notifications, and log events which trigger
the notification delivery. Checks the whole notification mechanism.

\TODO{mozna jen odkaz do User Guide}
