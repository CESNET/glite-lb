%
%% Copyright (c) Members of the EGEE Collaboration. 2004-2010.
%% See http://www.eu-egee.org/partners for details on the copyright holders.
%% 
%% Licensed under the Apache License, Version 2.0 (the "License");
%% you may not use this file except in compliance with the License.
%% You may obtain a copy of the License at
%% 
%%     http://www.apache.org/licenses/LICENSE-2.0
%% 
%% Unless required by applicable law or agreed to in writing, software
%% distributed under the License is distributed on an "AS IS" BASIS,
%% WITHOUT WARRANTIES OR CONDITIONS OF ANY KIND, either express or implied.
%% See the License for the specific language governing permissions and
%% limitations under the License.
%
\def\lbdoc{\LB Documentation and versions overview}
\section*{\lbdoc}
\addcontentsline{toc}{section}{\texorpdfstring{\uppercase{\lbdoc}}{\lbdoc}}
\def\lbdoc{}

The Logging and Bookkeeping service (\LB\ for short) was initially developed in
the EU DataGrid
project\footnote{\url{http://eu-datagrid.web.cern.ch/eu-datagrid/}} as a~part
of the Workload Management System (WMS). The development continued in the EGEE,
EGEE-II and EGEE-III projects,\footnote{\url{http://www.eu-egee.org/}} where
\LB became an independent part of the
gLite\footnote{\url{http://www.glite.org}} middleware~\cite{glite}, and then in the EMI Project.\footnote{\url{http://www.eu-emi.eu/}}

The complete \LB Documentation consists of the following parts:
\begin{itemize}
\item \textbf{\LB User's Guide} 
   \ifx\insideUG\undefined{\cite{lbug}}\else{-- this document}\fi. 
   %
%% Copyright (c) Members of the EGEE Collaboration. 2004-2010.
%% See http://www.eu-egee.org/partners for details on the copyright holders.
%% 
%% Licensed under the Apache License, Version 2.0 (the "License");
%% you may not use this file except in compliance with the License.
%% You may obtain a copy of the License at
%% 
%%     http://www.apache.org/licenses/LICENSE-2.0
%% 
%% Unless required by applicable law or agreed to in writing, software
%% distributed under the License is distributed on an "AS IS" BASIS,
%% WITHOUT WARRANTIES OR CONDITIONS OF ANY KIND, either express or implied.
%% See the License for the specific language governing permissions and
%% limitations under the License.
%
% when changed, update also http://egee.cesnet.cz/en/JRA1/LB/lb.html
% (in CVSROOT=:gserver:lindir.ics.muni.cz:/cvs/edg, cvsweb/lb.html)
The User's Guide explains how to use the Logging and Bookkeeping (\LB) service
from the user's point of view. The service architecture is described
thoroughly. Examples on using \LB's event logging commands to log user tags and
change job ACLs are given, as well as \LB query and notification use cases.

\item \textbf{\LB Administrator's Guide} 
   \ifx\insideAG\undefined{\cite{lbag}}\else{-- this document}\fi. 
   % when changed, update also http://egee.cesnet.cz/en/JRA1/LB/lb.html
% (in CVSROOT=:gserver:lindir.ics.muni.cz:/cvs/edg, cvsweb/lb.html)
This administrator's guide explains how to administer the Logging and
Bookkeeping (\LB) service. Several deployment scenarios are described together
with the installation, configuration, running and troubleshooting steps.

\item \textbf{\LB Developer's Guide} 
   \ifx\insideDG\undefined{\cite{lbdg}}\else{-- this document}\fi. 
   %
%% Copyright (c) Members of the EGEE Collaboration. 2004-2010.
%% See http://www.eu-egee.org/partners for details on the copyright holders.
%% 
%% Licensed under the Apache License, Version 2.0 (the "License");
%% you may not use this file except in compliance with the License.
%% You may obtain a copy of the License at
%% 
%%     http://www.apache.org/licenses/LICENSE-2.0
%% 
%% Unless required by applicable law or agreed to in writing, software
%% distributed under the License is distributed on an "AS IS" BASIS,
%% WITHOUT WARRANTIES OR CONDITIONS OF ANY KIND, either express or implied.
%% See the License for the specific language governing permissions and
%% limitations under the License.
%
% when changed, update also http://egee.cesnet.cz/en/JRA1/LB/lb.html
% (in CVSROOT=:gserver:lindir.ics.muni.cz:/cvs/edg, cvsweb/lb.html)
This developer's guide explains how to use the Logging and Bookkeeping (\LB)
service API. Logging (producer), querying (consumer) and notification API as
well as the Web Services Interface is described in details together with
programing examples.

\item \textbf{\LB Test Plan} 
   \ifx\insideTP\undefined{\cite{lbtp}}\else{-- this document}\fi. 
   This test plan document explains how to test the Logging and Bookkeeping (\LB)
service. Tests are described at six different layers, from elementary tests if
the service is up and running, through tests of the fully supported
functionality, performance and stress tests to interoperability tests.

\end{itemize}

\textbf{The following versions of \LB service are covered by these documents:}
\begin{itemize}
%\item \LBver{x.x}: included in the EMI-3 \emph{Monte Bianco} release
\item \LBver{3.2}: included in the EMI-2 \emph{Matterhorn} release
\item \LBver{3.1}: an update for the EMI-1 \emph{Kebnekaise} release
\item \LBver{3.0}: included in the EMI-1 \emph{Kebnekaise} release
\item \LBver{2.1}: replacement for \LBver{2.0} in gLite 3.2
\item \LBver{2.0}: included in gLite 3.2 release
\item \LBver{1.x}: included in gLite 3.1 release
\end{itemize}

\textbf{\LB packages can be obtained from two distinguished sources:}

\nopagebreak
\begin{itemize}
\item \textbf{gLite releases}: gLite node-type repositories, offering a specific repository for each node type such as \emph{glite-LB}. Only binary RPM packages are available from that source.
\item \textbf{emi releases}: EMI repository\footnote{\url{http://emisoft.web.cern.ch/emisoft/}} or EGI's UMD repository,\footnote{\url{http://repository.egi.eu/}} offering all EMI middleware packages from a single repository. There are RPM packages, both source and binary, the latter relying on EPEL for dependencies. There are also DEB packages (starting with EMI-2) and \texttt{tar.gz} archives.
\end{itemize}

\emph{Note:} Despite offering the same functionality, binary packages obtained from different repositories differ and switching from one to the other for upgrades may not be altogether straightforward.

Updated information about \LB service (including the \LB service roadmap) is available at the
\LB homepage:
\href{http://egee.cesnet.cz/en/JRA1/LB}{http://egee.cesnet.cz/en/JRA1/LB}

