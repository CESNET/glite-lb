%
%% Copyright (c) Members of the EGEE Collaboration. 2004-2010.
%% See http://www.eu-egee.org/partners for details on the copyright holders.
%% 
%% Licensed under the Apache License, Version 2.0 (the "License");
%% you may not use this file except in compliance with the License.
%% You may obtain a copy of the License at
%% 
%%     http://www.apache.org/licenses/LICENSE-2.0
%% 
%% Unless required by applicable law or agreed to in writing, software
%% distributed under the License is distributed on an "AS IS" BASIS,
%% WITHOUT WARRANTIES OR CONDITIONS OF ANY KIND, either express or implied.
%% See the License for the specific language governing permissions and
%% limitations under the License.
%
\section{User tools}
\label{s:lb-tools}

In this section we give a description of the CLI tools that a regular grid user
might want to use. If not stated otherwise, the tools are distributed in the
\verb'glite-lb-client' package.

\subsection{Environment variables}

Behaviour of the commands can be changed by setting enviroment variables, specifing mostly
location of servers or setting timeouts for various operations. 
For a complete list of environment variables, their form and default value
description, see Appendix~\ref{a:environment}. Setting the environment variable
is for some commands mandatory, so reading the documentaion below and
appropriate manpages is highly recommended.


\subsection{glite-wms-job-status and glite-wms-job-logging-info}

We start with tools that are distributed in \verb'glite-wms-ui-cli-python' package
and that can be found probably on all UI machines. Description of all user
commands that are used during the job submission process (generating proxy,
creating a JDL describing the job, submitting a job, retrieving output,
cancelling a job, etc.) is beoynd this document and it can be found for example
in \cite{wmsug}. We mention here only the commands that are related to
job monitoring.

Once job has been submitted to WMS (by \verb'glite-wms-job-submit'), 
a user can retrieve the job status by
\begin{verbatim}
    glite-wms-job-status <jobId>
\end{verbatim}
or if job ID is saved in the file
\begin{verbatim}
    glite-wms-job-status -i <job_id_file>
\end{verbatim}
or if user wants to see status of all his/her jobs
\begin{verbatim}
    glite-wms-job-status --all
\end{verbatim}
List of all possible job states is summarised in Appendix \ref{a:jobstat}.

Logging details on submitted job can be accessed via
\begin{verbatim}
    glite-wms-job-logging-info -v <verbosity_level> <job_ID>
\end{verbatim}
or if job ID is saved in the file
\begin{verbatim}
   glite-wms-job-logging-info -v <verbosity_level> -i <job_id_file>
\end{verbatim}
where verbosity level can be from 0 to 3. 


\subsection{glite-lb-logevent}
\label{glite-lb-logevent}

Besides the API's \LB\ offers its users a simple command-line interface for
logging events. The command glite-lb-logevent is used for this purpose. However, it
is intended for internal WMS debugging tests in the first place and should not
be used for common event logging because of possibility of confusing \LB\
server job state automaton.

The only legal user usage is for logging \verb'UserTag' and \verb'ChangeACL' events. The following description is therefore concentrating only on options dealing with these two events.

Command usage is:

\begin{verbatim}
    glite-lb-logevent [-h] [-p] [-c seq_code] \
        -j <dg_jobid> -s Application -e <event_name> [key=value ...]
\end{verbatim}

where

\begin{tabularx}{\textwidth}{lX}
\texttt{  -h  -{}-help} &           this help message\\
\texttt{  -p  -{}-priority} &       send a priority event\\
\texttt{  -c  -{}-sequence} &       event sequence code\\
\texttt{  -j  -{}-jobid} &          JobId\\
\texttt{  -e  -{}-event} &           select event type (see -e help)\\
\end{tabularx}

\medskip

Each event specified after \verb'-e' option has different sub-options enabling to set event specific values.

Sub-options usable with \verb'UserTag' event are:


\begin{tabularx}{\textwidth}{lX}
\texttt{      -{}-name}  &          tag name\\
\texttt{      -{}-value} &          tag value\\
\end{tabularx}

\medskip

Sub-options usable with \verb'ChangeACL' event are:

\begin{tabularx}{\textwidth}{lX}
\texttt{      -{}-operation} &       operation requested to perform with ACL (add, remove)\\
\texttt{      -{}-permission} &      ACL permission to change (currently only READ)\\
\texttt{      -{}-permission\_type} & type of permission requested (0 = 'allow', 1 = 'deny')\\
\texttt{      -{}-user\_id} &         DN or VOMS parameter (in format VO:group)\\
\texttt{      -{}-user\_id\_type} &    type of information given in \verb'user_id' (DN or VOMS)\\
\end{tabularx}

\bigskip

To be able to use this command several environmental variables must be set properly. User must specify where the event should be sent. This is address and port of glite-lb-logd daemon. It is done using environmental variable \verb'EDG_WL_LOG_DESTINATION' in a form \verb'address:port'.

Because user is allowed to change ACL or add user tags only for her jobs, paths to valid X509 user credentials has to be set to authorise her. This is done using two environmental variables \verb'EDG_WL_X509_KEY' and \verb'EDG_WL_X509_CERT' in a form \verb'path_to_cred'.


%
%% Copyright (c) Members of the EGEE Collaboration. 2004-2010.
%% See http://www.eu-egee.org/partners for details on the copyright holders.
%% 
%% Licensed under the Apache License, Version 2.0 (the "License");
%% you may not use this file except in compliance with the License.
%% You may obtain a copy of the License at
%% 
%%     http://www.apache.org/licenses/LICENSE-2.0
%% 
%% Unless required by applicable law or agreed to in writing, software
%% distributed under the License is distributed on an "AS IS" BASIS,
%% WITHOUT WARRANTIES OR CONDITIONS OF ANY KIND, either express or implied.
%% See the License for the specific language governing permissions and
%% limitations under the License.
%
\subsection{glite-lb-notify}
\label{s:lb-notify}

\verb'glite-lb-notify' is a fairly simple wrapper on the \LB notification API
(see also \cite{lbdg}).
It allows to create a notification (with a restricted richness of specifying 
conditions), bind to it for receiving notifications, and drop it finally.

\LB notification is a user-initiated trigger at the server.
Whenever a job enters a state matching conditions specified with the notification,
the current state of the job is sent to the notification client.
On the other hand, the notification client is a network listener
which receives server-initiated connections to get the notifications.
Unless \verb'-s' is specified, the notification library creates the listener
socket.

Within the notification validity, clients can disappear and even migrate.
However, only a single active client for a notification is allowed.

\LB server and port to contact must be specified with GLITE\_WMS\_NOTIF\_SERVER 
environment variable.

\verb'glite-lb-notify' is supported by \LBver{2.x} and newer. In \LBver{1.x}, \verb'glite-lb-notify' 
with very limited functionality can be found in \verb'examples' directory.

\verb'glite-lb-notify' support these actions:

\begin{tabularx}{\textwidth}{lX}
\texttt{new} & Create new notification registration.\\
\texttt{bind} &  Binds an notification registration to a client.\\
\texttt{refresh} &  Enlarge notification registration validity.\\
\texttt{receive}  & Binds to an existing notification registration and listen to
server.\\
\texttt{drop}     & Drop the notification registration.\\
\end{tabularx}

For action \verb'new', command usage is:

\begin{verbatim}
  glite-lb-notify new [ { -s socket_fd | -a fake_addr } -t requested_validity
           -j jobid  { -o owner | -O }  -n network_server 
           -v virtual_organization --states state1,state2,... -c -f flags]
\end{verbatim}

For action \verb'bind', command usage is:
\begin{verbatim}
  glite-lb-notify bind [ { -s socket_fd | -a fake_addr } -t requested_validity ] 
           notifid
\end{verbatim}

For action \verb'refresh', command usage is:
\begin{verbatim}
  glite-lb-notify refresh [-t requested_validity ] notifid
\end{verbatim}

For action \verb'receive', command usage is:
\begin{verbatim}
  glite-lb-notify receive  [ { -s socket_fd | -a fake_addr } ] [-t requested_validity ] [-i timeout] [-r ] [-f field1,field2,...] [notifid]
\end{verbatim}

For action \verb'drop', command usage is:
\begin{verbatim}
   glite-lb-notify drop notifid
\end{verbatim}

where

\begin{tabularx}{\textwidth}{lX}
\texttt{  notifid} & Notification ID \\
\texttt{  -s socket\_fd} &  allows  to  pass  a opened listening socket  \\
\texttt{  -a fake\_addr} &  fake the client address \\
\texttt{  -t requested\_validity} & requested validity of the notification (in seconds)   \\
\texttt{  -j jobid} & job ID to connect notification registration with   \\
\texttt{  -o owner} & match this owner DN   \\
\texttt{  -O} & match owner on current user's DN \\
\texttt{  -n network\_server} &  match only this network server (WMS entry point)  \\
\texttt{  -v virtual\_organization} & match only jobs of this Virtual Organization  \\
\texttt{  -i timeout} & timeout to receive operation in seconds   \\
\texttt{  -f field1,field2,...} & list of status fields to print (only owner by default)   \\
\texttt{  -c} & notify only on job state change \\
\texttt{  -S, --state state1,state2,...} & match on events resulting in listed states   \\
\texttt{  -r} & refresh automatically the notification registration while receiving data\\
\end{tabularx}

For additional information see also manual page glite-lb-notify(1).

\subsubsection{Example: Registration and waiting for simple notification}
\label{e:notify}

Following steps describe basic testing procedure of the notification
system by registering a notification on any state change of this job
and waiting for notification.

Register notification for a given jobid
%(\verb'https://skurut68-2.cesnet.cz:9100/D1qbFGwvXLnd927JOcja1Q'), 
with validity 2 hours (7200 seconds):

\begin{verbatim}
  GLITE_WMS_NOTIF_SERVER=skurut68-2.cesnet.cz:9100 glite-lb-notify new \
   -j https://skurut68-2.cesnet.cz:9100/D1qbFGwvXLnd927JOcja1Q -t 7200
\end{verbatim}

returns:

\begin{verbatim}
  notification ID: https://skurut68-2.cesnet.cz:9100/NOTIF:tOsgB19Wz-M884anZufyUw 
\end{verbatim}


Wait for notification (with timeout 120 seconds):
\begin{verbatim}
  GLITE_WMS_NOTIF_SERVER=skurut68-2.cesnet.cz:9100 glite-lb-notify receive \
   -i 120 https://skurut68-2.cesnet.cz:9100/NOTIF:tOsgB19Wz-M884anZufyUw 
\end{verbatim}

returns:
\begin{verbatim}
  notification is valid until: '2008-07-29 15:04:41' (1217343881)
  https://skurut68-2.cesnet.cz:9100/D1qbFGwvXLnd927JOcja1Q        Waiting
      /DC=cz/DC=cesnet-ca/O=Masaryk University/CN=Miroslav Ruda
  https://skurut68-2.cesnet.cz:9100/D1qbFGwvXLnd927JOcja1Q        Ready
      /DC=cz/DC=cesnet-ca/O=Masaryk University/CN=Miroslav Ruda
  https://skurut68-2.cesnet.cz:9100/D1qbFGwvXLnd927JOcja1Q        Scheduled
      /DC=cz/DC=cesnet-ca/O=Masaryk University/CN=Miroslav Ruda
  https://skurut68-2.cesnet.cz:9100/D1qbFGwvXLnd927JOcja1Q        Running
      /DC=cz/DC=cesnet-ca/O=Masaryk University/CN=Miroslav Ruda
\end{verbatim}

Destroy notification:
\begin{verbatim}
  GLITE_WMS_NOTIF_SERVER=skurut68-2.cesnet.cz:9100 glite-lb-notify drop \
    https://skurut68-2.cesnet.cz:9100/NOTIF:tOsgB19Wz-M884anZufyUw
\end{verbatim}


\subsubsection{Example: Waiting for notifications on all user's jobs}

Instead of waiting for one job, user may want to accept notification about 
state changes of all his jobs.

Register notification for actual user:
\begin{verbatim}
  GLITE_WMS_NOTIF_SERVER=skurut68-2.cesnet.cz:9100 glite-lb-notify new -O
\end{verbatim}

returns:

\begin{verbatim}
  notification ID: https://skurut68-2.cesnet.cz:9100/NOTIF:tOsgB19Wz-M884anZufyUw 
\end{verbatim}

And continue with \verb'glite-lb-notify receive' similarly to previous example.
But in this case, we want to display also other information about job --
not job owner, but destination (where job is running) and location (which component is 
serving job):

\begin{verbatim}
  GLITE_WMS_NOTIF_SERVER=skurut68-2.cesnet.cz:9100 glite-lb-notify receive \
   -i 240 -f destination,location \
   https://skurut68-2.cesnet.cz:9100/NOTIF:tOsgB19Wz-M884anZufyUw
\end{verbatim}

returns:

\begin{verbatim}
  notification is valid until: '2008-07-29 15:43:41' (1217346221)

 https://skurut68-2.cesnet.cz:9100/qbRFxDFCg2qO4-9WYBiiig        Waiting
   (null) NetworkServer/erebor.ics.muni.cz/
 https://skurut68-2.cesnet.cz:9100/qbRFxDFCg2qO4-9WYBiiig        Waiting
   (null)  destination queue/erebor.ics.muni.cz/
 https://skurut68-2.cesnet.cz:9100/qbRFxDFCg2qO4-9WYBiiig        Waiting
   (null) WorkloadManager/erebor.ics.muni.cz/
 https://skurut68-2.cesnet.cz:9100/qbRFxDFCg2qO4-9WYBiiig        Waiting
   (null)  name of the called component/erebor.ics.muni.cz/
 https://skurut68-2.cesnet.cz:9100/qbRFxDFCg2qO4-9WYBiiig        Waiting
   destination CE/queue WorkloadManager/erebor.ics.muni.cz/
 https://skurut68-2.cesnet.cz:9100/qbRFxDFCg2qO4-9WYBiiig        Waiting
   destination CE/queue WorkloadManager/erebor.ics.muni.cz/
 https://skurut68-2.cesnet.cz:9100/qbRFxDFCg2qO4-9WYBiiig        Ready
   destination CE/queue destination queue/erebor.ics.muni.cz/
 https://skurut68-2.cesnet.cz:9100/qbRFxDFCg2qO4-9WYBiiig        Ready
   destination CE/queue JobController/erebor.ics.muni.cz/
 https://skurut68-2.cesnet.cz:9100/qbRFxDFCg2qO4-9WYBiiig        Ready
   destination CE/queue LRMS/destination hostname/destination instance
 https://skurut68-2.cesnet.cz:9100/qbRFxDFCg2qO4-9WYBiiig        Ready
   destination CE/queue LogMonitor/erebor.ics.muni.cz/
 https://skurut68-2.cesnet.cz:9100/qbRFxDFCg2qO4-9WYBiiig        Scheduled
   destination CE/queue LRMS/destination hostname/destination instance
 https://skurut68-2.cesnet.cz:9100/qbRFxDFCg2qO4-9WYBiiig        Running
   destination CE/queue LRMS/worknode/worker node

\end{verbatim}

\subsubsection{Example: Registering for notifications to be delivered over ActiveMQ}
\label{e:notifymsg}

Delivering notification messages over the messaging infrastructure provided by ActiveMQ is a feature introduced in \LBver{3.0}. It uses the fake address capability (\texttt{-a} option) to specify messaging topic to use when generating messages.

\begin{verbatim}
  GLITE_WMS_NOTIF_SERVER=skurut68-2.cesnet.cz:9100 glite-lb-notify new \
   -O -a x-msg://grid.emi.lbexample
\end{verbatim}

Rather than using the \LB notification API to receive messages, access the messaging infrastructure and tap into the given messaging topic (\texttt{grid.emi.lbexample} in our case).

Note that production environments can impose restrictions on topic names. In the context of EGI, for instance, prefix ``\texttt{grid.emi.}'' is mandatory. A full list of permissible topic may be obtained from the \LB Server's configuration page (Section \ref{s:findbroker}).

Also note that in case you are unsure what messaging brokers are available in your grid environment, you can read that information in the \LB Server's configuration page (Section \ref{s:findbroker}).

\subsubsection{Example: Waiting for more notifications on one socket}

The foloving example demonstrates possibility to reuse existing socket for receiving
multiple notifications. Perl script \verb'notify.pl' (available in 
examples directory) creates socket, which is then reused for all
\verb'glite-lb-notify' commands.

\begin{verbatim}
GLITE_WMS_NOTIF_SERVER=skurut68-2.cesnet.cz:9100 NOTIFY_CMD=glite-lb-notify \
 ./notify.pl -O
\end{verbatim}

returns:

\begin{verbatim}
notification ID: https://skurut68-2.cesnet.cz:9100/NOTIF:EO73rjsmexEZJXuSoSZVDg
valid: '2008-07-29 15:14:06' (1217344446)
got connection
https://skurut68-2.cesnet.cz:9100/ANceuj5fXdtaCCkfnhBIXQ        Submitted
/DC=cz/DC=cesnet-ca/O=Masaryk University/CN=Miroslav Ruda
glite-lb-notify: Connection timed out (read message)
got connection
https://skurut68-2.cesnet.cz:9100/p2jBsy5WkFItY284lW2o8A        Submitted
/DC=cz/DC=cesnet-ca/O=Masaryk University/CN=Miroslav Ruda
glite-lb-notify: Connection timed out (read message)
got connection
https://skurut68-2.cesnet.cz:9100/p2jBsy5WkFItY284lW2o8A        Waiting
/DC=cz/DC=cesnet-ca/O=Masaryk University/CN=Miroslav Ruda
\end{verbatim}


\subsubsection{Example: Waiting for notifications on jobs reaching terminal states}

This example shows how to set up notifications for jobs reaching state \emph{done} or \emph {aborted}. Since we prefer to receive just one notification per job, we will also use the \texttt{-c} option, which makes sure that notifications will be generated only on actual job state change.


\begin{verbatim}
  GLITE_WMS_NOTIF_SERVER=skurut68-2.cesnet.cz:9100 glite-lb-notify new \
   --state done,aborted -c
\end{verbatim}

returns:

\begin{verbatim}
  notification ID: https://skurut68-2.cesnet.cz:9100/NOTIF:6krjMRshTouH2n7D9t-xdg 
  valid: '2009-04-30 06:59:18 UTC' (1241074758)	
\end{verbatim}

Wait for notification (with timeout 120 seconds):
\begin{verbatim}
  GLITE_WMS_NOTIF_SERVER=skurut68-2.cesnet.cz:9100 glite-lb-notify receive \
   -i 120 https://skurut68-2.cesnet.cz:9100/NOTIF:6krjMRshTouH2n7D9t-xdg 
\end{verbatim}

returns:
\begin{verbatim}
https://skurut68-2.cesnet.cz:9100/eIbQNz3oHpv-OkYVu-cXNg	Done
    /DC=cz/DC=cesnet-ca/O=Masaryk University/CN=Miroslav Ruda
https://skurut68-2.cesnet.cz:9100/GpBy2gfIZOAXR2hKOAYGgg	Aborted
    /DC=cz/DC=cesnet-ca/O=Masaryk University/CN=Miroslav Ruda
https://skurut68-2.cesnet.cz:9100/KWXmsqvsTQKizQ4OMiXItA	Done
    /DC=cz/DC=cesnet-ca/O=Masaryk University/CN=Miroslav Ruda
https://skurut68-2.cesnet.cz:9100/O1zs50Nm1r03vx2GLyaxQw	Done
    /DC=cz/DC=cesnet-ca/O=Masaryk University/CN=Miroslav Ruda
\end{verbatim}




\subsection{HTML and plain text interface}

It is possible to use a standard Web browser or a command-line tool such as \texttt{wget} or \texttt{curl} to extract information from the \LB server. Although the querying power is higly limited, the HTTP or Plain Text interface can still serve as a simple tool to access information.

\subsubsection{Job ID or Notification ID as URL}
\label{simple}
Since the gLite jobId has the form of a unique URL, it is possible to cut and paste it directly
to the web browser to view the information about the job (esp. its status), e.g.
\begin{verbatim}
  firefox https://pelargir.ics.muni.cz:9000/1234567890
\end{verbatim}
To list all user's jobs, it is possible to query only the \LB server address, e.g.
\begin{verbatim}
  firefox https://pelargir.ics.muni.cz:9000
\end{verbatim}
To list all user's notification registrations curently valid on a given \LB server, use an URL constructed as in folowing example:
\begin{verbatim}
  firefox https://pelargir.ics.muni.cz:9000/NOTIF
\end{verbatim}

On top of that, \LB super users may use query strings \texttt{?all} or \texttt{?foreign} to display registrations belonging to anyone or anyone but themselves, respectively.

A notification ID also takes the form of an URL. If you direct your browser to a particular notification ID, the \LB server will provide related registration details.
\begin{verbatim}
  firefox https://pelargir.ics.muni.cz:9000/NOTIF:1234567890
\end{verbatim}
In all cases it is necessary to have the user certificate installed in the browser.

\subsubsection{Plain Text Modifier}
\label{HTML:plaintext}

Since \LBver{2.0}, it is also possible to obtain the above results in a machine readable 
\verb'key=value' form by adding a suffix \verb'text' to the above URLs. For example
\begin{verbatim}
  curl -3 --key $X509_USER_KEY --cert $X509_USER_CERT \
    --capath /etc/grid-security/certificates \ 
    https://pelargir.ics.muni.cz:9000?text
\end{verbatim}
or
\begin{verbatim}
  curl -3 --key $X509_USER_KEY --cert $X509_USER_CERT \
    --capath /etc/grid-security/certificates \ 
    https://pelargir.ics.muni.cz:9000/1234567890?text
\end{verbatim}

\subsubsection{Querying for Jobs by Attribute}
\label{HTML:queries}

A subset of the \LB querying API---specifically job queries---has been made available over the HTML interface as of \LBver{3.3}. It supports the full range\footnote{There a single exception: it does not implement any syntax for using the \emph{within} operator, which must therefore be expressed as a combination of \emph{greater than AND lower than} conditions} of job queries as discussed in \cite{lbdg} and is also subject to the same limitations, i. e., that multiple conditions may only be combined in the following structure:
$$(attr_{1}cond_{1} \vee \dots \vee attr_{1}cond_{n}) \wedge \dots \wedge (attr_{n}cond_{1} \vee \dots \vee attr_{n}cond_{n})$$
In other words, different attributes given in the query are always \emph{AND}ed, meaning that they must all be satisfied at once, but different \emph{OR} conditions can be given for the same attribute within the same query.

Conditions are conveyed to the \LB server through a query string starting with \texttt{?query}, which becomes a part of the URL, allowing frequently used queries to be stored as bookmarks in any Web browser. For instance, to check for all your currenlty running jobs, an URL constructed along the following pattern could be used:

\begin{verbatim}
https://pelargir.ics.muni.cz:9000/?query=status=running
\end{verbatim}

It is important to note that in its default configuration, \LB server requires at least one indexed attribute to be included in the query.\footnote{The list of indexed attributes can be received from the given \LB server's configuration page (see Section~\ref{s:findbroker}).} In most installations, attributes such as \emph{owner} or \emph{last update time} will be indexed. Toquery for all running jobs in such conditions, add an indexed attribute to your query. For instance:

\begin{verbatim}
https://pelargir.ics.muni.cz:9000/?query=status=running&lastupdatetime>1577836800
\end{verbatim}

Note that temporal attributes such as \emph{last update time} or \emph{state enter time} accept UNIX time values.

Since \emph{OR}ed conditions may only apply to a single attribute, the name of the attribute is given only once in the condition. For instance, to query for jobs in state \emph{running} or \emph{scheduled}:

\begin{verbatim}
https://pelargir.ics.muni.cz:9000/?query=status=scheduled|=running
\end{verbatim}

{\raggedright{}Currently the query attributes supported are: \texttt{jobid}, \texttt{owner}, \texttt{status}, \texttt{location}, \texttt{destination}, \texttt{donecode}, \texttt{usertag}, \texttt{time}, \texttt{level}, \texttt{host}, \texttt{source}, \texttt{instance}, \texttt{type}, \texttt{chkpt\_tag}, \texttt{resubmitted}, \texttt{parent\_job}, \texttt{exitcode}, \texttt{jdl}, \texttt{stateentertime}, \texttt{lastupdatetime}, \texttt{networkserver} and \texttt{jobtype}.}

Supported operators are: \texttt{=}, \texttt{<>}, \texttt{>} or \texttt{<}.

Supported job state values (attribute \texttt{status}) are listed in Appendix~\ref{a:jobstat}.

{\raggedright{}Supported job type values are: \jobtypenames.}

All queries result in a list of JobIDs of jobs that match the specified criteria.

This feature is supported since \LBver{3.3}

\subsubsection{Applying Flags}
\label{HTML:flags}

Query conditions explained in Section~\ref{HTML:queries} but also simple requests may be combined with flags supplied through the \texttt{?flags} query string. For instance, when querying for a colection status, the following use of flags will initiate thorough recalculation of child job state histogram and make sure that classad data (namely JDL) are retrieved.

\begin{verbatim}
https://skurut68-2.cesnet.cz:9100/D1qbFGwvXLnd927JOcja1Q?flags=classadd+childhist_thorough
\end{verbatim}

The following flags are supported in job and job status queries:
\texttt{classadd},
\texttt{children},
\texttt{childstat},
\texttt{no\_jobs},
\texttt{no\_states},
\texttt{childhist\_fast},
\texttt{childhist\_thorough},
\texttt{anonymized}, and
\texttt{history}.

This feature is supported since \LBver{3.3}

\subsubsection{Reading \LB Server Configuration over HTTPs}

As of \LBver{3.0}, it is also possible to use the HTTPs interface to retrieve essential information on \LB Server configuration. For example:
\begin{verbatim}
  firefox https://pelargir.ics.muni.cz:9000/?configuration
\end{verbatim}

Among others, the following fields may be discerned from the URL:

\begin{tabularx}{\textwidth}{lX}
\label{s:findbroker}
\texttt{msg\_brokers} & A list of messaging brokers that \LB server uses to deliver messages.\\
\texttt{msg\_prefixes} & A list of permissible prefixes that must be used in messaging topics.\\
\end{tabularx}



\subsubsection{Summary of Applicable Query Strings}
\label{HTML:querystrings}

\begin{tabularx}{\textwidth}{>{\tt}lX}
\texttt{?agu} & Suppresses normal output and returns up-to-date WSDL for AGU methods \\
\texttt{?all} & Applies only to notifications. Makes sure notification registrations by all users are displayed if privileges suffice. Explained in~\ref{simple}. \\
\texttt{?configuration} & Suppresses normal output and displays various configuration properties of the server. Explained in~\ref{s:findbroker}. \\
\texttt{?flags} & Specifies query flags, explained in~\ref{HTML:flags}. \\
\texttt{?foreign} & Applies only to notifications. Makes sure notification registrations by all users but yourselves are displayed if privileges suffice. Explained in~\ref{simple}. \\
\texttt{?query} & Specifies query conditions when querying for jobs. Explained in~\ref{HTML:queries}. \\
\texttt{?stats} & Suppresses normal output and displays server usage statistics instead. \\
\texttt{?text} & Instead of HTML, returns output in plain text \texttt{key=value} format. Explained in~\ref{HTML:plaintext}.\\
\texttt{?types} & Suppresses normal output and returns up-to-date WSDL for WS type definitions. \\
\texttt{?version} & Suppresses normal output and displays server version instead. \\
\texttt{?wsdl} & Suppresses normal output and returns the up-to-date WSDL used by \LB's WS interface. \\
\end{tabularx}

\subsection{Job state changes as an RSS feed}
The \LB includes an RSS interface allowing users to keep trace of their jobs in a very simple way using an RSS reader. The parameters of the RSS feeds are predefined, so no configuration is required.

The address of a feed is given by the URL of the \LB server and a \textit{/RSS:rss\_feed\_name} postfix, e.g.
\begin{verbatim}
   https://pelargir.ics.muni.cz:9000/RSS:finished
\end{verbatim}  

There are currently 3 feeds implemented in LB:
\begin{itemize}
 \item \textit{finished} for jobs in terminal states (Done/OK, Aborted and Canceled)
 \item \textit{running} for running jobs
 \item \textit{aborted} for aborted jobs
\end{itemize}

\subsection{Other useful tools}

For debugging purposes, low-level commands for getting \LB job status and job related events are provided in 
\verb'examples' directory (\verb'glite-lb-job_status' and \verb'glite-lb-job_log'). The same directory
contains also debugging commands for getting of all user jobs (\verb'glite-lb-user_jobs') and
CE-reputability rank (see Section \ref{s:ce-rank}, \verb'glite-lb-stats').

