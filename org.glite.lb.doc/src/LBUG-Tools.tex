%
%% Copyright (c) Members of the EGEE Collaboration. 2004-2010.
%% See http://www.eu-egee.org/partners for details on the copyright holders.
%% 
%% Licensed under the Apache License, Version 2.0 (the "License");
%% you may not use this file except in compliance with the License.
%% You may obtain a copy of the License at
%% 
%%     http://www.apache.org/licenses/LICENSE-2.0
%% 
%% Unless required by applicable law or agreed to in writing, software
%% distributed under the License is distributed on an "AS IS" BASIS,
%% WITHOUT WARRANTIES OR CONDITIONS OF ANY KIND, either express or implied.
%% See the License for the specific language governing permissions and
%% limitations under the License.
%
\section{User tools}
\label{s:lb-tools}

In this section we give a description of the CLI tools that a regular grid user
might want to use. If not stated otherwise, the tools are distributed in the
\verb'glite-lb-client' package.

\subsection{Environment variables}

Behaviour of the commands can be changed by setting enviroment variables, specifing mostly
location of servers or setting timeouts for various operations. 
For a complete list of environment variables, their form and default value
description, see Appendix~\ref{a:environment}. Setting the environment variable
is for some commands mandatory, so reading the documentaion below and
appropriate manpages is highly recommended.


\subsection{glite-wms-job-status and glite-wms-job-logging-info}

We start with tools that are distributed in \verb'glite-wms-ui-cli-python' package
and that can be found probably on all UI machines. Description of all user
commands that are used during the job submission process (generating proxy,
creating a JDL describing the job, submitting a job, retrieving output,
cancelling a job, etc.) is beoynd this document and it can be found for example
in \cite{wmsug}. We mention here only the commands that are related to
job monitoring.

Once job has been submitted to WMS (by \verb'glite-wms-job-submit'), 
a user can retrieve the job status by
\begin{verbatim}
    glite-wms-job-status <jobId>
\end{verbatim}
or if job ID is saved in the file
\begin{verbatim}
    glite-wms-job-status -i <job_id_file>
\end{verbatim}
or if user wants to see status of all his/her jobs
\begin{verbatim}
    glite-wms-job-status --all
\end{verbatim}
List of all possible job states is summarised in Appendix \ref{a:jobstat}.

Logging details on submitted job can be accessed via
\begin{verbatim}
    glite-wms-job-logging-info -v <verbosity_level> <job_ID>
\end{verbatim}
or if job ID is saved in the file
\begin{verbatim}
   glite-wms-job-logging-info -v <verbosity_level> -i <job_id_file>
\end{verbatim}
where verbosity level can be from 0 to 3. 


\subsection{glite-lb-logevent}
\label{glite-lb-logevent}

Besides the API's \LB\ offers its users a simple command-line interface for
logging events. The command \verb'glite-lb-logevent' is used for this purpose.
However, it is intended for internal WMS debugging tests in the first place and
should not be used for common event logging because of possibility of confusing
\LB\ server job state automaton.

The command \verb'glite-lb-logevent' is a complex logging tool and the complete
list of parameters can be obtained using the \verb'-h' option.  However,
the only legal user usage is for logging \verb'UserTag' and \verb'ChangeACL'
events. The following description is therefore concentrating only on options
dealing with these two events.

Command usage is:

\begin{verbatim}
    glite-lb-logevent [-h] [-p] [-c seq_code] 
        -j <dg_jobid> -s Application -e <event_name> [key=value ...]
\end{verbatim}

where

\begin{tabularx}{\textwidth}{lX}
\texttt{  -p  -{}-priority} &       send a priority event\\
\texttt{  -c  -{}-sequence} &       event sequence code\\
\texttt{  -j  -{}-jobid} &          JobId\\
\texttt{  -e  -{}-event} &           select event type (see -e help)\\
\end{tabularx}

%\medskip

Each event specified after \verb'-e' option has different sub-options enabling
to set event specific values.

Address of local-logger, daemon responsible for further message delivery, must 
be specified by environment
variable \verb'GLITE_WMS_LOG_DESTINATION' in a form \verb'address:port'.

Because user is allowed to change ACL or add user tags only for her jobs, paths
to valid X509 user credentials has to be set to authorise her. This is done
using standard X509 environment variables \verb'X509_USER_KEY' and
\verb'X509_USER_CERT'.

For additional information see also manual page glite-lb-logevent(1).

\input log_usertag.tex

\input change_acl.tex


\subsection{glite-lb-notify}

\TODO{ljocha}



\subsection{HTML and plain text interface}

It is possible to use a standard Web browser or a command-line tool such as \texttt{wget} or \texttt{curl} to extract information from the \LB server. Although the querying power is higly limited, the HTTP or Plain Text interface can still serve as a simple tool to access information.

\subsubsection{Job ID or Notification ID as URL}
\label{simple}
Since the gLite jobId has the form of a unique URL, it is possible to cut and paste it directly
to the web browser to view the information about the job (esp. its status), e.g.
\begin{verbatim}
  firefox https://pelargir.ics.muni.cz:9000/1234567890
\end{verbatim}
To list all user's jobs, it is possible to query only the \LB server address, e.g.
\begin{verbatim}
  firefox https://pelargir.ics.muni.cz:9000
\end{verbatim}
To list all user's notification registrations curently valid on a given \LB server, use an URL constructed as in folowing example:
\begin{verbatim}
  firefox https://pelargir.ics.muni.cz:9000/NOTIF
\end{verbatim}

On top of that, \LB super users may use query strings \texttt{?all} or \texttt{?foreign} to display registrations belonging to anyone or anyone but themselves, respectively.

A notification ID also takes the form of an URL. If you direct your browser to a particular notification ID, the \LB server will provide related registration details.
\begin{verbatim}
  firefox https://pelargir.ics.muni.cz:9000/NOTIF:1234567890
\end{verbatim}
In all cases it is necessary to have the user certificate installed in the browser.

\subsubsection{Plain Text Modifier}
\label{HTML:plaintext}

Since \LBver{2.0}, it is also possible to obtain the above results in a machine readable 
\verb'key=value' form by adding a suffix \verb'text' to the above URLs. For example
\begin{verbatim}
  curl -3 --key $X509_USER_KEY --cert $X509_USER_CERT \
    --capath /etc/grid-security/certificates \ 
    https://pelargir.ics.muni.cz:9000?text
\end{verbatim}
or
\begin{verbatim}
  curl -3 --key $X509_USER_KEY --cert $X509_USER_CERT \
    --capath /etc/grid-security/certificates \ 
    https://pelargir.ics.muni.cz:9000/1234567890?text
\end{verbatim}

\subsubsection{Querying for Jobs by Attribute}
\label{HTML:queries}

A subset of the \LB querying API---specifically job queries---has been made available over the HTML interface as of \LBver{3.3}. It supports the full range\footnote{There a single exception: it does not implement any syntax for using the \emph{within} operator, which must therefore be expressed as a combination of \emph{greater than AND lower than} conditions} of job queries as discussed in \cite{lbdg} and is also subject to the same limitations, i. e., that multiple conditions may only be combined in the following structure:
$$(attr_{1}cond_{1} \vee \dots \vee attr_{1}cond_{n}) \wedge \dots \wedge (attr_{n}cond_{1} \vee \dots \vee attr_{n}cond_{n})$$
In other words, different attributes given in the query are always \emph{AND}ed, meaning that they must all be satisfied at once, but different \emph{OR} conditions can be given for the same attribute within the same query.

Conditions are conveyed to the \LB server through a query string starting with \texttt{?query}, which becomes a part of the URL, allowing frequently used queries to be stored as bookmarks in any Web browser. For instance, to check for all your currenlty running jobs, an URL constructed along the following pattern could be used:

\begin{verbatim}
https://pelargir.ics.muni.cz:9000/?query=status=running
\end{verbatim}

It is important to note that in its default configuration, \LB server requires at least one indexed attribute to be included in the query.\footnote{The list of indexed attributes can be received from the given \LB server's configuration page (see Section~\ref{s:findbroker}).} In most installations, attributes such as \emph{owner} or \emph{last update time} will be indexed. Toquery for all running jobs in such conditions, add an indexed attribute to your query. For instance:

\begin{verbatim}
https://pelargir.ics.muni.cz:9000/?query=status=running&lastupdatetime>1577836800
\end{verbatim}

Note that temporal attributes such as \emph{last update time} or \emph{state enter time} accept UNIX time values.

Since \emph{OR}ed conditions may only apply to a single attribute, the name of the attribute is given only once in the condition. For instance, to query for jobs in state \emph{running} or \emph{scheduled}:

\begin{verbatim}
https://pelargir.ics.muni.cz:9000/?query=status=scheduled|=running
\end{verbatim}

{\raggedright{}Currently the query attributes supported are: \texttt{jobid}, \texttt{owner}, \texttt{status}, \texttt{location}, \texttt{destination}, \texttt{donecode}, \texttt{usertag}, \texttt{time}, \texttt{level}, \texttt{host}, \texttt{source}, \texttt{instance}, \texttt{type}, \texttt{chkpt\_tag}, \texttt{resubmitted}, \texttt{parent\_job}, \texttt{exitcode}, \texttt{jdl}, \texttt{stateentertime}, \texttt{lastupdatetime}, \texttt{networkserver} and \texttt{jobtype}.}

Supported operators are: \texttt{=}, \texttt{<>}, \texttt{>} or \texttt{<}.

Supported job state values (attribute \texttt{status}) are listed in Appendix~\ref{a:jobstat}.

{\raggedright{}Supported job type values are: \jobtypenames.}

All queries result in a list of JobIDs of jobs that match the specified criteria.

This feature is supported since \LBver{3.3}

\subsubsection{Applying Flags}
\label{HTML:flags}

Query conditions explained in Section~\ref{HTML:queries} but also simple requests may be combined with flags supplied through the \texttt{?flags} query string. For instance, when querying for a colection status, the following use of flags will initiate thorough recalculation of child job state histogram and make sure that classad data (namely JDL) are retrieved.

\begin{verbatim}
https://skurut68-2.cesnet.cz:9100/D1qbFGwvXLnd927JOcja1Q?flags=classadd+childhist_thorough
\end{verbatim}

The following flags are supported in job and job status queries:
\texttt{classadd},
\texttt{children},
\texttt{childstat},
\texttt{no\_jobs},
\texttt{no\_states},
\texttt{childhist\_fast},
\texttt{childhist\_thorough},
\texttt{anonymized}, and
\texttt{history}.

This feature is supported since \LBver{3.3}

\subsubsection{Reading \LB Server Configuration over HTTPs}

As of \LBver{3.0}, it is also possible to use the HTTPs interface to retrieve essential information on \LB Server configuration. For example:
\begin{verbatim}
  firefox https://pelargir.ics.muni.cz:9000/?configuration
\end{verbatim}

Among others, the following fields may be discerned from the URL:

\begin{tabularx}{\textwidth}{lX}
\label{s:findbroker}
\texttt{msg\_brokers} & A list of messaging brokers that \LB server uses to deliver messages.\\
\texttt{msg\_prefixes} & A list of permissible prefixes that must be used in messaging topics.\\
\texttt{server\_version} & Version of the \LB server binary.\\
\texttt{server\_identity} & The subject of the \LB server's certificate.\\
\texttt{server\_indices} & Indices configured in the the \LB server's database. \ifx\insideAG\undefined{\LB job queries typically require at least one indexed attribute among the query conditions}\else{See more on managing indices in Section~\ref{maintain:index} (page \pageref{maintain:index})}\fi.
\\
\end{tabularx}

Configuration details shown only to \LB super users:
\nopagebreak

\begin{tabularx}{\textwidth}{lX}
\texttt{database\_name} & Name of database storing \LB data.\\
\texttt{database\_host} &  Database server used by \LB.\\
\texttt{authz\_policy\_file} & Full copy of the current authz configuration file.\ifx\insideAG\undefined{}\else{ See more on defining the authorization policy in Section~\ref{inst:authz} (page \pageref{inst:authz}).}\fi\\
\texttt{admins} & A list of DNs that are allowed super-user access. The list will include superusers given in the authz file as well as those specified from command line through the \texttt{-{}-super-user} option.\\
\texttt{dump\_start} & Starting time of the latest server dump.\ifx\insideAG\undefined{}\else{See more on backup dumps in Section~\ref{run:dump} (page \pageref{run:dump}).}\fi\\
\texttt{dump\_end} & Finishing time of the latest server dump.\\
\end{tabularx}



\subsubsection{Summary of Applicable Query Strings}
\label{HTML:querystrings}

\begin{tabularx}{\textwidth}{>{\tt}lX}
\texttt{?agu} & Suppresses normal output and returns up-to-date WSDL for AGU methods \\
\texttt{?all} & Applies only to notifications. Makes sure notification registrations by all users are displayed if privileges suffice. Explained in~\ref{simple}. \\
\texttt{?configuration} & Suppresses normal output and displays various configuration properties of the server. Explained in~\ref{s:findbroker}. \\
\texttt{?flags} & Specifies query flags, explained in~\ref{HTML:flags}. \\
\texttt{?foreign} & Applies only to notifications. Makes sure notification registrations by all users but yourselves are displayed if privileges suffice. Explained in~\ref{simple}. \\
\texttt{?query} & Specifies query conditions when querying for jobs. Explained in~\ref{HTML:queries}. \\
\texttt{?stats} & Suppresses normal output and displays server usage statistics instead. \\
\texttt{?text} & Instead of HTML, returns output in plain text \texttt{key=value} format. Explained in~\ref{HTML:plaintext}.\\
\texttt{?types} & Suppresses normal output and returns up-to-date WSDL for WS type definitions. \\
\texttt{?version} & Suppresses normal output and displays server version instead. \\
\texttt{?wsdl} & Suppresses normal output and returns the up-to-date WSDL used by \LB's WS interface. \\
\end{tabularx}

\subsection{Job state changes as an RSS feed}
The \LB includes an RSS interface allowing users to keep trace of their jobs in a very simple way using an RSS reader. The parameters of the RSS feeds are predefined, so no configuration is required.

The address of a feed is given by the URL of the \LB server and a \textit{/RSS:rss\_feed\_name} postfix, e.g.
\begin{verbatim}
   https://pelargir.ics.muni.cz:9000/RSS:finished
\end{verbatim}  

There are currently 3 feeds implemented in LB:
\begin{itemize}
 \item \textit{finished} for jobs in terminal states (Done/OK, Aborted and Canceled)
 \item \textit{running} for running jobs
 \item \textit{aborted} for aborted jobs
\end{itemize}

\subsection{Other useful tools}

For debugging purposes, low-level commands for getting \LB job status and job related events are provided in 
\verb'examples' directory (\verb'glite-lb-job_status' and \verb'glite-lb-job_log'). The same directory
contains also debugging commands for getting of all user jobs (\verb'glite-lb-user_jobs') and
CE-reputability rank (see Section \ref{s:ce-rank}, \verb'glite-lb-stats').

