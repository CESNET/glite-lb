The \LB\ infrastructure ensures high level of security for information it
processes. All the \LB\ components communicate solely over authenticated
connections and users who query the \LB\ server also must authenticate properly
using their PKI certificates. All messages sent over the network are encrypted
and their content is not accessible to unauthorized people.

By default, information about a job stored in the \LB\ server is only available
to the user who submitted the job, i.e. the job owner.  Besides this default
functionality, the \LB\ server also allows the job owner to share job
information with another users. Each job can be assigned an access control list
(ACL) that specifies another users who are also allowed to access the job
information. The management of ACL's is entirely under control of the job owner
so she can modify the ACL arbitrarily, specifying the set of users who have
access to the job information. The users in the ACL's can be specified using
either the subject names from their X.509 certificates or names of VOMS groups.

Current ACL for a job is returned as part of the job status information
returned by the \verb'glite-job-status' command. The commands output ACL's in
the original XML format as specified by GACL/GridSite. 

Example of an ACL:
\begin{verbatim}
<?xml version="1.0"?><gacl version="0.0.1">
   <entry>
      <voms-cred><vo>VOCE</vo><group>/VOCE</group></voms-cred>
      <allow><read/></allow>
   </entry>
   <entry>
      <person><dn>/O=CESNET/O=Masaryk University/CN=Daniel Kouril</dn></person>
      <deny><read/></deny>
   </entry>
</gacl>
\end{verbatim}

This ACL allows access to all people in the VOMS group /VOCE in virtual
organization VOCE, but denies access to user Daniel Kouril (even if he was a
member of the /VOCE group).

The job owner herself is not specified in the ACL as she is always allowed to
access the information regardless the content of the job ACL.

An ACL for a job can be changed using the \verb'glite-lb-logevent' command-line
program, see section~\ref{change_acl}.

%provided in the example subdirectory. In order to use change\_acl, the \LB\
%daemons locallogger and interlogger must be running. The usage of the command
%is as follows:
%
%\LB\ server configuration
%In order to support the VOMS groups in the ACL's,
%glite_lb_bkserverd must be able to verify client's VOMS proxy certificate using
%a trusted VOMS service certificate stored on a local disk. Default directory
%with trusted VOMS certificates is /etc/grid-security/vomsdir, another location
%can be specified using by either the -V option to glite_lb_bkserverd or setting
%the VOMS_CERT_DIR environment variable.

