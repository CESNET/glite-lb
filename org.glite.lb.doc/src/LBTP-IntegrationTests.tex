% Integration tests

% Installation and configuration tests are done manually by the certification team,
% there is no need to specify them here at the moment, they can be added later

% \section{Installation tests}
% \label{s:installation}
% \TODO{salvet: list installed packages, check all dependencies, ...}
% \begin{center}
% \url{https://twiki.cern.ch/twiki/bin/view/EGEE/Glite-LB}
% \end{center}


% \newpage
% \section{Configuration tests}
% \label{s:configuration}
% \TODO{salvet: check all config files, ...}
% \begin{center}
% \url{https://twiki.cern.ch/twiki/bin/view/EGEE/Glite-LB}
% \end{center}


% \newpage
\section{Service ping tests}
\label{s:ping}

In this section we describe basic tests of \LB if the services are up and running.
% Local tests are meant to be run on the install node (where the appropriate daemon is running)
% whereas remote tests are meant to be run via another node.

\subsection{Test Suite Overview}

This subsection gives a comprehensive overview of all service ping tests.

\begin{tabularx}{\textwidth}{|l|l|X|}
\hline
     {\bf Executable} & {\bf Status} & {\bf Use} \\
\hline
{\tt lb-test-logger-local.sh} & Implemented & Test job logging facilities on a local machine (processes running, ports listening, etc.). \\
\hline
{\tt lb-test-logger-remote.sh} & Implemented & Test the local logger availability remotely (open ports). \\
\hline
{\tt lb-test-server-local.sh} & Implemented & Test for LB server running on a local machine (processes running, ports listening, etc.). \\
\hline
{\tt lb-test-server-remote.sh} & Implemented & Test the LB server availability remotely (open ports). \\
\hline
\end{tabularx}

\subsection{Logger (local \& inter)}

\subsubsection{Local tests}
\what\ check if both \texttt{glite-lb-logd} and \texttt{glite-lb-interlogd} are running,
check if \texttt{glite-lb-logd} is listening on which port (9002 by default),
socket-connect to \texttt{glite-lb-logd},
check if enough disk capacity is free for \code{dglog*} files,
socket-connect to \texttt{glite-lb-interlogd}.

\how\ \ctblb{lb-test-logger-local.sh}


\subsubsection{Remote tests}
\req\ environment variable \texttt{GLITE\_WMS\_LOG\_DESTINATION} set, GSI credentials set

\what\ network ping,
check GSI credentials,
socket-connect,
gsi-connect

\how\ \ctblb{lb-test-logger-remote.sh}


\subsection{Server}

\subsubsection{Local tests}
\req\ environment variables \texttt{GLITE\_WMS\_LBPROXY\_STORE\_SOCK},
and \texttt{GLITE\_WMS\_LBPROXY\_SERVE\_SOCK} set

\what\ check MySQL (running, accessible, enough disk capacity, ...),
check if both daemons \texttt{glite-lb-bkserverd} and \texttt{glite-lb-notif-interlogd} are running,
check if \texttt{glite-lb-bkserverd} is listening on which ports (9000, 9001 and 9003 by default),
socket-connect to all \texttt{glite-lb-bkserverd} ports and sockets,
check if enough disk capacity is free for dumps,
socket-connect to \texttt{glite-lb-notif-interlogd}.

\how\ \ctblb{lb-test-server-local.sh}


\subsubsection{Remote tests}
\req\ environment variable \texttt{GLITE\_WMS\_QUERY\_SERVER} set, GSI credentials set

\what\ network ping,
check GSI credentials,
socket-connect to all server ports,
gsi-connect to all server ports,
WS getVersion.

\how\ \ctblb{lb-test-server-remote.sh}



