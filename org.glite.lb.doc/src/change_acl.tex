%
%% Copyright (c) Members of the EGEE Collaboration. 2004-2010.
%% See http://www.eu-egee.org/partners for details on the copyright holders.
%% 
%% Licensed under the Apache License, Version 2.0 (the "License");
%% you may not use this file except in compliance with the License.
%% You may obtain a copy of the License at
%% 
%%     http://www.apache.org/licenses/LICENSE-2.0
%% 
%% Unless required by applicable law or agreed to in writing, software
%% distributed under the License is distributed on an "AS IS" BASIS,
%% WITHOUT WARRANTIES OR CONDITIONS OF ANY KIND, either express or implied.
%% See the License for the specific language governing permissions and
%% limitations under the License.
%
\subsubsection{Example: Changing Job Access Control List}
\label{e:change-acl}

In order to change the Access Control List (ACL) for a job, a special event
\verb'ChangeACL' is used. This event can be logged by the job owner using the
\verb'glite-lb-logevent' command (see also Sect.~\ref{glite-lb-logevent}).
The general template for changing the ACL is as follows:

\begin{verbatim}
glite-lb-logevent -e ChangeACL -s UserInterface -p -j <job_id>
        --user_id <user_id>                                             
        --user_id_type <user_id_type>                                   
        --permission READ
        --permission_type <permission_type> --operation <operation>
\end{verbatim}

where

\begin{tabularx}{\textwidth}{>{\texttt}lX}
\verb'<job_id>'    & specifies the job to change access to\\
\verb'<user_id>'   & specifies the user to grant or revoke permission. The
               parameter can be either an X.500 name
               (subject name), a VOMS group (of the form VO:Group), or a Full
               qualified attribute name (FQAN). \\
\verb'<user_id_type>' & indicates the type of the user\_id given above.
               \verb'DN', \verb'GROUP', and \verb'FQAN' can be given to
               specify X.500 name, VOMS group, or FQAN, respectively \\
\verb'<permission>' & ACL permission to change, currently only \verb'READ' is
               supported. \\
\verb'<permission_type>' & Type of permission requested. \verb'ALLOW' or
               \verb'DENY' can be specified. \\
\verb'<operation>' & Operation requested to be performed with ACL. \verb'ADD'
               or \verb'REMOVE' can be specified. \\
\end{tabularx}

Adding a user specified by his or her subject name to the ACL (\ie granting
access rights to another user):

\begin{verbatim}
glite-lb-logevent -e ChangeACL -s UserInterface -p -j <job_id>          \
        --user_id '/O=CESNET/O=Masaryk University/CN=Daniel Kouril'     \
        --user_id_type DN --permission READ --permission_type ALLOW     \
        --operation ADD
\end{verbatim}


Removing a user specified by his or her subject name from the ACL (\ie
revoking access right to another user):

\begin{verbatim}
glite-lb-logevent -e ChangeACL -s UserInterface -p -j <job_id>          \
        --user_id '/O=CESNET/O=Masaryk University/CN=Daniel Kouril'     \
        --user_id_type DN --permission READ --permission_type ALLOW     \
        --operation REMOVE
\end{verbatim}


Adding a VOMS attribute to the ACL:

\begin{verbatim}
glite-lb-logevent -e ChangeACL -s UserInterface -p -j <job_id>          \
        --user_id '/VOCE/Role=Administrator' --user_id_type FQAN        \
        --permission READ --permission_type ALLOW                       \
        --operation ADD
\end{verbatim}


Note that \LBold supported only using VOMS group names, not full FQANs,
whose support has been introduced only in \LBnew. \LBold also did not
allowed the users to use symbolic names for the values specifying ACL
setting and integers must be used instead. For example, to grant access
right on a \LBold server one has to use following syntax:

\begin{verbatim}
glite-lb-logevent -e ChangeACL -s UserInterface -p -j <job_id>          \
        --user_id '/O=CESNET/O=Masaryk University/CN=Daniel Kouril'     \
        --user_id_type 0 --permission 1 --permission_type 0 --operation 0
\end{verbatim}
