\subsubsection{Example: Logging a UserTag event}
\label{e:usertag}

User tag is an arbitrary ``name=value'' pair with which the user can assign
additional information to a job. Further on, LB can be queried based also on
values of user tags. \LB treats all values as strings only, semantic meaning
is left to user application. For internal reasons, all tag names are stored 
in lower-case format. Support for case-sensitivenes is planned in future 
versions of \LB.

In order to add user tag for a job a special event \verb'UserTag' is used. This
event can be logged by the job owner using the glite-lb-logevent command (see
also sec.\ref{glite-lb-logevent}). Here we suppose the command is used from
user's running application because a correct setting of environment variables
needed by the command is assured.

General template for adding user tag is as follows:

\begin{verbatim}
glite-lb-logevent -s Application -e UserTag    \
        -j <job_id>                         \
        -c <seq_code>                       \
        --name <tag_name>                   \
        --value <tag_value>
\end{verbatim}

where

\begin{tabularx}{\textwidth}{lX}
\texttt{<tag\_name>}  & specifies the name of user tag \\
\texttt{<tag\_value>} & specifies the value of user tag \\
\end{tabularx}

The user application is always executed from within a JobWrapper script (part
of Workload Management System \cite{jgc}). The wrapper  sets the  appropriate
\verb'JobId' in the environment variable \verb'GLITE_WMS_JOBID'. The user
should pass this value to the \verb'-j' option of \verb'glite-lb-logevent'.
Similarly, the wrapper sets an initial value of the event sequence code in the
environment variable \verb'GLITE_WMS_SEQUENCE_CODE'.

If the user application calls \verb'glite-lb-logevent' just once, it is
sufficient to pass this value to the \verb'-c' option.  However, if there are
more  subsequent calls,  the  user is responsible for capturing an updated
sequence code from the stdout of \verb'glite-lb-logevent' and using it in
subsequent calls. The \LB\ design requires the sequence codes in  order  to be
able to sort events correctly while not relying on strictly synchronized
clocks.

The example bellow is a job consisting of 100 phases. A user tag phase is used
to log the phase currently being executed. Subsequently, the user may monitor
execution of the job phases as a part of the job status returned by \LB.

\begin{verbatim}
  #!/bin/sh

  for p in `seq 1 100`; do

  # log the UserTag event
  GLITE_WMS_SEQUENCE_CODE=`glite-lb-logevent -s Application
    -e UserTag
    -j $GLITE_WMS_JOBID -c $GLITE_WMS_SEQUENCE_CODE
    --name=phase --value=$p`

  # do the actual computation here
  done
\end{verbatim}
