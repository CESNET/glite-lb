%
%% Copyright (c) Members of the EGEE Collaboration. 2004-2010.
%% See http://www.eu-egee.org/partners for details on the copyright holders.
%% 
%% Licensed under the Apache License, Version 2.0 (the "License");
%% you may not use this file except in compliance with the License.
%% You may obtain a copy of the License at
%% 
%%     http://www.apache.org/licenses/LICENSE-2.0
%% 
%% Unless required by applicable law or agreed to in writing, software
%% distributed under the License is distributed on an "AS IS" BASIS,
%% WITHOUT WARRANTIES OR CONDITIONS OF ANY KIND, either express or implied.
%% See the License for the specific language governing permissions and
%% limitations under the License.
%
% System tests

\section{System functionality tests}
\label{s:functionality}

\subsection{Test Suite Overview}

This subsection gives a comprehensive overview of all system functionality tests.

\subsubsection{Test Scripts}

Besides pure System Functionality Tests, this list also includes In-the-Wild tests and Regression Tests discussed in a few following chapters. They are used in the same manner and, typically, on the same occasions, which is why they are all listed in the same place. 

\begin{tabularx}{\textwidth}{|l|X|}
\hline
     {\bf Executable} & {\bf Use} \\
\hline
{\tt lb-test-job-registration.sh} & Tries to register a job and checks if the registration worked. \\
\hline
{\tt lb-test-event-delivery.sh} & Tries to register a job and log events. Checks if the registration worked and events resulted in state change accordingly. \\
\hline
{\tt lb-test-https.sh} & Test the HTTPs interface. \\
\hline
{\tt lb-test-logevent.sh} & Test if local logger accepts events correctly (i.e. returns 0). \\
\hline
{\tt lb-test-il-recovery.sh} & Tests if interlogger recovers correctly and processes events logged in between when restarted. \\
\hline
{\tt lb-test-job-states.sh} & Test that job state queries return correctly and that testing jobs are in expected states. \\
\hline
{\tt lb-test-collections.sh} & Perform various collection-specific tests. \\
\hline
{\tt lb-test-proxy-delivery.sh} & Test correct event delivery through \LB proxy. \\
\hline
{\tt lb-test-ws.sh} & Query events and job states through the Web-Service interface. \\
\hline
{\tt lb-test-notif.sh} & Test if notifications are delivered correctly for testing jobs. \\
\hline
%{\tt lb-test-notif-additional.sh} & Proposed & Test notification delivery with additional options. \\
%\hline
{\tt lb-test-notif-switch.sh} & Test the correct behavior of a notification once its target jobid changes.  \\
\hline
{\tt lb-test-notif-recovery.sh} & Test if notification client receives notifications correctly upon restart.  \\
\hline
{\tt lb-test-purge.pl} & Test that \LB server purge works correctly. \\
\hline
{\tt lb-test-wild.pl} & Test \LB ``in the wild'' (test with real-life WMS). \\
\hline
{\tt lb-test-bdii.sh} & Test \LB server is published correctly over BDII. \\
\hline
{\tt lb-test-sandbox-transfer.sh} & Test \LB's support for logging sandbox transfers. \\
\hline
{\tt lb-test-changeacl.sh} & Test proper parsing of \code{ChangeACL} events. \\
\hline
{\tt lb-test-statistics.sh} & Test statistic functions provided by \LB \\
\hline
{\tt lb-test-threaded.sh} & Rudimentary test for threaded clients \LB \\
\hline
{\tt lb-test-notif-msg.sh} & Test delivery of \LB notifications over ActiveMQ \\
\hline
{\tt lb-test-permissions.sh} & Check ownership and permission settings for config and runtime files \\
\hline
{\tt lb-test-nagios-probe.sh} & Run the nagios probe and check results \\
\hline
{\tt lb-test-notif-keeper.sh} & Test the \texttt{notif-keeper} script \\
\hline
{\tt lb-test-dump-load.sh} & Test the dump (backup) and load (restore) procedure \\
\hline
{\tt lb-test-collections.sh} & A placeholder for collection-specific regression tests \\
\hline
\end{tabularx}

\subsubsection{Event logging examples}

There is an {\tt examples} subdirectory in {\tt GLITE\_LOCATION}. It holds various example files---both binaries and scripts. There is---among others---a suite of scripts aimed at testing event delivery and the proper operation of the \LB state machine. Scripts named {\tt glite-lb-<state>.sh}---where {\tt <state>} corresponds with a job state---can be used to generate sequences of events that will always get an existing job into that state. (For example the {\tt glite-lb-running.sh} script logs a series of 12 events resulting in the job state turning to running.) Some of these scripts are used by system functionality tests detailed bellow but all of them can also be used for manual testing.

\subsection{Event delivery}

\subsubsection{Normal event delivery}
\label{normal}
% event delivery
% poslat .sh, job log vrati to, co bylo ve fajlech

\req\ all \LB\ daemons running (\path{glite-lb-logd}, \path{glite-lb-interlogd},
\path{glite-lb-bkserverd})

\what\ 
\begin{enumerate}
\item Register jobs with \code{edg\_wll\_RegisterJob} 
\item Log reasonable sequences of events with \code{edg\_wll\_Log*}, both through logger and/or proxy
\item Check with \code{edg\_wll\_JobLog} that the events got delivered afterwards (approx. 10s).
\item Also check delivery and processing of events related to collections.
\end{enumerate}

\how\ \ctblb{lb-test-event-delivery.sh}
% org.glite.testsuites.ctb/LB/lb-l2.sh now does the following:
% - array_job_reg: registeres $JOBS_ARRAY_SIZE jobs
% - logEvents: logs events by glite-lb-$state.sh example scripts
% - logTags: logs user tags
% - testLB: calls glite-lb-job_log for all jobs
% - testLB2: calls glite-lb-job_status for all jobs
%
% What needs to be done:
% - rename the script, tidy it
% - create some meaningful sequence of events for logEvents

\note\ The test includes artificial delays. Takes approx. 60\,s to finish.

\result\ All sub tests (API calls) should return 0. The same events that were logged must be returned.


\subsubsection{Job registration only}
\label{reg}
\req\ running \path{glite-lb-bkserverd}

\what\ call \code{edg\_wll\_RegisterJob}. Jobids should preferably point
to a~remote \LB\ server. Try re-registration using flag \code{EDG\_WLL\_LOGLFLAG\_EXCL} in various scenarios.

\how\ \ctblb{lb-test-job-registration.sh}

\result\ All sub tests (API calls) return 0.

%\begin{hints}
%\path{glite-lb-regjob} example can be used. It generates a~unique jobid,
%prints it and calls \LB\ API appropriately.
%\end{hints}



\subsubsection{Standalone local logger -- log event}
\label{log}
% async -- prida do fajlu, OK
% logevent

\req\ running \path{glite-lb-logd} only, jobs registered in test~\ref{reg}.

\what\ call \code{edg\_wll\_Log*} for various event types in a~sequence
resembling real \LB\ usage, using the same jobids as in test~\ref{reg}

\how\ \ctblb{lb-test-logevent.sh [event-file-prefix]}

\result\ All sub tests (API calls) return 0, events are added one per line to the local logger files.



\subsubsection{Interlogger recovery}
\label{recover}
% recover interloggeru
% il & server (remote)
% spustit, protlaci soubory na server, soubory zmizi, lze se dotazat na stav

\req\ running \path{glite-lb-bkserverd} on the machine and port where
jobids from \ref{reg} point to; files generated in~\ref{log};
\path{glite-lb-interlogd} is stopped.

\what\ Make a~copy of the files created in~\ref{log}, then start
\path{glite-lb-interlogd}. After approx. 10\,s check the jobs
with \code{edg\_wll\_JobLog} call.

\how\ \ctblb{lb-test-il-recovery.sh}

\note\ The test includes artificial delays. Takes approx. 15 to 75\,s to finish.

\result \code{edg\_wll\_JobLog} should return the same events that were
contained in the local logger files. The files should be removed by
interlogger after approx. 1 min.

%\begin{hints}
%\path{glite-lb-joblog} example outputs the events in (almost) the same
%format as the local logger files.
%\end{hints}




\subsection{Job state computation}

\subsubsection{Normal job states}
\label{state}
% normal event delivery & job state machine
% .sh, dotaz na stav

\req\ \path{glite-lb-bkserverd} running, events from \ref{normal} logged.

\what\ Check state of the jobs with \code{edg\_wll\_JobStatus}. Check all possible job states 
(if necessary, log relevant events). Query both server and/or proxy.

\how\ \ctblb{lb-test-job-states.sh}

\note\ The test includes artificial delays. Takes approx. 150\,s to finish.

\result\ The API call should return 0, the jobs should be in the expected
states. Thorough tests may also cross check the values supplied in the
events (e.g. destination computing element) wrt. the values reported in the job states.


\subsubsection{Sandbox Transfers}

\req\ All \LB\ services running

\what\
\begin{enumerate}
\item Register a compute job.
\item Register input sandbox transfer.
\item Register output sandbox transfer.
\item Generate events to trigger job state changes in one of the sandbox transfer jobs.
	\begin{enumerate}
        \item Start the transfer and check that state has changed appropriately.
        \item Finish the transfer and check that state has changed appropriately. 
	\end{enumerate}
\item Use another sandbox transfer job registered above to start, then fail the transfer and check that this is reflected by the resulting transfer job status. 
\item Check that the compute job and its sandbox transfer jobs link up correctly. 

\end{enumerate}

\how\ \ctblb{lb-test-sandbox-transfer.sh}

\note\ The test includes artificial delays. Takes approx. 50\,s to finish.

\result\ Job states should change on event delivery as expect, related jobs should ``know'' their IDs.


\subsubsection{Collection-specific Tests}

\req\ All \LB\ services running

\what\
There is no specific workflow for this test. It is a placeholder for various minor feature and regression tests related to job collections.

\how\ \ctblb{lb-test-collections.sh}

\result\ All included checks should finish OK.


%\subsubsection{Non-simple job states}
%\TODO{dags, collections, their states and states (and histogram) of their children/subjobs, ...}



%\subsection{Query tests}
%\TODO{query all my jobs, query one job, query with some structured conditions, some other queries that caused problems in the past, ...}


\subsection{\LB server and proxy combined test}

\req\ running \path{glite-lb-proxy}, \path{glite-lb-interlogd} and
\path{glite-lb-bkserverd}

\what\ Register jobs with \code{edg\_wll\_RegisterJobProxy}, log events
using \code{edg\_wll\_LogEventProxy} and check the job states against
both lbproxy (using \code{edg\_wll\_JobStatusProxy}) and bkserver
(using \code{edg\_wll\_JobStatus}). Also test job collections and registration, including registration of subjobs. Pay special attention to job reaching final 
job status and to the automatic purge from proxy.

%  - check the timeouts. - ??tam byly nejaky timeouty???

\how\ \ctblb{lb-test-proxy-delivery.sh}

%\TODO{mozna to ma prijit do nejakeho testsuitu, netusim} 
{\tt glite-lb-running.sh -x -m LB\_HOST:PORT} \\
- logs sequence of events and returns JOBID \\

{\tt Q1: glite-lb-job\_status -x JOBID } \\
{\tt Q2: glite-lb-job\_status JOBID } \\
- Q1 queries LB proxy, Q2 queries LB server - both should return status of the job \\

{\tt glite-lb-cleared.sh -x -m JOBID} \\
- logs sequence of events to JOBID pushing it to terminal state \\

{\tt Q1: glite-lb-job\_status -x JOBID } \\
{\tt Q2: glite-lb-job\_status JOBID } \\
- Q1 returns {\em error: edg\_wll\_JobStatusProxy: No such file or directory (no matching jobs found)} while Q2 returns state of the job until it is purged \\

\result\ A new job state should be available immediately at the
lbproxy and probably with a small delay also at the bkserver. Jobs that reach the final job state
are really purged from the proxy.

\note\ The test includes artificial delays. Takes approx. 50\,s to finish.


\subsection{WS interface}
%\TODO{fila, valtri: tests using Java example}
\req\ \path{glite-lb-bkserverd} running, events from \ref{normal} logged

\what\ retrieve both events and job states with the \LB\ WS interface
(operations \code{JobStatus}, \code{QueryEvents}).

\how\ \ctblb{lb-test-ws.sh}

\note\ The test includes artificial delays. Takes approx. 10\,s to finish.

\result\ the returned data should match those returned by the legacy API calls.




%\subsection{HTML interface}
%\TODO{fila: query tests using wget/curl}



%\subsection{Change ACL}
%\TODO{kouril: to be added later with new authz schema}




\subsection{Notifications}
% notifikace
% regjob, reg notifikace na vsechno, poslat udalosti, hlidat notif

All notification tests require the \LB server to be run with notifications enabled,
i.e. to be run for example with options 
\texttt{--notif-il-sock /tmp/sock.test\_notif --notif-il-fprefix /tmp/test\_notif}, 
where \texttt{/tmp/sock.test\_notif} is a socket of notification interlogger.

Please see also \cite{lbug}, Section 2.4, for other possible scenarios how to test the notification delivery.


\subsubsection{Single job, any state change}
\label{notif1}
\req\ All \LB\ services running

\what
\begin{enumerate}
\item Register a job.
\item Start a~notification client,
register with \code{edg\_wll\_NotifNew} for any state changes of the job,
and invoke \code{edg\_wll\_NotifReceive}.
\item Send events triggering job state changes.
\end{enumerate}

\how\ \ctblb{lb-test-notif.sh}

\note\ The test includes artificial delays. Takes approx. 12\,s to finish.

\result\ All the events should trigger notifications reported by the running
notification client.

%\begin{hints}
%\path{glite-lb-notify} example can be used with its commands \path{new} and \path{receive}.
%\end{hints}


\subsubsection{Include another job}
\label{notif2}
\req\ All \LB\ services running, notification from \ref{notif1} still active

\how\
\begin{enumerate}
\item Register another job.
\item Augment the notification registration with the new jobid using
\code{edg\_wll\_NotifChange}.
\item Start notification client, bind to the registration with
\code{edg\_wll\_NotifBind}.
\item Send events for the new job.
\end{enumerate}

\how\ \ctblb{lb-test-notif-switch.sh}

\note\ The test includes artificial delays. Takes approx. 25\,s to finish. Also note that this test will not work with \LB versions older than 2.0.

\result\ Notifications should be received by the client.

%\begin{hints}
%Commands \path{change} and \path{receive} of \path{glite-lb-notify}
%can be used.
%\end{hints}

% notifikace -- zmena adresy/portu
% pak poslat udalost, musi dojit
% uz je v predchozim implicitne



\subsubsection{Delayed delivery}
% notifikace -- zpozdene doruceni
% registrovat, odpojit, poslat udalosti, pripojit se

\req\ All \LB\ services running

\what\
\begin{enumerate}
\item Register another job.
\item Register a~notification as in~\ref{notif1} but terminate the client
immediately.
\item Log events for the job.
\item Restart the client, binding to the notification and call
\code{edg\_wll\_NotifReceive} repeatedly.
\end{enumerate}

\how\ \ctblb{lb-test-notif-recovery.sh}

\note\ The test includes artificial delays. Takes approx. 25\,s to finish.

\result\ Delayed notifications should be received by the client almost
immediately.


\subsubsection{Delivery to ActiveMQ}
\label{notifamq}
\req\ All \LB\ services running, ActiveMQ broker configured and running

\what
\begin{enumerate}
\item Register a job.
\item Start a~notification client,
\item Retrieve broker address from the server (since \LBver{3.2})
register with \code{edg\_wll\_NotifNew} for any state changes of the job,
and listen for messages. 
\end{enumerate}

\how\ \ctblb{lb-test-notif-msg.sh}

\note\ The test includes artificial delays. Takes approx. 25\,s to finish.

\result\ All the events should trigger notifications that will be reported 
through ActiveMQ messages.

%\begin{hints}
%The \path{glite-lb-notify} example can be used to register a notification and \path{glite-lb-cmsclient} can be used to receive messages.
%\end{hints}


\subsection{Site Notification Maintenance}
\label{notifkeeper}

\req\ All \LB\ services running.

\what
\begin{enumerate}
\item Create or modify a \texttt{site-notif.conf} file
\item Run the ``keeper'' script and check if the effect was as desired
\begin{itemize}
\item Setting up a new notification
\item Changing its conditions
\item Registering for anonymized notifications
\item \dots
\end{itemize}
\item Repeat from step~1 with different settings until all scenarios have been checked
\end{enumerate}

\how\ \ctblb{lb-test-notif-keeper.sh}

\note\ The test includes artificial delays. Takes approx. 30\,s to finish.

\result\ Notifications must be registered and messages must arrive matching the currently applicable contents of the config file.


\subsection{Server purge}

\textbf{WARNING: This test is destructive, it destroys ALL data in an
existing \LB\ database.}

The test is fairly complex but it does not make too much sense to split it
artificially.

\req\ All \LB services running, preferably a~dedicated server for this test.

\what
\begin{enumerate}
\item Purge all data on the server with \path{glite-lb-purge}
\item Log two sets of jobs, separated with delay of at least 60s so
that the sets can be distinguished from each other.
\item \label{purge1}
Using \code{edg\_wll\_JobLog} retrieve events of all the jobs
\item \label{purge2}
Purge the first set of jobs (by specifying appropriate timestamp),
letting the server dump the purged events.
\item \label{purge3} Purge the other set of jobs, also dumping the events.
\item \label{purge4} Run purge once more.
\item Check if purged jobs turned into zombies.
\item In addition, check if a \emph{cron} task exists to run the \emph{purge} operation regularly and that it logs its output correctly.
\end{enumerate}

\how\ \ctblb{lb-test-purge.pl}

\note\ The test includes artificial delays. Takes approx. 3.5\,minutes to finish.

\note\ This test is destructive to your data. You need to run it with the \texttt{-{}-i-want-to-purge} option to confirm your intention. Also, you need to provide the \LB server \texttt{address:port} explicitly as an argument to rule out any confusion. 

\result\ Data dumped in steps \ref{purge1} and \ref{purge2} should be the
same as retrieved in~\ref{purge1}. The final purge invocation should
do nothing (i.e. nothing was left in the database).

% test_purge
%\begin{hints}
%The example \path{org.glite.lb.client/examples/purge\_test} does exactly this sequence of steps,
%including the checks.
%\end{hints}

\subsection{Other Service Tests}

\subsubsection{ACL Parsing}

\req\ All \LB\ services running

\what\
\begin{enumerate}
\item Register a job.
\item Send a \code{ChangeACL} event to add authorized user.
\item Check job status to see if the event was processed correctly.
\end{enumerate}

\how\ \ctblb{lb-test-changeacl.sh}

\note\ The test includes artificial delays. Takes approx. 15\,s to finish.

\note\ This test does not check if the setting actually takes effect. Implementing that functionality -- relying on the use of at least two different identities -- in an automated test is rather challenging. 

\result\ The \emph{test~DN} should be listed in the job's ACL.

\subsubsection{Statistics}

\req\ All \LB\ services running

\what\
\begin{enumerate}
\item Register a series of jobs.
\item Generate events to push jobs to a given state.
\item Run statistics to calculate rate of jobs reaching that state. 
\item Query for average time needed by test jobs to go from one state to another.
\item Check if the statistics returned reasonable results. 
\end{enumerate}

\how\ \ctblb{lb-test-statistics.sh}

\note\ The test includes artificial delays. Takes approx. 30\,s to finish.

\result\ Meaningful values should be returned by both tests. 

\subsubsection{Multi-Threaded Operation}

\req\ \LB\ server running

\what\
\begin{enumerate}
\item Register a series of jobs.
\item Run a client using multiple threads to query the server simultaneously.
\item Check if all threads finished OK.
\end{enumerate}

\how\ \ctblb{lb-test-threaded.sh}

\note\ This is not a thorough test. It is not capable of discovering rare or improbable problems but it will check the essentials of multi-threaded opration.

\result\ The test must not hang. Meaningful results (albeit errors) must be returned by all threads.

\subsubsection{Config and Runtime File Permissions}
\label{permissions}
\req\ All \LB\ services configured and running.

\what
\begin{enumerate}
\item Decide on permission/ownership masks for various files identified for checking.
\item Check ownership and permission settings for selected files.
\item Compare that with a pre-determined mask.
\end{enumerate}

\how\ \ctblb{lb-test-permissions.sh}

\result\ The status of all files should match the pre-determined mask.


\subsubsection{Nagios Probe}
\label{permissions}
\req\ All \LB\ services configured and running, nagios probe\footnote{see also page \pageref{s:nagios}} installed.

\what
\begin{enumerate}
\item Check probe for presence
\item Run the probe
\item Check if text and exit code are OK or at least consistent
\end{enumerate}

\how\ \ctblb{lb-test-nagios-probe.sh}

\result\ The probe has returned OK and exit code was 0.

\note\ The probe includes artificial delays. The test takes approx. 10\,s to finish.

\subsubsection{HTTPs Interface}
\label{permissions}
\req\ All \LB\ services configured and running.

\what
\begin{enumerate}
\item Register a test job
\item Register a test notification
\item Query user jobs over the HTTPs interface
\item Query test job status over the HTTPs interface
\item Query notification status over the HTTPs interface
\item Download server configuration page and check if essential values are present
\item Download statistics page and check if essential values are present and $> 0$
\end{enumerate}

\how\ \ctblb{lb-test-https.sh}

\result\ All information was succesfully downloaded and matched expectations.

\subsubsection{Dump \& Load Events}
\label{dumpload}
\req\ All \LB\ services configured and running.

\what
\begin{enumerate}
\item Register tests jobs of all applicable types, including collections and DAGs
\item Generate events to change the state of all types of test jobs. At least one subjob in any type of collection must remain free of events to test embryonic registration
\item Check all those jobs are in their expected states.
\item Dump events for all test jobs
\item Purge all test jobs from the \LB server
\item Test that all test jobs were actually purged (status query must return \texttt{EIDRM})
\item Load dumped events
\item Check that all test jobs are in their expected states as before
\end{enumerate}

\how\ \ctblb{lb-test-dump-load.sh}

\result\ All jobs were first purged and then restored to their previous states.

\note\ The test registers multiple jobs for each type to reduce the probability of an accidental positive result (could be caused by random event ordering).

\note\ The probe includes artificial delays (about 12\,s worth) but also performs a lot of action. The test takes approx. 150\,s to finish.

\subsubsection{Collection-specific tests}
\label{dumpload}
\req\ All \LB\ services configured and running.

\what
\begin{enumerate}
\item Register a collection
\item Try querying for jobs by parent ID only
\end{enumerate}

\how\ \ctblb{lb-test-collections.sh}

\result\ Children were returned.


\section{LB ``In the Wild''---Real-World WMS Test}
\req\ All \LB services running, working grid infrastructure accessible (including permissions). 

\what
\begin{enumerate}
\item Submit a simple \emph{hello-world}-type job.
\item Submit a simple job and cancel it.
\item Submit a collection of simple jobs.
\item Submit a collection and cancel it. 
\item Submit a simple job that is sure to fail.
\item Submit a collection of jobs, one of which is sure to fail. 
\end{enumerate}

In all above cases: Watch the life cycle. Check the resulting state (\emph{Cleared}, \emph{Cancelled} or \emph{Aborted}). Check events received in the course of job execution; events from all relevant components must be present (NS, WM, JC, LM, and LRMS). 

\how\ \ctblb{lb-test-wild.sh}

\result\ Jobs were submitted. Cancel operation worked where applicable. Resulting state was as expected (\emph{Cleared}, \emph{Cancelled} or \emph{Aborted}). Events were received from all components as expected.

\note\ The test runs automatically. Bear in mind that in a real-life grid, even valid jobs may not always finish successfully. Analyze failures and re-run if necessary.

\note\ The number of jobs to generate may be specified using the \texttt{-n} argument

\note\ Job submissions are limited to VOCE CEs in normal circumstances. Use the \texttt{-w} argument to override. 

\note\ Due to the nature of grid computing, this test may take hours to complete!  



\section{Regression testing}

\subsection{Publishing Correct Service Version over BDII}
\req\ All \LB and BDII services running.

\what
\begin{enumerate}
\item Regression test for bug 55482 (\texttt{http://savannah.cern.ch/bugs/?55482})
\item Query for information on the server.
\item Check version returned by the query.
\end{enumerate}

\how\ \ctblb{lb-test-bdii.sh}

\result\ Query returns proper service status. Proper LB Server version is returned.

\note\ The test will also ask the \LB server for a version number through its WS interface, and compare the two values. This, however, will not be done if the machine you use to run the tests does not have the proper binaries installed (\texttt{glite-lb-ws\_getversion}), or if you do not make a proxy certificate available. Should that be the case, the test only checks if a version is returned and prints it out, without comparing.

