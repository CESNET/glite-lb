\documentclass{egee}
\usepackage{doxygen}
\usepackage{tabularx}

\def\LB{L\&B}
\def\eg{e.\,g.}
\def\ie{i.\,e.}
\def\Dash{\,---\,\penalty-1000}
\def\todo#1{\par\textbf{TODO:} #1\par}

\title{Logging and Bookkeeping}
\Subtitle{User's Guide}
\author{CESNET EGEE II JRA1 team}
\DocIdentifier{EGEE-II....}
\Date{\today}
\Activity{JRA1: Middleware Engineering and Integration}
\DocStatus{DRAFT}
\Dissemination{PUBLIC}
\DocumentLink{http://...}

\Abstract{
This user's guide explains how to use the Logging and Bookkeeping (\LB) service. 
The service architecture is described briefly. 
Examples on using \LB\ event logging command to log a~user tag and change job ACL are given,
as well as \LB\ query and notification API use cases.
% The reference section contains complete description of both the logging command and the API's.
}

\begin{document}

\begin{center}
{\bf Delivery Slip}
\end{center}
\begin{tabularx}{\textwidth}{|l|l|l|X|X|}
\hline
           & {\bf Name} & {\bf Partner} & {\bf Date} & {\bf Signature} \\
\hline
{\bf From} & Ale\v{s} K\v{r}enek & CESNET & August 1, 2008 & \\
\hline
{\bf Reviewed by} & &  & & \\

\hline
{\bf Approved by} & & & & \\
\hline
\end{tabularx}

\begin{center}
{\bf Document Change Log}
\end{center}

\begin{tabularx}{\textwidth}{|l|l|X|X|}
\hline
{\bf Issue } & {\bf Date  } & {\bf Comment } & {\bf Author  } \\   \hline
1.0.0 & August 1, 2008 & Initial version & CESNET team \\

\hline
\end{tabularx}

\begin{center}
{\bf Document Change Record}
\end{center}

\begin{tabularx}{\textwidth}{|l|l|X|}
\hline
{\bf Issue } & {\bf Item  } & {\bf Reason for Change } \\   \hline

\hline
\end{tabularx}

%
% Official text received on October 6, 2004
%
\vfill{\bf Copyright }\copyright{\bf Members of the EGEE Collaboration. 2004. 
See http://eu-egee.org/partners for details on the copyright holders. 

EGEE (``Enabling Grids for E-science in Europe'') is a project funded by
the European Union.  For more information on the project, its partners
and contributors please see http://www.eu-egee.org.

You are permitted to copy and distribute verbatim copies of this
document containing this copyright notice, but modifying this document
is not allowed. You are permitted to copy this document in whole or in
part into other documents if you attach the following reference to the
copied elements: ``Copyright }\copyright{\bf 2004. Members of the EGEE
Collaboration. http://www.eu-egee.org''

The information contained in this document represents the views of
EGEE as of the date they are published. EGEE does not guarantee that
any information contained herein is error-free, or up to date.

EGEE MAKES NO WARRANTIES, EXPRESS, IMPLIED, OR STATUTORY, BY
PUBLISHING THIS DOCUMENT.}


\clearpage

\newpage
\tableofcontents
\newpage

\section{Introduction}
%This Section should give a brief summary of what the service described
%is supposed to achieve and how it fits into the overall
%architecture. It should contain subsections on the service internal
%architecture and its relations to other services. 

The Logging and Bookkeeping (\LB) service tracks jobs managed by 
the gLite WMS (workload management system).
It gathers events from various WMS components in a~reliable way
and processes them in order to give a~higher level view, the
\emph{status of job}.

Virtually all the important
data are fed to \LB\ internally from various gLite middleware
components, transparently from the user's point of view.
On the contrary, \LB\ provides public interfaces for querying the job
information synchronously as well as registering for asynchronous
notifications.
API's for this functionality are described in this document in detail.


\subsection{Service Architecture}
Within the gLite WMS, upon creation
 each job is assigned a~unique, virtually non-recyclable
\emph{job identifier} (JobId) in an~URL form.
The server part of the URL designates the \emph{bookkeeping server} which
gathers and provides information on the job for its whole life.

High level view on the \LB\ architecture is shown in Fig.~\ref{fig-arch}
on page~\pageref{fig-arch}.

\LB\ tracks jobs in terms of \emph{events} (\eg\ \emph{Transfer} from a~WMS
component to another one, \emph{Run} and \emph{Done} when the jobs starts
and stops execution, \dots).
Each event type carries its specific attributes.
The entire architecture is specialized for this purpose and is job-centric\Dash
any event is assigned to a~unique Grid job.

\subsubsection{Event delivery and storage}
The events are gathered from various WMS components by the
\emph{\LB\ producer library}
and passed on to the \emph{locallogger} daemon,
running physically close to avoid
any sort of network problems.
The locallogger's task is storing the accepted event in a~local disk file.
Once it's done, confirmation is sent back and the logging library call
returns, reporting success.
Consequently, logging calls have local, virtually non-blocking semantics.

Further on, event delivery is managed by the \emph{interlogger} daemon.
It takes the events from the locallogger (or the disk files on crash recovery),
and repeatedly tries to deliver them to the destination
bookkeeping server (known from the JobId) until it succeeds finally.
Therefore the entire event delivery is highly reliable.
However, in the standard mode described so far it is asynchronous
(there is a~synchronous mode for special usage not discussed here)
there is no direct way for the caller to see whether an event has been
already delivered.
Our experience shows that the semantics is suitable in the prevailing number
of cases while being the most efficient in the erratic Grid environment.

The bookkeeping server processes the incoming events
to give a~higher level view
on the job states (\eg\ \emph{Submitted, Running, Done}),
each having an appropriate set of attributes again.
\LB\ provides a~public interface (Sect.~\ref{query-C})
to retrieve them via synchronous queries.

Further on, upon each event delivery to the \LB\ server the new computed 
job state is matched against the set of registered requests for notification.
If some of them match, special events\Dash\emph{notifications} are generated
and passed to a~modified 
\emph{notification interlogger}.
It takes over the notification from LB server, stores it into file and
periodically tries to deliver it to the address where the corresponding
notification client is listening.
If the user changes this address (IP or port) 
\LB\ server instructs the notification interlogger to change 
the destination of possible pending notifications.

\subsubsection{Queries}
\label{arch-queries}
One part of the \LB\ interface is the query API (Sect.~\ref{query-C}).
Two types of queries are supported\Dash\emph{job queries} which return
one or more jobs, including a~detailed description of their states,
and \emph{event queries} returning the raw \LB\ events.
In general, job queries are used to track normal life of jobs,
event queries are used mostly for tracing abnormal behaviour. 

Each query is formed of several conditions (\eg\ concrete jobid's,
owner of jobs, particular job state etc.).
The \LB\ library formats the conditions into a~query message, passes it to
the server, and waits for the response which is passed to the user
synchronously.

\subsubsection{Notifications}
\label{notification}
On the contrary, the notification API (Sect.~\ref{notify-C}) allows
the user to 
register for notifications. These are delivered to the listening
client asynchronously, when the particular event (a~change of job status
typically) occurs.
The main purpose of this \LB\ functionality is avoiding unnecessary load
on the \LB\ server serving many repeated queries (polling) with the same
result most of the time.

Using a~notification client (program that uses LB client
API to handle notifications) the user registers with a~\LB\ server
to receive notifications. 
She  must specify conditions under which the
notification is sent. These conditions are a~subset of the conditions
available for synchronous queries (Sect.~\ref{arch-queries}).
Currently due to implementation constraints, one
or more jobid's are required among the conditions and only
a~single occurrence of a~specific attribute is allowed. The registration is 
sent to the \LB\ server in the same way as synchronous queries,
and stored there.
In response, the server generates a~unique notification ID which is used
by the user to refer to this notification further on.
The user may
change conditions which trigger notification, prolong validity of
the registration, remove the registration from LB server,
or even change the destination of notifications, \ie\ the address where
a~client listens for notifications.

The registration is soft-state; it exists only for limited amount of time. The
validity is returned by LB server together with notification ID.

While the registration is valid, 
the user may stop the notification client and launch another, even
on a~different machine.
Notifications generated during the time when there was no client listening
for them are kept by the notification interlogger.
Once a~new listening address is announced to the
server, the pending notifications are delivered. 




\subsection{Security and Access Control}
\input{security}


\subsection{Interactions with other Services}
The main \LB\ functionality is keeping track of jobs managed by
the Workload Management System otherwise.
Therefore its basic usage is done internally by the WMS components.
However, the query and notification interface is completely available
to user-level applications. Upto limited extent (two event types)
this holds for the event-logging interface too.

\subsubsection{Event sources}
\LB\ events are generated by the following WMS components:
\begin{description}
\item[User Interface] registers the job with \LB\ and provides details
on transfer of the job to the resource broker.
\item[Resource Broker,] consisting of several WMS components,
logs various events as the job is passed among these components, 
as well as other important job-related information (\eg\ the chosen
destination Computing Element).
\item[Computing Element,] via the Job Wrapper script, provides the immediate
information on job execution.
\end{description}
Besides these WMS components the job payload may also log UserTag events
(see Sect.~\ref{log_usertag}) containing arbitrary user information.

Checkpointable jobs also use \LB\ to keep track of the job progress.
This is done internally by the checkpoint support library.

Finally, changes of job access control lists are done by logging
another event. This may be done directly by the user or using the WMS
user interface.

\subsubsection{Queries}

The \LB\ queries with a~user-visible output
are executed from within WMS User Interface
commands glite-job-status and glite-job-logging-info.

Besides those several WMS components use \LB\ internally to query
information on job status which is relevant for their processing.

\subsubsection{Notifications}

Notifications on job state change are used by WMS GUI 
to monitor the state of jobs periodically.





\newpage
\section{Quickstart Guide}
% The quickstart guide should explain in simple terms and with examples
% how a user is supposed to achieve the most common usecases. E.g. how
% to submit and cancel a job, how to receive a job's output. How to
% create a grid file, move it around, locate it, and delete it. How to
% monitor the progress on an application etc. 

\subsection{Command-Line Tools}
This section describes usage of event-logging \LB\ command in the two
cases which are ment for the end-user: adding a~user description (tag)
to a~job, and changing a~job access control list.

\subsubsection{Logging a UserTag event}
\label{log_usertag}
\subsection{Logging a UserTag event}
\label{log_usertag}

User tag is an arbitrary ``name=value'' pair with which the user can assign
additional information to a job. Further on, LB can be queried based also on
values of user tags.  % (see Sect.~\ref{tag-query}).

In order to add user tag for a job a special event \verb'UserTag' is used. This
event can be logged by the job owner using the glite-lb-logevent command (see
also sec.\ref{glite-lb-logevent}). Here we suppose the command is used from
user's running application because a correct setting of environment variables
needed by the command is assured.

General template for adding user tag is as follows:

\begin{verbatim}
glite-lb-logevent -s Application -e UserTag    \
        -j <job_id>                         \
        -c <seq_code>                       \
        --name <tag_name>                   \
        --value <tag_value>
\end{verbatim}

where

\begin{tabularx}{\textwidth}{lX}
\verb'<job_id>'    & specifies the job to change \\
\verb'<seq_code>'  & specifies the sequence code returned by previous call
			of verb 'glite-lb-logevent' \\
\verb'<tag_name>'  & specifies the name of user tag \\
\verb'<tag_value>' & specifies the value of user tag \\
\end{tabularx}

The user application is always executed from within a JobWrapper script (part
of Workload Management System \cite{jgc}). The wrapper  sets the  appropriate
\verb'JobId' in the environment variable \verb'GLITE_WMS_JOBID'. The user
should pass this value to the \verb'-j' option of \verb'glite-lb-logevent'.
Similarly, the wrapper sets an initial value of the event sequence code in the
environment variable \verb'GLITE_WMS_SEQUENCE_CODE'.

If the user application calls \verb'glite-lb-logevent' just once, it is
sufficient to pass this value to the \verb'-c' option.  However, if there are
more  subsequent calls,  the  user is responsible for capturing an updated
sequence code from the stdout of \verb'glite-lb-logevent' and using it in
subsequent calls. The \LB\ design requires the sequence codes in  order  to be
able to sort events correctly while not relying on strictly synchronized
clocks.  

\TODO{What are the possible values/types of the tag? String/Integer/Float/Double?}

The example bellow is a job consisting of 100 phases. A user tag phase is used
to log the phase currently being executed. Subsequently, the user may monitor
execution of the job phases as a part of the job status returned by \LB.

\begin{verbatim}
  #!/bin/sh

  for p in `seq 1 100`; do

  # log the UserTag event
  GLITE_WMS_SEQUENCE_CODE=`glite-lb-logevent -s Application
    -e UserTag
    -j $GLITE_WMS_JOBID -c $GLITE_WMS_SEQUENCE_CODE
    --name=phase --value=$p`

  # do the actual computation here
  done
\end{verbatim}

\TODO{Add more examples.}



\subsubsection{Changing Job Access Control List}
\label{change_acl}
%
%% Copyright (c) Members of the EGEE Collaboration. 2004-2010.
%% See http://www.eu-egee.org/partners for details on the copyright holders.
%% 
%% Licensed under the Apache License, Version 2.0 (the "License");
%% you may not use this file except in compliance with the License.
%% You may obtain a copy of the License at
%% 
%%     http://www.apache.org/licenses/LICENSE-2.0
%% 
%% Unless required by applicable law or agreed to in writing, software
%% distributed under the License is distributed on an "AS IS" BASIS,
%% WITHOUT WARRANTIES OR CONDITIONS OF ANY KIND, either express or implied.
%% See the License for the specific language governing permissions and
%% limitations under the License.
%
%
%% Copyright (c) Members of the EGEE Collaboration. 2004-2010.
%% See http://www.eu-egee.org/partners for details on the copyright holders.
%% 
%% Licensed under the Apache License, Version 2.0 (the "License");
%% you may not use this file except in compliance with the License.
%% You may obtain a copy of the License at
%% 
%%     http://www.apache.org/licenses/LICENSE-2.0
%% 
%% Unless required by applicable law or agreed to in writing, software
%% distributed under the License is distributed on an "AS IS" BASIS,
%% WITHOUT WARRANTIES OR CONDITIONS OF ANY KIND, either express or implied.
%% See the License for the specific language governing permissions and
%% limitations under the License.
%
\subsubsection{Example: Changing Job Access Control List}
\label{e:change-acl}

In order to change the Access Control List (ACL) for a job, a special event
\verb'ChangeACL' is used. This event can be logged by the job owner using the
\verb'glite-lb-logevent' command (see also Sect.~\ref{glite-lb-logevent}).
General template for changing the ACL is as follows:

\begin{verbatim}
glite-lb-logevent -e ChangeACL -s UserInterface -p --permission 1          
        -j <job_id>                                                     
        --user_id <user_id>                                             
        --user_id_type <user_id_type>                                   
        --permission_type <permission_type> --operation <operation>
\end{verbatim}

where

\begin{tabularx}{\textwidth}{>{\texttt}lX}
<job\_id>    & specifies the job to change \\
<user\_id>   & specifies the user to use, can be either an X.500 name
               (subject name), a VOMS group (of the form VO:Group), or a Full
               qualified attribute name (FQAN). FQANs are only supported in \LBnew. \\
<user\_id\_type>   & \texttt{0}, \texttt{1}, or \texttt{2} indicating \texttt{user\_id}
                     specifies X.500 name, VOMS group, or FQAN, respectively \\
<permission\_type> & \texttt{0} or \texttt{1} indicating the user is 
                     \textit{allowed} or \textit{denied}, respectively \\
<operation>  & \texttt{0} or \texttt{1} indicating the record carried in
               the event shall be added or removed, respectively from
               the ACL \\
\end{tabularx}


Adding a user specified by his or her subject name to the ACL \\
(\verb'user_id' = subject name, \verb'user_id_type' = 0, 
\verb'permission_type' = 0, \verb'operation' = 0):

\begin{verbatim}
glite-lb-logevent -e ChangeACL -s UserInterface -p --permission 1       \
        -j https://scientific.civ.zcu.cz:9000/PC8Y6jBitHt_fKMTEKFnVw    \
        --user_id '/O=CESNET/O=Masaryk University/CN=Daniel Kouril'     \
        --user_id_type 0 --permission_type 0 --operation 0
\end{verbatim}


Removing a user specified by his or her subject name from the ACL \\
(\verb'user_id' = subject name, \verb'user_id_type' = 0, 
\verb'permission_type' = 0, \verb'operation' = 1):

\begin{verbatim}
glite-lb-logevent -e ChangeACL -s UserInterface -p --permission 1       \
        -j https://scientific.civ.zcu.cz:9000/PC8Y6jBitHt_fKMTEKFnVw    \
        --user_id '/O=CESNET/O=Masaryk University/CN=Daniel Kouril'     \
        --user_id_type 0 --permission_type 0 --operation 1
\end{verbatim}


Adding a VOMS group to the ACL \\
(\verb'user_id' = VOMS group, \verb'user_id_type' = 1, 
\verb'permission_type' = 0, \verb'operation' = 0):

\begin{verbatim}
glite-lb-logevent -e ChangeACL -s UserInterface -p --permission 1       \
        -j https://scientific.civ.zcu.cz:9000/PC8Y6jBitHt_fKMTEKFnVw    \
        --user_id 'VOCE:/VOCE'                                          \
        --user_id_type 1 --permission_type 0 --operation 0
\end{verbatim}


Denying a particular user from accessing information about the job, can be
combined e.g. with VOMS groups (\verb'user_id' = subject name,
\verb'user_id_type' = 0, \verb'permission_type' = 1, \verb'operation' = 0):

\begin{verbatim}
glite-lb-logevent -e ChangeACL -s UserInterface -p --permission 1       \
        -j https://scientific.civ.zcu.cz:9000/PC8Y6jBitHt_fKMTEKFnVw    \
        --user_id '/O=CESNET/O=Masaryk University/CN=Daniel Kouril'     \
        --user_id_type 0 --permission_type 1 --operation 0
\end{verbatim}




%% \subsection{\LB Producer API}
%% \todo{honik}
%% This API is not public at the moment, it may change later.
%% % -*- mode: latex -*-

\section{\LB\ Logging (Producer) API}
\label{ProdOverview}
The \LB\ logging API (or producer API) is used to create and deliver
events to the \LB\ server and/or proxy, depending on the function
used:

\begin{table}[h]
\begin{tabularx}{\textwidth}{lX}
\bf Function & \bf Delivered to \\
\hline
\small\verb'edg_wll_LogEvent(...)' & asynchronously through
locallogger/interlogger to the \LB\ server \\
\small\verb'edg_wll_LogEventSync(...)' & synchronously through
locallogger/interlogger to the \LB\ server \\
\small\verb'edg_wll_LogEventProxy(...)' & through \LB\ proxy to the \LB\ server \\
\small\verb'edg_wll_Register*(...)' & directly to both \LB\ server and proxy \\
\small\verb'edg_wll_ChangeACL(...)' & directly to the \LB\ server \\
\end{tabularx}
\end{table}

These general functions take as an argument event format (which
defines the ULM string used) and variable number of arguments corresponding
to the given format. For each defined event there is predefined format
string in the form \verb'EDG_WLL_FORMAT_'\textit{EventType}, \eg\
\verb'EDG_WLL_FORMAT_UserTag', as well as three convenience functions
\verb'edg_wll_LogUserTag(...)', \verb'edg_wll_LogUserTagSync(...)',
\verb'edg_wll_LogUserTagProxy(...)'. 

For most developers (\ie\ those not developing the WMS itself) the
\verb'edg_wll_LogUserTag*(...)' and \verb'edg_wll_ChangeACL(...)' are
the only functions of interest.

\subsection{Call semantics}
\LB producer calls generally do not have transaction semantics, the
query following succesful logging call is not guaranteed to see
updated \LB server state. The typical call -- loging an event -- is
returned immediatelly and the success of the call means that the first
\LB infrastructure component takes over the event and queues it for
delivery. If you require transaction semantics, you have to use
synchronous \verb'edg_wll_LogEventSync(...)' call. 

The \LB proxy on the other hand provides a \emph{local view}
semantics, events logged into proxy using
\verb'edg_wll_LogEventProxy(...)' are guaranteed to by accessible by
subsequent queries \emph{on that proxy}.

Job registrations are all synchronous.

\subsection{Header files}

\begin{table}[h]
\begin{tabularx}{\textwidth}{>{\tt}lX}
glite/lb/producer.h & Prototypes for all event logging functions. \\
\end{tabularx}
\end{table}

\subsection{Context parameters}
The following table summarizes context parameters relevant to the
event logging. If  parameter is not set in the context explicitly, the
\LB\ library will search for value of corresponding environment
variable. The symbolic C names should be prepended with
\verb'EDG_WLL_PARAM_' prefix, \ie\ \verb'EDG_WLL_PARAM_HOST'.

\begin{table}[h]
\begin{tabularx}{\textwidth}{llX}
{\bf C name} & {\bf Env. variable} & {\bf Description} \\
\hline
\small\verb'HOST' & & Hostname that appears as event origin. \\
\small\verb'SOURCE' & & Event source component. \\
\small\verb'DESTINATION' & \small\verb'GLITE_WMS_LOG_DESTINATION' & Hostname of machine running
locallogger/interlogger. \\
\small\verb'DESTINATION_PORT' & \small\verb'GLITE_WMS_LOG_DESTINATION' & Port the locallogger is listening
on. \\
\small\verb'LOG_TIMEOUT' & \small\verb'GLITE_WMS_LOG_TIMEOUT' & Logging timeout for asynchronous
logging. \\
\small\verb'LOG_SYNC_TIMEOUT' & \small\verb'GLITE_WMS_LOG_SYNC_TIMEOUT' & Logging timeout for synchronous
logging. \\
\small\verb'LBPROXY_STORE_SOCK' & \small\verb'GLITE_WMS_LBPROXY_STORE_SOCK' & \LB\ Proxy store socket path (if
logging through \LB\ Proxy) \\
\small\verb'LBPROXY_USER' & \small\verb'GLITE_WMS_LBPROXY_USER' & Certificate subject of the user (if
logging through \LB\ proxy).
\end{tabularx}
\end{table}
The \verb'GLITE_WMS_LOG_DESTINATION' environment variable contains
both locallogger host and port separated by colon (\ie\ ``host:port'').

\subsection{Return values}
The logging functions return 0 on success and one of {\texttt EINVAL,
ENOSPC, ENOMEM, ECONNREFUSED, EAGAIN} on error. If {\texttt EAGAIN} is
returned, the function should be called again to retry the delivery;
it is not guaranteed, however, that the event was not delivered by the
first call. Possibly duplicated events are discarded by the \LB\
server or proxy.

The synchronous variants of logging functions can in addition return
\verb'EDG_WLL_ERROR_NOJOBID' or \verb'EDG_WLL_ERROR_DB_DUP_KEY'.

\subsection{Logging events}
In this section we will give an example how to log an UserTag event to
the \LB.

First we have to include neccessary headers:
\lstinputlisting[firstline=8,lastline=10]{prod_example1.c}

Initialize context and set parameters:
\lstinputlisting[firstline=61,lastline=84]{prod_example1.c}

\TODO{proper setting of sequence codes}
\lstinputlisting[firstline=86,lastline=91]{prod_example1.c}

Log the event:
\lstinputlisting[firstline=93,lastline=107]{prod_example1.c}

The \verb'edg_wll_LogEvent()' function is defined as follows:
\begin{lstlisting}[numbers=none]
extern int edg_wll_LogEvent(
        edg_wll_Context context,
        edg_wll_EventCode event,
        char *fmt, ...);
\end{lstlisting}
If you use this function, you have to provide event code, format
string and corresponding arguments yourself. The UserTag event has
only two arguments, tag name and value, but other events require more
arguments. 

Instead of using the generic \verb'edg_wll_LogEvent()', we could also
write:
\begin{lstlisting}[firstnumber=92]
err = edg_wll_LogUserTag(ctx, name, value);
\end{lstlisting}

\subsection{Java binding}



\subsection{\LB\ Querying API}
%\todo{valtri}
%
%% Copyright (c) Members of the EGEE Collaboration. 2004-2010.
%% See http://www.eu-egee.org/partners for details on the copyright holders.
%% 
%% Licensed under the Apache License, Version 2.0 (the "License");
%% you may not use this file except in compliance with the License.
%% You may obtain a copy of the License at
%% 
%%     http://www.apache.org/licenses/LICENSE-2.0
%% 
%% Unless required by applicable law or agreed to in writing, software
%% distributed under the License is distributed on an "AS IS" BASIS,
%% WITHOUT WARRANTIES OR CONDITIONS OF ANY KIND, either express or implied.
%% See the License for the specific language governing permissions and
%% limitations under the License.
%
% -*- mode: latex -*-

\section{\LB\ Querying (Consumer) API}
\label{s:Consumer-API}
The \LB Consumer API is used to obtain information from \LB server
or Proxy using simple query language (see
Sect.~\ref{s:querylang}). There are two types of queries based
on the results returned:
\begin{itemize}
\item query for events -- the result contains events satisfying given
criteria,
\item query for jobs -- the result contains JobId's and job states of jobs
satisfying given criteria.
\end{itemize}
The potential result sets can be very large; the \LB server imposes
limits on the result set size, which can be further restricted by the
client.


\subsection{Query Language}
\label{s:querylang}
The \LB query language is based on simple value assertions on job and
event attributes. There are two types of queries based on the
complexity of selection criteria, \textit{simple} and
\textit{complex}.
Simple queries are can be described by the following formula:
\begin{displaymath}
\textit{attr}_1 \mathop{\textrm{ OP }} \textit{value}_1 \wedge \dots \wedge
\textit{attr}_n \mathop{\textrm{ OP }} \textit{value}_n
\end{displaymath}
where $\textit{attr}_i$ is attribute name, $\mathop{\textrm{ OP }}$ is
one of the $=$, $<$, $>$, $\neq$ and $\in$ relational operators and
$\textit{value}$ is single value (or, in the case of $\in$ operator,
interval) from attribute type.

Complex queries can be described using the following formula:
\begin{multline*}
(\textit{attr}_1 \mathop{\textrm{ OP }} \textit{value}_{1,1} \vee \dots \vee
\textit{attr}_1 \mathop{\textrm{ OP }} \textit{value}_{1,i_1}) \wedge \\
(\textit{attr}_2 \mathop{\textrm{ OP }} \textit{value}_{2,1} \vee \dots \vee
\textit{attr}_2 \mathop{\textrm{ OP }} \textit{value}_{2,i_2}) \wedge \\
\vdots \\
\wedge (\textit{attr}_n \mathop{\textrm{ OP }} \textit{value}_{n,1} \vee \dots \vee
\textit{attr}_n \mathop{\textrm{ OP }} \textit{value}_{n,i_n})
\end{multline*}
The complex query can, in contrast to simple query, contain more
assertions on value of single attribute, which are ORed together.

\marginpar{Indexed attributes}%
The query must always contain at least one attribute indexed
on the \LB server; this restriction is necessary to avoid matching the
selection criteria against all jobs in the \LB database. The list of
indexed attributes for given \LB server can be obtained by \LB API
call.

\subsection{C Language Binding}

\subsubsection{Call Semantics}
The \LB server queries are, in contrast to logging event calls,
synchronous (for asynchronous variant see Sect.~\ref{s:Notification-API},
notifications). The server response contains \jobid's, job states
and/or events known to the server at the moment of processing the
query. Due to the asynchronous nature of event delivery it may not
contain all data that was actually sent; the job state computation is
designed to be resilient to event loss to some extent.

\marginpar{Result size limits}%
When the item count returned by \LB\ server exceeds the defined
limits, the \verb'E2BIG' error occur. There are two limits\,---\,the server
and the user limit. The user defined limit may be set in the context
at the client side, while the server imposed limit is configured at
the server and can be only queried by the client. The way the \LB
library and server handles the over--limit result size can be
specified by setting context parameter
\verb'EDG_WLL_PARAM_QUERY_RESULTS' to one of the following values:
\begin{itemize}
\item \verb'EDG_WLL_QUERYRES_NONE'\,---\,In case the limit is reached,
no results are returned at all.
\item \verb'EDG_WLL_QUERYRES_LIMITED'\,---\,A result contains at most
``limit'' item count.
\item \verb'EDG_WLL_QUERYRES_ALL'\,---\,All results are returned and
limits have no effect. This option is available only in special cases
such as ``user jobs query'' and  the ``job status query''. Otherwise
the EINVAL error is returned.
\end{itemize}
Default value is \verb'EDG_WLL_QUERYRES_NONE'.


\subsubsection{Header Files}
\begin{table}[h!]
\begin{tabularx}{\textwidth}{>{\tt}lX}
glite/lb/consumer.h & Prototypes for all query functions. \\
\end{tabularx}
\end{table}

\subsubsection{Context Parameters}
The table~\ref{t:ccontext} shows parameters relevant to the query API.

\begin{table}[h!]
\begin{tabularx}{\textwidth}{lX}
{\bf Name} & {\bf Description} \\
\hline
\lstinline'EDG_WLL_PARAM_QUERY_SERVER' &
Default server name to query.
\\
\lstinline'EDG_WLL_PARAM_QUERY_SERVER_PORT' &
Default server port to query.
\\
\lstinline'EDG_WLL_PARAM_QUERY_SERVER_OVERRIDE' &
host:port parameter setting override even values in \jobid (useful for
debugging \& hacking only)
\\
\lstinline'EDG_WLL_PARAM_QUERY_TIMEOUT' &
Query timeout.
\\
\lstinline'EDG_WLL_PARAM_QUERY_JOBS_LIMIT' &
Maximal query jobs result size.
\\
\lstinline'EDG_WLL_PARAM_QUERY_EVENTS_LIMIT' &
Maximal query events result size.
\\
\lstinline'EDG_WLL_PARAM_QUERY_RESULTS' &
Flag to indicate handling of too large results.
\\
\end{tabularx}
\caption{Consumer specific context parameters}
\label{t:ccontext}
\end{table}


\subsubsection{Return Values}
\LB\ server returns errors which are classified as hard and soft errors.
The main difference between these categories is that in the case of soft
errors results may still be returned. The authorization errors belong to
``soft error'' sort. Hard errors like \verb'ENOMEM' are typically all
unrecoverable, to obtain results the query must be repeated, possibly
after correcting the failure condition the error indicated.

Depending on the setting of context parameter
\verb'EDG_WLL_PARAM_QUERY_RESULTS', the \verb'E2BIG' error may fall into both
categories. 


\subsubsection{Query Condition Encoding}
\label{s:queryrec}
The \LB query language is mapped into (one- or two-dimensional) array
of attribute value assertions represented by
\verb'edg_wll_QueryRec' structure:
\begin{lstlisting}
typedef struct _edg_wll_QueryRec {
        edg_wll_QueryAttr       attr;   //* \textit{attribute to query}
        edg_wll_QueryOp         op;     //* \textit{query operation}
        union {
                char *                  tag;    //* \textit{user tag name / JDL attribute "path"}
                edg_wll_JobStatCode     state;  //* \textit{job status code}
        } attr_id;
        union edg_wll_QueryVal {
                int     i;      	//* \textit{integer query attribute value}
                char    *c;     	//* \textit{character query attribute value}
                struct timeval  t;      //* \textit{time query attribute value}
                glite_jobid_t   j;      //* \textit{JobId query attribute value}
        } value, value2;
} edg_wll_QueryRec;
\end{lstlisting}

% \TODO{pro prehlednost bych mozna pridal seznam vsech atributu na ktere se lze ptat}

The table~\ref{t:cqueryattr} shows the most common query attributes.
For a complete list see \texttt{query\_rec.h}.

\begin{table}[ht]
\begin{tabularx}{\textwidth}{lX}
{\bf Name} & {\bf Description} \\
\hline
\lstinline'EDG_WLL_QUERY_ATTR_JOBID' & Job ID to query. \\
\lstinline'EDG_WLL_QUERY_ATTR_OWNER' & Job owner. \\
\lstinline'EDG_WLL_QUERY_ATTR_STATUS' & Current job status. \\
\lstinline'EDG_WLL_QUERY_ATTR_LOCATION' & Where is the job processed. \\
\lstinline'EDG_WLL_QUERY_ATTR_DESTINATION' & Destination CE. \\
\lstinline'EDG_WLL_QUERY_ATTR_DONECODE' & Minor done status (OK,failed,cancelled). \\
\lstinline'EDG_WLL_QUERY_ATTR_USERTAG' & User tag. \\
\lstinline'EDG_WLL_QUERY_ATTR_JDL_ATTR' & Arbitrary JDL attribute. \\
\lstinline'EDG_WLL_QUERY_ATTR_STATEENTERTIME' & When entered current status. \\
\lstinline'EDG_WLL_QUERY_ATTR_LASTUPDATETIME' & Time of the last known event of the job. \\
\lstinline'EDG_WLL_QUERY_ATTR_JOB_TYPE' & Job type. \\
\end{tabularx}
\caption{Query record specific attributes.}
\label{t:cqueryattr}
\end{table}

The table~\ref{t:cqueryop} shows all supported query operations. 

\begin{table}[ht]
\begin{tabularx}{\textwidth}{lX}
{\bf Name} & {\bf Description} \\
\hline
\lstinline'EDG_WLL_QUERY_OP_EQUAL' & Attribute is equal to the operand value. \\
\lstinline'EDG_WLL_QUERY_OP_LESS' & Attribute is grater than the operand value. \\
\lstinline'EDG_WLL_QUERY_OP_GREATER' & Attribute is less than the operand value. \\
\lstinline'EDG_WLL_QUERY_OP_WITHIN' & Attribute is in given interval. \\
\lstinline'EDG_WLL_QUERY_OP_UNEQUAL' & Attribute is not equal to the operand value. \\
\lstinline'EDG_WLL_QUERY_OP_CHANGED' & Attribute has changed from last check (supported since \LBver{2.0} in notification matching). \\
\end{tabularx}
\caption{Query record specific operations.}
\label{t:cqueryop}
\end{table}



\subsubsection{Query Jobs Examples}
\label{s:qjobs}

The simplest use case corresponds to the situation when an exact job ID
is known and the only information requested is the job status. The job ID
format is described in~\cite{djra1.4}. Since \LBver{2.0}, it is also possible to
query all jobs belonging to a specified user, VO or RB.

The following example shows how to retrieve the status information
about all user's jobs running at a specified CE.

First we have to include neccessary headers:
\lstinputlisting[title={\bf File: }\lstname,numbers=left,linerange=headers-end\ headers]{cons_example1.c}

Define and initialize variables:
\lstinputlisting[title={\bf File: }\lstname,numbers=left,linerange=variables-end\ variables]{cons_example1.c}

Initialize context and set parameters:
\lstinputlisting[title={\bf File: }\lstname,numbers=left,linerange=context-end\ context]{cons_example1.c}

Set the query record to \emph{all (user's) jobs running at CE 'XYZ'}:
\lstinputlisting[title={\bf File: }\lstname,numbers=left,linerange=queryrec-end\ queryrec]{cons_example1.c}

Query jobs:
\lstinputlisting[title={\bf File: }\lstname,numbers=left,linerange=query-end\ query]{cons_example1.c}

Now we can for example print the job states:
\lstinputlisting[title={\bf File: }\lstname,numbers=left,linerange=printstates-end\ printstates]{cons_example1.c}


In many cases the basic logic using only conjunctions is not sufficient.
For example, if you need all your jobs running at the destination XXX or at
the destination YYY, the only way to do this with the \texttt{edg\_wll\_QueryJobs()}
call is to call it twice. The \texttt{edg\_wll\_QueryJobsExt()} call allows to make
such a~query in a single step.
The function accepts an array of condition lists. Conditions within a~single list are
OR-ed and the lists themselves are AND-ed.

The next query example describes how to get all user's jobs running at
CE 'XXX' or 'YYY'. 

We will need an array of three conditions (plus one last empty):

\lstinputlisting[title={\bf File: }\lstname,numbers=left,linerange=variables-end\ variables]{cons_example2.c}

The query condition is the following:

\lstinputlisting[title={\bf File: }\lstname,numbers=left,linerange=queryrec-end\ queryrec]{cons_example2.c}

As can be clearly seen, there are three lists supplied to
\texttt{edg\_wll\_QueryJobsExt()}. The first list specifies the owner of the
job, the second list provides the required status (\texttt{Running}) and
the last list specifies the two destinations.
The list of lists is terminated with \texttt{NULL}.
This query equals to the formula
\begin{quote}
\texttt{(user=NULL) and (state=Running) and (dest='XXX' or dest='YYY')}.
\end{quote}

To query the jobs, we simply call
\lstinputlisting[title={\bf File: }\lstname,numbers=left,linerange=query-end\ query]{cons_example2.c}



\subsubsection{Query Events Examples}

Event queries and job queries are similar. Obviously, the return type is
different \Dash the \LB\ raw events. There is one more input parameter
representing specific conditions on events (possibly empty) in addition to
conditions on jobs.

The following example shows how to select all events (and therefore jobs)
marking red jobs (jobs that were marked red at some time in the past) as green.

\lstinputlisting[title={\bf File: }\lstname,numbers=left,linerange=variables-end\ variables]{cons_example3.c}

\lstinputlisting[title={\bf File: }\lstname,numbers=left,linerange=queryrec-end\ queryrec]{cons_example3.c}

This example uses \texttt{edg\_wll\_QueryEvents()} call. Two condition lists are
given to \texttt{edg\_wll\_QueryEvents()} call. One represents job conditions and
the second represents event conditions. These two lists are joined together with
logical and (both condition lists have to be satisfied). This is necessary as
events represent a state of a job in a particular moment and this changes in time.

\lstinputlisting[title={\bf File: }\lstname,numbers=left,linerange=query-end\ query]{cons_example3.c}

The \texttt{edg\_wll\_QueryEvents()} returns matched events and save them in the
\texttt{eventsOut} variable. Required job IDs are stored in the edg\_wll\_Event
structure.

\lstinputlisting[title={\bf File: }\lstname,numbers=left,linerange=printevents-end\ printevents]{cons_example3.c}

In a similar manor to \texttt{edg\_wll\_QueryJobsExt()}, there exists also \texttt{edg\_wll\_QueryEventsExt()} 
that can be used to more complex queries related to events. See also \texttt{README.queries} for more examples.


Last \LB Querying API call is \texttt{edg\_wll\_JobLog()} that returns all events related to a single job.
In fact, it is a convenience wrapper around \texttt{edg\_wll\_QueryEvents()} and its usage is clearly
demonstrated in the client example \texttt{job\_log.c} (in the client module).



\subsection{C++ Language Binding}
The querying C++ \LB API is modelled after the C \LB API using these basic principles:
\begin{itemize}
\item queries are expressed as vectors of
\verb'glite::lb::QueryRecord' instances,
\item \LB context and general query methods are represented by class
\verb'glite::lb::ServerConnection',
\item \LB job specific queries are encapsulated within class
\verb'glite::lb::Job',
\item query results are returned as (vector or list of)
\verb'glite::lb::Event' or \verb'glite::lb::JobStatus' read-only instances.
\end{itemize}


\subsubsection{Header Files}
Header files for the \LB consumer API are summarized in table~\ref{t:ccppheaders}.
\begin{table}[h]
\begin{tabularx}{\textwidth}{>{\tt}lX}
glite/lb/Event.h & Event class for event query results. \\
glite/lb/JobStatus.h & JobStatus class for job query results. \\
glite/lb/ServerConnection.h & Core of the C++ \LB API, defines
\verb'QueryRecord' class for specifying queries and
\verb'ServerConnection' class for performing the queries. \\
glite/lb/Job.h & Defines \verb'Job' class with methods for job
specific queries. \\
\end{tabularx}
\caption{Consumer C++ API header files}
\label{t:ccppheaders}
\end{table}

\subsubsection{QueryRecord}
The \verb'glite::lb::QueryRecord' class serves as the base for mapping
the \LB query language into C++, similarly to the C counterpart
\verb'edg_wll_QueryRecord'. The \verb'QueryRecord' object represents
condition on value of single attribute:
\begin{lstlisting}
using namespace glite::lb;

QueryRecord a(QueryRecord::OWNER, QueryRecord::EQUAL, "me");
\end{lstlisting}
The \verb'QueryRecord' class defines symbolic names for attributes (in
fact just aliases to \verb'EDG_WLL_QUERY_ATTR_' symbols described in table\
\ref{t:cqueryattr}) and for logical operations (aliases to
\verb'EDG_WLL_QUERY_OP_' symbols, table\ \ref{t:cqueryop}). The last
parameter to the \verb'QueryRecord' constructor is the attribute
value.

There are constructors with additional arguments for specific
attribute conditions or logical operators that require it, \ie\
the \verb'QueryRecord::WITHIN' operator and queries about state enter
times. The query condition ``job that started running between \verb'start'
and \verb'end' times' can be represented in the following way:
\begin{lstlisting}
struct timeval start, end;

QueryRecord a(QueryRecord::TIME, QueryRecord::WITHIN, JobStatus::RUNNING, 
              start, end);
\end{lstlisting}


\subsubsection{Event}
The objects of class \verb'glite::lb::Event' are returned by the \LB event
queries. The \verb'Event' class intgstr	roduces symbolic names for event
type (enum \verb'Event::Type'), event attributes (enum
\verb'Event::Attr') and their types (enum
\verb'Event::AttrType'), feature not available through the C API, as
well as (read only) access to the attribute values.  Using 
these methods you can:
\begin{itemize}
\item get the event type (both symbolic and string):
\begin{lstlisting}
      Event event;

      // we suppose event gets somehow filled in
      cout << "Event type: " << event.type << endl;
   
      cout << "Event name:" << endl;
      // these two lines should print the same string
      cout << Event::getEventName(event.type) << endl;
      cout << event.name() << endl;
\end{lstlisting}
\item get the list of attribute types and values (see
line~\ref{l:getattrs} of the example),
\item get string representation of attribute names,
\item get value of given attribute.
\end{itemize}
The following example demonstrates this by printing event name and attributes:
\lstinputlisting[title={\bf File:}\lstname,numbers=left,linerange=event-end\ event]{util.C}

\subsubsection{JobStatus}
The \verb'glite::lb::JobStatus' is a result type of job status
queries in the same way the \verb'glite::lb::Event' is used in event
queries. The \verb'JobStatus' class provides symbolic names for job
states (enum \verb'JobStatus::Code'), state attributes
(enum \verb'JobStatus::Attr') and their types (enum
\verb'JobStatus::AttrType'), and read only access to the
attribute values. Using the \verb'JobStatus' interface you can:
\begin{itemize}
\item get the string name for the symbolic job state:
\begin{lstlisting}
	JobStatus status;

	// we suppose status gets somehow filled in
        cout << "Job state: " << status.type << endl;

        cout << "State name: " << endl;
        // these two lines should print the same string
        cout << JobStatus::getStateName(status.type) << endl;
        cout << status.name() << endl;
\end{lstlisting}
\item get the job state name (both symbolic and string),
\item get the list of job state attributes and types,	
\item convert the attribute names from symbolic to string form and
vice versa,
\item get value of given attribute.
\end{itemize}
The following example demostrates this by printing job status (name
and attributes):
\lstinputlisting[title={\bf File:}\lstname,numbers=left,linerange=status-end\ status]{util.C}

\subsubsection{ServerConnection}\label{s:ServerConnection}
The \verb'glite::lb::ServerConnection' class represents particular \LB
server and allows for queries not specific to particular job (these
are separated into \verb'glite::lb:Job' class). The
\verb'ServerConnection' instance thus encapsulates client part of
\verb'edg_wll_Context' and general query methods.

There are accessor methods for every consumer context parameter listed
in table \ref{t:ccontext}, \eg for \verb'EDG_WLL_PARAM_QUERY_SERVER'
we have the following methods:
\begin{lstlisting}
void setQueryServer(const std::string& host, int port);
std::pair<std::string, int> getQueryServer() const;
\end{lstlisting}
We can also use the generic accessors defined for the parameter types
\verb'Int', \verb'String' and \verb'Time', \eg:
\begin{lstlisting}
void setParam(edg_wll_ContextParam name, int value);
int getParamInt(edg_wll_ContextParam name) const;
\end{lstlisting}

The \verb'ServerConnection' class provides methods for both event and job queries:
\begin{lstlisting}
void queryJobs(const std::vector<QueryRecord>& query,
	       std::vector<glite::jobid::JobId>& jobList) const;

void queryJobs(const std::vector<std::vector<QueryRecord> >& query,
	       std::vector<glite::jobid::JobId>& jobList) const;

void queryJobStates(const std::vector<QueryRecord>& query, 
		    int flags,
		    std::vector<JobStatus> & states) const;

void queryJobStates(const std::vector<std::vector<QueryRecord> >& query, 
		    int flags,
		    std::vector<JobStatus> & states) const;

void queryEvents(const std::vector<QueryRecord>& job_cond,
		 const std::vector<QueryRecord>& event_cond,
		 std::vector<Event>& events) const;

void queryEvents(const std::vector<std::vector<QueryRecord> >& job_cond,
		 const std::vector<std::vector<QueryRecord> >& event_cond,
		 std::vector<Event>& eventList) const;
\end{lstlisting}
You can see that we use \verb'std::vector' instead of \verb'NULL' terminated
arrays for both query condition lists and results. The API does
not differentiate simple and extended queries by method name
(\verb'queryJobs' and \verb'queryJobsExt' in C), but by parameter
type (\verb'vector<QueryRecord>'
vs. \verb'vector<vector<QueryRecord>>'). On the other hand there are
different methods for obtaining \jobid's and full job states as well 
as convenience methods for getting user jobs. 

Now we can show the first example of job query from section
\ref{s:qjobs} rewritten in C++. First we have to include the headers:
\lstinputlisting[title={\bf File: }\lstname,numbers=left,linerange=headers-end\ headers]{cons_example1.cpp}

Define variables:
\lstinputlisting[title={\bf File: }\lstname,numbers=left,linerange=variables-end\ variables]{cons_example1.cpp}

Initialize server object:
\lstinputlisting[title={\bf File: }\lstname,numbers=left,linerange=queryserver-end\ queryserver]{cons_example1.cpp}

Create the query condition vector:
\lstinputlisting[title={\bf File: }\lstname,numbers=left,linerange=querycond-end\ querycond]{cons_example1.cpp}

Perform the query:
\lstinputlisting[title={\bf File: }\lstname,numbers=left,linerange=query-end\ query]{cons_example1.cpp}

Print the results:
\lstinputlisting[title={\bf File: }\lstname,numbers=left,linerange=printstates-end\ printstates]{cons_example1.cpp}

The operations can throw an exception, so the code should be enclosed
within try--catch clause.

The second example rewritten to C++ is shown here; first the query
condition vector:
\lstinputlisting[title={\bf File: }\lstname,numbers=left,linerange=queryrec-end\ queryrec]{cons_example2.cpp}

The query itself:
\lstinputlisting[title={\bf File: }\lstname,numbers=left,linerange=query-end\ query]{cons_example2.cpp}

The third example shows event query (as opposed to job state query in
the first two examples). We are looking for events of jobs, that were
in past painted (tagged by user) green, but now they are red. The
necessary query condition vectors are here:
\lstinputlisting[title={\bf File: }\lstname,numbers=left,linerange=queryrec-end\ queryrec]{cons_example3.cpp}

The query itself:
\lstinputlisting[title={\bf File: }\lstname,numbers=left,linerange=query-end\ query]{cons_example3.cpp}

The resulting event vector is dumped using the utility function
\verb'dumpEvent()' listed above:
\lstinputlisting[title={\bf File: }\lstname,numbers=left,linerange=printevents-end\ printevents]{cons_example3.cpp}


\subsubsection{Job}
The \verb'glite::lb::Job' class encapsulates \LB server queries
specific for particular job as well as client part of context. The
\verb'Job' object provides method for getting the job status and the
event log (\ie all events belonging to the job):
\begin{lstlisting}
JobStatus status(int flags) const;

void log(std::vector<Event> &events) const;
\end{lstlisting}

\marginpar{\bf Important!}%
It is important to notice that \verb'Job' contain
\verb'ServerConnection' as private member and thus encapsulate client
part of context. That makes them relatively heavy--weight objects and
therefore it is not recommended to create too many instances, but
reuse one instance by assigning different \jobid's to it.


\subsection{Web-Services Binding}\label{s:Consumer-API-WS}

\TODO{ljocha: Complete review, list of all relevant (WSDL) files, their location, etc.}

In this section we describe the operations defined in the \LB\ WSDL
file (\texttt{LB.wsdl}) as well as its custom types (\texttt{LBTypes.wsdl}).

For the sake of readability this documentation does not follow the structure
of WSDL strictly, avoiding to duplicate information which is already present
here. Consequently, the SOAP messages are not documented, for example, as they
are derived from operation inputs and outputs mechanically.
The same holds for types: \eg\ we do not document defined elements
which correspond 1:1 to types but are required due to the literal SOAP
encoding.

For exact definition of the operations and types see the WSDL file.

\TODO{ljocha: Add fully functional WS examples - in Java, Python, C?}


Aby se na to neapomnelo:

perl-SOAP-Lite-0.69 funguje
perl-SOAP-Lite-0.65 ne 	(stejne rve document/literal support is EXPERIMENTAL in SOAP::Lite ), tak ma asi pravdu


musi mit metodu ns()




\subsection{\LB\ Notification API}
\section{\LB\ Notification API}

The purpose of this section is demonstrating the usage of the \LB\ notification API. Two examples of basic API usage are given.


\subsection{Registering and receiving notification}

The following example registers on \LB\ server to receive notifications triggered by events belonging to job with \verb'jobid' and waits for notification until \verb'timeout'. 
The code assumes the user to prepare a~reasonable value in \verb'jobid'
(\ie\ identifying an existing job).

%The glite-lb-bkserverd and glite-lb-notif-interlogd daemons have to be running. The first one user registers to, the second one delivers notifications to the example program (as described in \ref{notification}).

\begin{verbatim}
  #include <time.h>
  #include <stdio.h>
  #include <stdlib.h>
  #include <string.h>
  #include <unistd.h>

  #include "glite/lb/context.h"
  #include "glite/lb/lb_gss.h"
  #include "glite/lb/notification.h"

  /* jobid magically appears here */
  char                    *jobid;

  edg_wll_Context         ctx;
  edg_wll_QueryRec        **conditions;
  edg_wll_NotifId         notif_id = NULL, recv_notif_id = NULL;
  edg_wlc_JobId           my_jobId = NULL;
  edg_wll_JobStat         stat;
  time_t                  valid;
  struct timeval          timeout = {220, 0};
  ...

  edg_wll_InitContext(&ctx);

  memset(&stat, 0, sizeof(stat));

  conditions = (edg_wll_QueryRec **)calloc(2,sizeof(edg_wll_QueryRec *));
  conditions[0] = (edg_wll_QueryRec *)calloc(2,sizeof(edg_wll_QueryRec));

  edg_wlc_JobIdParse(jobid, &my_jobId);

  conditions[0][0].attr = EDG_WLL_QUERY_ATTR_JOBID;
  conditions[0][0].op = EDG_WLL_QUERY_OP_EQUAL;
  conditions[0][0].value.j = my_jobId;

  /* register notification on BK server */
  if (edg_wll_NotifNew(ctx, conditions, -1, NULL, &notif_id, &validity))
    goto error;

  /* the ID string my be used to receive notifications */
  /* from another computer later on                    */
  printf("notification ID: %s\n", edg_wll_NotifIdUnparse(notif_id));

  if (edg_wll_NotifReceive(ctx, -1, &timeout, &status, &recv_notif_id))
    /* timeout or error */
    goto error;
  else {
    /* notification arrived */
    /* check recv_notif_id if you have registered more notifications */
    /* to know which one you received. If you have just this one     */
    /* do not bother.                                                */

    printf("Status of my job is: %s\n", edg_wll_StatToString(stat.state));
    edg_wll_FreeStatus(&stat);
    edg_wll_NotifIdFree(recv_notif_id);
  }
 
  /* Release registration on BK server. Don't do this if notif_id is reused. */
  edg_wll_NotifDrop(ctx, notif_id);

  edg_wll_NotifIdFree(notif_id);
  edg_wll_NotifCloseFd(ctx);
	
  ...

error:
	/* clean-up */
  ...

\end{verbatim}

First of all the context is initialised. During this procedure user's credentials are loaded. %(see \ref{cmdln_interface} for information on environmental variables pointing to user's X509 credentials). 

Then conditions under which notifications are sent are set. In this example, user is notified every time when event with given jobId is logged to \LB\ server. For more complicated conditions, please, consider the conditions limitations mentioned in \ref{notification}.

Afterwards, a new registration is created and a unique notification ID is
returned.
Notifications are recieved with the \verb'edg_wll_NotifReceive' call.
If no notification is ready for
delivery, the call waits until some notification arrival or timeout.

If user does not want to receive notifications any more, \verb'edg_wll_NotifDrop' call removes the registration for notifications from \LB server.


\subsection{Changing destination for notifications}

The second example illustrates how to receive notifications from different host or different application. 

Let us suppose that user is registered for receiving notifications on \LB\ server. She obtained a notification ID and information how long this registration will be valid as described in the previous example but she did not delete registration with \verb'edg_wll_NotifDrop'.

If the registration is still valid, she can start new client for receiving notifications, even from different machine, using notification ID returned during registration in previous step.

The sequence of function calls (without catching errors) would be as follows:

\begin{verbatim}
  edg_wll_NotifId         nid;
  char                    *ns;  // notification ID string returned 
                                // in previous example
  ...
  edg_wll_NotifIdParse(ns, &nid);
  edg_wll_NotifBind(ctx, nid, -1, NULL, &valid);

  /* prolong the registration validity if necessary */
  if (...validity close to expire...)
    edg_wll_NotifRefresh(ctx, nid, &valid);  

  edg_wll_NotifReceive(ctx, -1, &timeout, &status, &recv_notif_id);
  ...
\end{verbatim}

The call \verb'edg_wll_NotifReceive' will return any notification
that was generated while there was no listening client for it.
If there are none, it will wait for new notifications as in the previous
example.



\newpage
\section{Reference Guide}

%The reference guide should contain detailed descriptions of all
%provided CLIs and APIs. There should be two subsections for those. 

\subsection{Command-Line Interfaces}
\label{cmdln_interface} 
Besides the API's \LB\ offers its users a simple command-line interface for
logging events. The command glite-lb-logevent is used for this purpose. However, it
is intended for internal WMS debugging tests in the first place and should not
be used for common event logging because of possibility of confusing \LB\
server job state automaton.

The only legal user usage is for logging \verb'UserTag' and \verb'ChangeACL' events. The following description is therefore concentrating only on options dealing with these two events.

Command usage is:

\begin{verbatim}
    glite-lb-logevent [-h] [-p] [-c seq_code] \
        -j <dg_jobid> -s Application -e <event_name> [key=value ...]
\end{verbatim}

where

\begin{tabularx}{\textwidth}{lX}
\texttt{  -h  -{}-help} &           this help message\\
\texttt{  -p  -{}-priority} &       send a priority event\\
\texttt{  -c  -{}-sequence} &       event sequence code\\
\texttt{  -j  -{}-jobid} &          JobId\\
\texttt{  -e  -{}-event} &           select event type (see -e help)\\
\end{tabularx}

\medskip

Each event specified after \verb'-e' option has different sub-options enabling to set event specific values.

Sub-options usable with \verb'UserTag' event are:


\begin{tabularx}{\textwidth}{lX}
\texttt{      -{}-name}  &          tag name\\
\texttt{      -{}-value} &          tag value\\
\end{tabularx}

\medskip

Sub-options usable with \verb'ChangeACL' event are:

\begin{tabularx}{\textwidth}{lX}
\texttt{      -{}-operation} &       operation requested to perform with ACL (add, remove)\\
\texttt{      -{}-permission} &      ACL permission to change (currently only READ)\\
\texttt{      -{}-permission\_type} & type of permission requested (0 = 'allow', 1 = 'deny')\\
\texttt{      -{}-user\_id} &         DN or VOMS parameter (in format VO:group)\\
\texttt{      -{}-user\_id\_type} &    type of information given in \verb'user_id' (DN or VOMS)\\
\end{tabularx}

\bigskip

To be able to use this command several environmental variables must be set properly. User must specify where the event should be sent. This is address and port of glite-lb-logd daemon. It is done using environmental variable \verb'EDG_WL_LOG_DESTINATION' in a form \verb'address:port'.

Because user is allowed to change ACL or add user tags only for her jobs, paths to valid X509 user credentials has to be set to authorise her. This is done using two environmental variables \verb'EDG_WL_X509_KEY' and \verb'EDG_WL_X509_CERT' in a form \verb'path_to_cred'.



\newpage

\subsection{\LB\ Web Service Interface}

The \LB\ web service interface currently reflects the functionality of legacy
\LB\ query API (Sect.~\ref{query-C}). 

The following sections describe the operations defined in the \LB\ WSDL
file as well as its custom types.

For the sake of readability this documentation does not follow the structure
of WSDL strictly, avoiding to duplicate information which is already present
here.
Conseqently, the SOAP messages are not documented, for example, as they
are derived from operation inputs and outputs mechanically.
The same holds for types: \eg\ we do not document defined elements 
which correspond 1:1 to types but are required due to the literal SOAP
encoding.

For exact definition of the operations and types see the WSDL file.


{
\def\chapter#1{}
\def\section#1{\subsubsection{#1}}
\def\subsection#1{\par\textbf{#1}\par}

\let\odesc=\description
\let\oedesc=\enddescription
\renewenvironment{description}{\odesc\itemindent=1em
\listparindent=2em
}{\oedesc}
%\renewenvironment{description}{\list{}{\labelwidth 5cm\leftmargin 5cm}}
%{\endlist}

\let\null=\relax
%\input LB-ug
}


\newpage
\section{Known Problems and Caveats}
\subsection{Indexed attributes}\label{ConsIndx}
\LB\ queries can be fairly complicated and they can potentially return large 
amounts of data. In such cases, there is a~risk of server overload.
In order to prevent the overload, every \LB\ server configuration
includes a~set of configurable \emph{indices}, empty by default
(job ID is always indexed).
In general, at least one indexed attribute must be
present in each query\,---\,queries over only attributes without indices
are refused.

Unfortunately, in some cases queries that may be considered very important from
the user's point of view (like all user's jobs) fall into this category. 
Indexing specific attributes makes these queries tractable
(while increasing the risk of server overload).
It's left up to the server administrator to decide which query types are
supported by configuring indices over attribute sets of desired queries using
the \texttt{glite-lb-bkindex} utility.
I.\,e.\ if the attribute \texttt{EDG\_WLL\_QUERY\_ATTR\_OWNER} is not indexed,
some of 
queries in the presented examples 
% odkazuje na totez :-( (\eg\ Sect.'s~\ref{JQ-auj},\ref{JQ-rj})
(Sect.'s~\ref{JQ-auj})
are forbidden.

\subsection{Timeouts}

All blocking \LB\ API calls are subject to timeouts. Timeout values can be changed 
from their respective defaults with \texttt{edg\_wll\_SetParam} 
call or using variables of the process environment (floating point number of seconds).

\begin{tabular}{lcll}
affected calls&default(max)&parameter&env. variable\\
async. logging&120 (1800)&\texttt{EDG\_WLL\_PARAM\_LOG\_TIMEOUT}&\texttt{EDG\_WL\_LOG\_TIMEOUT}\\
sync. logging&120 (1800)&\texttt{EDG\_WLL\_PARAM\_LOG\_SYNC\_TIMEOUT}&\texttt{EDG\_WL\_LOG\_SYNC\_TIMEOUT}\\
queries&120 (1800)&\texttt{EDG\_WLL\_PARAM\_QUERY\_TIMEOUT}&\texttt{EDG\_WL\_QUERY\_TIMEOUT}\\
notifications&120 (1800)&\texttt{EDG\_WLL\_PARAM\_NOTIF\_TIMEOUT}&\texttt{EDG\_WL\_NOTIF\_TIMEOUT}\\
\end{tabular}

\subsection{Timestamps}
Timestamps of \LB\ event is recorded by the process calling a~logging API call. It is strongly recommended
to keep clocks of all systems that produce \LB\ events (UI, WM, WN) in reasonable synchronization so that
timestamp attributes of events and job status can be interpreted easily.

On the other hand, timestamps are not authoritative when \LB\ events
are sorted in order to compute job state\Dash a~more robust mechanism
of the hierarchical \LB\ sequence code is used instead. 
Consequently, strict timestamp sorting of events coming from desynchronised
sources may give different (incorrect) results from what \LB\ reports in job
state.

\subsection{Size limitations}

Current implementation of \LB\ has a~few built-in size limits, mostly related
to schema of underlaying MySQL database. By default, the limits are:

\begin{tabular}{lc}
item & maximum size\\
user certificate subject&255 bytes\\
event attribute (JDL etc.) &16 megabytes\\
tag name&200 bytes\\
tag value&255 bytes\\
job ACL&16 megabytes\\
notification destination&200 bytes\\
\end{tabular}

The restriction on tag value is twofold.
Values up to 16\,MB may be logged and retrieved as raw events.
However, when reported in job state longer values are truncated to 255 bytes.
The restriction is inherited from MySQL limit on index size.

\subsection{Running startup scripts}

Startup scripts of the \LB\ service daemons should not be called without preconditions 
being satisfied, \eg\ one should not try to start a~service when it is already running
or try to restart it from a~cron job without checking whether the machine is shutting down
at the time.

\subsection{Dependencies}

It is strongly discouraged to use \LB\ with different revisions
of external dependencies (MySQL, Globus Toolkit)
than those described in the release documentation. While the \LB\ has been designed to achieve
reasonable backward and forward compatibility, some disruptive changes in minor revisions
of external software have been observed before.



\begin{thebibliography}1
\bibitem[R1]{lbapi}\emph{\LB\ API Reference}, DataGrid-01-TED-0139.
\bibitem[R2]{lbarch}\emph{\LB\ Architecture release 2}, DataGrid-01-TED-0141.
\bibitem[R3]{WMS} G.Avellino at al., \emph{The DataGrid Workload Management System: Challenges and Results}, Journal of Grid Computing, ISSN: 1570-7873, 2005, accepted.
\end{thebibliography}

\clearpage

\appendix

\section{Component and Interaction Diagrams}
% \todo{nekdy priste}
% To help understanding the service a set of component and interaction
% diagrams might help. This section is not mandatory. 


\begin{figure}[h]
\centering
\includegraphics[width=.8\hsize]{logging-arch-notif}
\caption{Overview of the \LB\ architecture}
\label{fig-arch}
\end{figure}




\end{document}
