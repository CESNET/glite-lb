\section{Introduction}
\TODO{Add the information from the "Best practices" workshop, i.e. information how the whole LB is coded, why C, what were the general techniques - context, connection pool, .T templates description, etc.}

\subsection{Return values}
The return type of most of the API functions is return \verb'int'.
Unless specified otherwise the returned values are as follows:
\begin{description}
\item[0] success
\item[errno] an error occured, the nature of the error
matches meaning of one of standard \verb'<errno.h>' error codes.
\item[\LB specific] a~few errors can't be intuitively mapped to
\verb'<errno.h>', therefore the are specific \LB error codes, starting from \verb'EDG_WLL_ERROR_BASE'.
\end{description}
See Sect.~\ref{s:error} for details on error handling.

Few API function return \verb'char *'. In such a~case \verb'NULL' indicates
an error, non-null value means success.

\subsection{Function arguments}
In the following description the function arguments are classified as follows:
\begin{description}
\item[IN] pure input argument, not modified by the function.
If it is a~pointer, the C prototype includes \verb'const' (which is omitted in
this document for the sake of readability).
\item[OUT] pure output argument. The function expects a~pointer and fills in
the pointed object.

If the argument is \verb'**', or a~structure containing pointers,
the returned objects are \emph{always} dynamically allocated,
and has to be freed when not used anymore.
The same holds for directly returned pointers.

\item[INOUT] an argument taken as an input and modified by the function.
Typically it is the \LB context.

\end{description}

\subsubsection{Datatypes}
The API uses two classes of custom datatypes:
\begin{description}
\item[opaque types] (\eg context, jobid) do not expose their structure
to the user.
The only way to access them is via the API functions.
The internal structure of those may be subject to change.

\item[transparent types] (\eg event, status) expose the structure to the
user. The structure is documented and no incompatible changes will be done
without notice.
\end{description}

\subsection{Context and Parameter Settings}
\label{s:context}

All the API functions refer to a~\emph{context} argument.
The context type \verb'edg_wll_Context' is opaque.
Context objects preserve various status information
(\eg\ connection to server) among the API calls.
The API caller can create many context objects, those are guaranteed
to be independent on one another.
In this way thread-safety of the library is achieved --
one context object has to be used by only one thread at time.

Context is used to set and keep various parameters of the library
(\eg\ default server addresses, timeouts, \dots).
Upon initialization, all the parameters are assigned default values.
However, many of the parameters take their default value from environment
variables. If the corresponding environment variable is set,
the parameter is initialized to its value instead of the default.
Note that a~few parameters cannot be assigned default value; consequently
setting them either in environment or with an explicit API call
is mandatory before using the appropriate part of the API.

The context also stores details on errors of the recent API call.

For use with the \emph{producer} calls (see Sect.~\ref{s:producer})
the context has to be assigned a~single \jobid
(with the \verb'edg_wll_SetLoggingJob' call, see Sect.~\ref{s:sequence}),
and keeps track of an event \emph{sequence code} for the job 
(detailed description in~\cite{lbarch}).
