\documentclass{egee}
\usepackage{comment}

\def\LB{L\&B}

\title{\LB\ Test Plan}
\author{CESNET EGEE JRA1 team}
\DocIdentifier{EGEE-JRA1-??}
\Date{\today}
\Activity{JRA1: Middleware Engineering and Integration}
\DocStatus{DRAFT}
\Dissemination{PUBLIC}
\DocumentLink{}

\def\req{\noindent\textbf{Prerequisities:}}
\def\how{\noindent\textbf{How to run:}}
\def\result{\noindent\textbf{Expected result:}}

\def\path#1{\textsf{#1}}
\def\code#1{\textsf{#1}}


\specialcomment{hints}{\par\noindent\textbf{Hints: }\begingroup\itshape}{\endgroup}
%\includecomment{hints}

\begin{document}

\begin{center}
{\bf Delivery Slip}
\end{center}
\begin{tabularx}{\textwidth}{|l|l|l|X|X|}
\hline
           & {\bf Name} & {\bf Partner} & {\bf Date} & {\bf Signature} \\
\hline
{\bf From} & Ale\v{s} K\v{r}enek & CESNET & August 1, 2008 & \\
\hline
{\bf Reviewed by} & &  & & \\

\hline
{\bf Approved by} & & & & \\
\hline
\end{tabularx}

\begin{center}
{\bf Document Change Log}
\end{center}

\begin{tabularx}{\textwidth}{|l|l|X|X|}
\hline
{\bf Issue } & {\bf Date  } & {\bf Comment } & {\bf Author  } \\   \hline
1.0.0 & August 1, 2008 & Initial version & CESNET team \\

\hline
\end{tabularx}

\begin{center}
{\bf Document Change Record}
\end{center}

\begin{tabularx}{\textwidth}{|l|l|X|}
\hline
{\bf Issue } & {\bf Item  } & {\bf Reason for Change } \\   \hline

\hline
\end{tabularx}

%
% Official text received on October 6, 2004
%
\vfill{\bf Copyright }\copyright{\bf Members of the EGEE Collaboration. 2004. 
See http://eu-egee.org/partners for details on the copyright holders. 

EGEE (``Enabling Grids for E-science in Europe'') is a project funded by
the European Union.  For more information on the project, its partners
and contributors please see http://www.eu-egee.org.

You are permitted to copy and distribute verbatim copies of this
document containing this copyright notice, but modifying this document
is not allowed. You are permitted to copy this document in whole or in
part into other documents if you attach the following reference to the
copied elements: ``Copyright }\copyright{\bf 2004. Members of the EGEE
Collaboration. http://www.eu-egee.org''

The information contained in this document represents the views of
EGEE as of the date they are published. EGEE does not guarantee that
any information contained herein is error-free, or up to date.

EGEE MAKES NO WARRANTIES, EXPRESS, IMPLIED, OR STATUTORY, BY
PUBLISHING THIS DOCUMENT.}


\clearpage

\newpage
\tableofcontents
\newpage

\section{Test Cases}

\subsection{Event delivery}

% locallogger
% bez dalsich demonu, registrovat job, vrati EAGAIN, objevi se fajly
\subsubsection{Standalone locallogger -- job registration}
\label{reg}
\req\ running \path{glite-lb-logd} on the test node, don't start either
\path{glite-lb-interlogd} or \path{glite-lb-bkserverd}

\how\ call \code{edg\_wll\_RegisterJob}. Jobid's should preferably point
to a~remote \LB\ server.

\result\ The API call returns EAGAIN, but locallogger creates an event file
in its storage.
The file should contain single line RegJob event.

\begin{hints}
\path{glite-lb-regjob} example can be used. It generates a~unique jobid,
prints it and calls \LB\ API appropriately.
\end{hints}

% async -- prida do fajlu, OK
% logevent
\subsubsection{Standalone locallogger -- log event}
\label{log}
\req\ running \path{glite-lb-logd} only, files generated in test~\ref{reg}.

\how\ call \code{edg\_wll\_Log*} for various event types in a~sequence
resebmling real \LB\ usage, using the same jobid's as in test~\ref{reg}

\result\ API calls return 0, events are added one per line to the locallogger files

\begin{hints}
\path{glite-lb-logev} client program can be used.

\path{glite-lb-*.sh} examples may be adapted to produce reasonable seqences
of events.
\end{hints}

\subsubsection{Interlogger recovery}
\label{recover}
% recover interloggeru
% il & server (remote)
% spustit, protlaci soubory na server, soubory zmizi, lze se dotazat na stav
\req\ running \path{glite-lb-bkserverd} on the machine and port where 
jobid's from \ref{reg} point to; files generated in~\ref{log}.

\how\ Make a~copy of the files created in~\ref{log}, then start
\path{glite-lb-interlogd}. After approx. 10s check the jobs
with \code{edg\_wll\_JobLog} call. 

\result \code{edg\_wll\_JobLog} should return the same events that were
contained in the locallogger files. The files should be removed by 
interlogger after approx. 1 min.

\begin{hints}
\path{glite-lb-joblog} example outputs the events in (almost) the same
format as the locallogger files.
\end{hints}

% event delivery
% poslat .sh, job log vrati to, co bylo ve fajlech
\subsubsection{Normal event delivery}
\label{normal}
\req\ all \LB\ daemons running (\path{glite-lb-logd}, \path{glite-lb-interlogd},
\path{glite-lb-bkserverd}

\how\ Register jobs with \code{edg\_wll\_RegsterJob} and log reasonable
sequences of events with \code{edg\_wll\_Log*}.
Check with \code{edg\_wll\_JobLog}
that the events got delivered afterwards (approx. 10s).

\result\ API calls should return 0. The same events that were logged must be returned.

\begin{hints}
\path{glite-lb-*.sh} scripts produce reasonable seqences of events, including
the job initial registration.

There is approx. 1min time window in which the locallogger files exist.
They can be grabbed and used for comparing the events as in~\ref{recover}.

\end{hints}

\subsection{Job state computation}

% normal event delivery & job state machine
% .sh, dotaz na stav
\subsubsection{Normal job states}
\label{state}
\req\ \path{glite-lb-bkserverd} running, events from \ref{normal} logged.

\how\ Check state of the jobs with \code{edg\_wll\_JobStatus}.

\result\ The API call should return 0, the jobs should be in the expected
states. Thorough tests may also cross check the values supplied in the
events (e.g. destination computing element) wrt. the values reported in the job states.

\begin{hints}
\path{glite-lb-*.sh} scripts produce sequences of events resultning
in the job state same as the `*' part of the script name.
\end{hints}


\subsubsection{DAG job states}
\textbf{TODO}

% specialni stav DAGu, histogram potomku

\subsection{LB proxy}
\textbf{TODO}
% proxy -- honik

\subsection{Notifications}

% notifikace 
% regjob, reg notifikace na vsechno, poslat udalosti, hlidat notif

% rozsireni dotazu o dalsi job

% notifikace -- zmena adresy/portu
% pak poslat udalost, musi dojit

% notifikace -- zpozdene doruceni
% registrovat, odpojit, poslat udalosti, pripojit se

\subsection{Server purge}

% purge
% test_purge


\end{document}
